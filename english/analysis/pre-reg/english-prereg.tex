\documentclass[11pt,]{article}
\usepackage{lmodern}
\usepackage{amssymb,amsmath}
\usepackage{ifxetex,ifluatex}
\usepackage{fixltx2e} % provides \textsubscript
\ifnum 0\ifxetex 1\fi\ifluatex 1\fi=0 % if pdftex
  \usepackage[T1]{fontenc}
  \usepackage[utf8]{inputenc}
\else % if luatex or xelatex
  \ifxetex
    \usepackage{mathspec}
  \else
    \usepackage{fontspec}
  \fi
  \defaultfontfeatures{Ligatures=TeX,Scale=MatchLowercase}
    \setmainfont[]{DejaVu Sans}
\fi
% use upquote if available, for straight quotes in verbatim environments
\IfFileExists{upquote.sty}{\usepackage{upquote}}{}
% use microtype if available
\IfFileExists{microtype.sty}{%
\usepackage{microtype}
\UseMicrotypeSet[protrusion]{basicmath} % disable protrusion for tt fonts
}{}
\usepackage[margin=1in]{geometry}
\usepackage{hyperref}
\hypersetup{unicode=true,
            pdftitle={Pre-registration for Compensatory aspects of the effect of voicing on vowel duration in English},
            pdfauthor={Stefano Coretta},
            pdfborder={0 0 0},
            breaklinks=true}
\urlstyle{same}  % don't use monospace font for urls
\usepackage{natbib}
\bibliographystyle{unified.bst}
\usepackage{color}
\usepackage{fancyvrb}
\newcommand{\VerbBar}{|}
\newcommand{\VERB}{\Verb[commandchars=\\\{\}]}
\DefineVerbatimEnvironment{Highlighting}{Verbatim}{commandchars=\\\{\}}
% Add ',fontsize=\small' for more characters per line
\usepackage{framed}
\definecolor{shadecolor}{RGB}{248,248,248}
\newenvironment{Shaded}{\begin{snugshade}}{\end{snugshade}}
\newcommand{\AlertTok}[1]{\textcolor[rgb]{0.94,0.16,0.16}{#1}}
\newcommand{\AnnotationTok}[1]{\textcolor[rgb]{0.56,0.35,0.01}{\textbf{\textit{#1}}}}
\newcommand{\AttributeTok}[1]{\textcolor[rgb]{0.77,0.63,0.00}{#1}}
\newcommand{\BaseNTok}[1]{\textcolor[rgb]{0.00,0.00,0.81}{#1}}
\newcommand{\BuiltInTok}[1]{#1}
\newcommand{\CharTok}[1]{\textcolor[rgb]{0.31,0.60,0.02}{#1}}
\newcommand{\CommentTok}[1]{\textcolor[rgb]{0.56,0.35,0.01}{\textit{#1}}}
\newcommand{\CommentVarTok}[1]{\textcolor[rgb]{0.56,0.35,0.01}{\textbf{\textit{#1}}}}
\newcommand{\ConstantTok}[1]{\textcolor[rgb]{0.00,0.00,0.00}{#1}}
\newcommand{\ControlFlowTok}[1]{\textcolor[rgb]{0.13,0.29,0.53}{\textbf{#1}}}
\newcommand{\DataTypeTok}[1]{\textcolor[rgb]{0.13,0.29,0.53}{#1}}
\newcommand{\DecValTok}[1]{\textcolor[rgb]{0.00,0.00,0.81}{#1}}
\newcommand{\DocumentationTok}[1]{\textcolor[rgb]{0.56,0.35,0.01}{\textbf{\textit{#1}}}}
\newcommand{\ErrorTok}[1]{\textcolor[rgb]{0.64,0.00,0.00}{\textbf{#1}}}
\newcommand{\ExtensionTok}[1]{#1}
\newcommand{\FloatTok}[1]{\textcolor[rgb]{0.00,0.00,0.81}{#1}}
\newcommand{\FunctionTok}[1]{\textcolor[rgb]{0.00,0.00,0.00}{#1}}
\newcommand{\ImportTok}[1]{#1}
\newcommand{\InformationTok}[1]{\textcolor[rgb]{0.56,0.35,0.01}{\textbf{\textit{#1}}}}
\newcommand{\KeywordTok}[1]{\textcolor[rgb]{0.13,0.29,0.53}{\textbf{#1}}}
\newcommand{\NormalTok}[1]{#1}
\newcommand{\OperatorTok}[1]{\textcolor[rgb]{0.81,0.36,0.00}{\textbf{#1}}}
\newcommand{\OtherTok}[1]{\textcolor[rgb]{0.56,0.35,0.01}{#1}}
\newcommand{\PreprocessorTok}[1]{\textcolor[rgb]{0.56,0.35,0.01}{\textit{#1}}}
\newcommand{\RegionMarkerTok}[1]{#1}
\newcommand{\SpecialCharTok}[1]{\textcolor[rgb]{0.00,0.00,0.00}{#1}}
\newcommand{\SpecialStringTok}[1]{\textcolor[rgb]{0.31,0.60,0.02}{#1}}
\newcommand{\StringTok}[1]{\textcolor[rgb]{0.31,0.60,0.02}{#1}}
\newcommand{\VariableTok}[1]{\textcolor[rgb]{0.00,0.00,0.00}{#1}}
\newcommand{\VerbatimStringTok}[1]{\textcolor[rgb]{0.31,0.60,0.02}{#1}}
\newcommand{\WarningTok}[1]{\textcolor[rgb]{0.56,0.35,0.01}{\textbf{\textit{#1}}}}
\usepackage{graphicx,grffile}
\makeatletter
\def\maxwidth{\ifdim\Gin@nat@width>\linewidth\linewidth\else\Gin@nat@width\fi}
\def\maxheight{\ifdim\Gin@nat@height>\textheight\textheight\else\Gin@nat@height\fi}
\makeatother
% Scale images if necessary, so that they will not overflow the page
% margins by default, and it is still possible to overwrite the defaults
% using explicit options in \includegraphics[width, height, ...]{}
\setkeys{Gin}{width=\maxwidth,height=\maxheight,keepaspectratio}
\IfFileExists{parskip.sty}{%
\usepackage{parskip}
}{% else
\setlength{\parindent}{0pt}
\setlength{\parskip}{6pt plus 2pt minus 1pt}
}
\setlength{\emergencystretch}{3em}  % prevent overfull lines
\providecommand{\tightlist}{%
  \setlength{\itemsep}{0pt}\setlength{\parskip}{0pt}}
\setcounter{secnumdepth}{5}
% Redefines (sub)paragraphs to behave more like sections
\ifx\paragraph\undefined\else
\let\oldparagraph\paragraph
\renewcommand{\paragraph}[1]{\oldparagraph{#1}\mbox{}}
\fi
\ifx\subparagraph\undefined\else
\let\oldsubparagraph\subparagraph
\renewcommand{\subparagraph}[1]{\oldsubparagraph{#1}\mbox{}}
\fi

%%% Use protect on footnotes to avoid problems with footnotes in titles
\let\rmarkdownfootnote\footnote%
\def\footnote{\protect\rmarkdownfootnote}

%%% Change title format to be more compact
\usepackage{titling}

% Create subtitle command for use in maketitle
\newcommand{\subtitle}[1]{
  \posttitle{
    \begin{center}\large#1\end{center}
    }
}

\setlength{\droptitle}{-2em}

  \title{Pre-registration for Compensatory aspects of the effect of voicing on
vowel duration in English}
    \pretitle{\vspace{\droptitle}\centering\huge}
  \posttitle{\par}
    \author{Stefano Coretta}
    \preauthor{\centering\large\emph}
  \postauthor{\par}
      \predate{\centering\large\emph}
  \postdate{\par}
    \date{08/01/2019}

\usepackage{cleveref}

\begin{document}
\maketitle

\hypertarget{study-information}{%
\section{Study information}\label{study-information}}

\hypertarget{title}{%
\subsection{Title}\label{title}}

Compensatory aspects of the effect of voicing on vowel duration in
English.

\hypertarget{authorship}{%
\subsection{Authorship}\label{authorship}}

Stefano Coretta (The University of Manchester).

\hypertarget{research-questions}{%
\subsection{Research questions}\label{research-questions}}

Vowels in English (and other languages) are longer when followed by
voiced stops than when followed by voiceless stops \citep{heffner1937}.
The origin of this `voicing effect' is still object to debate. According
to the compensatory temporal adjustment account, the difference in vowel
duration is the consequence of the difference in stop closure durations
\citep{lindblom1967, slis1969a, slis1969, lehiste1970, lehiste1970a}.
Voiceless stops have longer closures than voiced stops
\citep{lisker1957, van-summers1987, davis1989, de-jong1991}, so that
vowels are longer when followed by the shorter closure durations of
voiced stops.

An exploratory study discussed in \citet{coretta2018j} indicates that
the duration of the interval between two consecutive stops releases is
not affected by consonant voicing in CV́CV words in Italian and Polish.
Coretta argues that the temporal stability of the release to release
interval plus the difference in consonant closure durations drives the
differences in vowel durations by a mechanism of compensation between
the duration of the vowel and that of the closure.

This study sets out to test whether English exhibits the same temporal
pattern found in Italian and Polish. The aim of the study is to answer
the following questions:

\begin{itemize}
\tightlist
\item
  Q1: Is the duration of the interval between two consecutive stop
  releases (the release to release interval) in monosyllabic and
  disyllabic words affected by the voicing of C2 in English?
\item
  Q2: Is the duration of the release to release interval affected by (a)
  the number of syllables of the word, (b) the quality of V1, and (c)
  the place of C2?
\end{itemize}

A third question pertains to differences between mono- and disyllabic
words:

\begin{itemize}
\tightlist
\item
  Q3: What is the estimated difference in the effect of voicing on vowel
  and stop closure duration between monosyllabic and disyllabic words?
\end{itemize}

\hypertarget{hypotheses}{%
\subsection{Hypotheses}\label{hypotheses}}

In relation to Q1, the following hypotheses will be tested:

\begin{itemize}
\tightlist
\item
  H1a: The duration of the release to release interval is not affected
  by C2 voicing in disyllabic words.
\item
  H1b: The release to release interval is longer in monosyllabic words
  with a voiced C2 than in monosyllabic words with a voiceless C2.
\end{itemize}

H1b is proposed based on \citet{jacewicz2009}, who report that
monosyllabic words are longer when C2 is voiced in American English.

The exploratory study of Italian and Polish indicates that the intrinsic
duration of the vowel and the consonant contribute to the duration of
the release to release interval. More specifically, the release to
release with a high vowel is shorter than that with a low vowel. It is
well known that cross-linguistically high vowels tend to be shorter than
low vowels
\citep{hertrich1997, esposito2002, mortensen2013, toivonen2015, kawahara2017}.
As for consonant place of articulation, if the consonant is velar the
interval is shorter. The closure of velar stops is shorter than that of
labial stops (see for example \citealt{sharf1962}). It is possible that
some of the durational difference of the release to release interval
seen in the exploratory study depend on intrinsic vowel and consonant
duration.

Considering the relation between intrinsic vowel and consonant duration
and release to release, we can formulate the following hypotheses in
relation to Q2 above:

\begin{itemize}
\tightlist
\item
  H2a: The release to release interval is longer in monosyllabic than in
  disyllabic words.
\item
  H2b: The duration of the release to release interval decreases
  according to the hierarchy /ɑː/, /əː/, /iː/.
\item
  H2c: The release to release interval is shorter when C2 is velar.
\end{itemize}

As for Q3:

\begin{itemize}
\tightlist
\item
  H3: The effect of voicing on vowel duration is greater in monosyllabic
  than in disyllabic words.
\end{itemize}

There is no specific hypothesis concerning stop closures.

\hypertarget{sampling-plan}{%
\section{Sampling plan}\label{sampling-plan}}

\hypertarget{existing-data}{%
\subsection{Existing data}\label{existing-data}}

\textbf{Registration prior to creation of data}: As of the date of
submission of this research plan for preregistration, the data have not
yet been collected, created, or realised.

\hypertarget{explanation-of-existing-data}{%
\subsection{Explanation of existing
data}\label{explanation-of-existing-data}}

NA.

\hypertarget{data-collection-procedures}{%
\subsection{Data collection
procedures}\label{data-collection-procedures}}

\hypertarget{participants}{%
\subsubsection{Participants}\label{participants}}

Inclusion rule: Native speakers of English from the Manchester area, 18+
yo, with no reported hearing or speaking disorders, and with normal or
corrected to normal vision.

\hypertarget{procedure}{%
\subsubsection{Procedure}\label{procedure}}

\label{s:procedure}

The participants will be recorded while reading sentences with CVC and
CV́CVC target words presented on a computer screen with PsychoPy
\citep{peirce2009}, in a sound attenuated room in the Phonetics
Laboratory at the University of Manchester. The test words are
C\textsubscript{1}V\textsubscript{1}C\textsubscript{2}(VC) words, where
C\textsubscript{1} = /t/, V\textsubscript{1} = /iː, əː, ɑː/,
C\textsubscript{2} = /p, b, k, g/, and (VC) = /əs/. This structure leads
to 24 possible combinations.

\begin{tabular}{llll}
teep & teepus & teek & teekus \\
teeb & teebus & teeg & teegus \\
terp & terpus & terk & terkus \\
terb & terbus & terg & tergus \\
tarp & tarpus & tark & tarkus \\
tarb & tarbus & targ & targus \\
\end{tabular}

Each target word is combined with the following frame sentences: I'll
say X this Thursday, You'll say X this Monday, She'll say X this Sunday,
We'll say X this Friday, They'll say X this Tuesday. Each word/frame
combination will be read once. Acoustic recording will be obtained with
a Zoom H4n Pro recorder and a RØDE Lavalier microphone, at a sample rate
of 44100 Hz (16-bit; downsampled to 22050 Hz for analysis). The
recordings will be subject to force alignment with SPPAS
\citep{bigi2015} for analysis in Praat \citep{boersma2018}.

\hypertarget{sample-size}{%
\subsection{Sample size}\label{sample-size}}

24 words, 5 repetitions (= 120 tokens per participant). We set a minimum
of 20 participants with additional participants if needed (see following
sections).

\hypertarget{sample-size-rationale}{%
\subsection{Sample size rationale}\label{sample-size-rationale}}

The approach of the region of practical equivalence (ROPE) will be used
for determining the participant sample size
\citep{kruschke2015, vasishth2018a}. This approach is based on
establishing a region around 0 which could theoretically be interpreted
as a `no effect' region. Based on estimates from the literature on the
just noticeable difference \citep{huggins1972, nooteboom1980}, the
chosen ROPE width is 20 milliseconds (from -10 to +10 milliseconds).

\hypertarget{stopping-rule}{%
\subsection{Stopping rule}\label{stopping-rule}}

Data collection will stop either:

\begin{enumerate}
\def\labelenumi{(\alph{enumi})}
\tightlist
\item
  When the width of the 95\% CI of the effect of C2 voicing will be less
  than 20 milliseconds, or
\item
  If the ROPE target hasn't been reached by April 30th 2019, due to time
  constraints imposed on the project.
\end{enumerate}

\hypertarget{variables}{%
\section{Variables}\label{variables}}

\hypertarget{manipulated-variables}{%
\subsection{Manipulated variables}\label{manipulated-variables}}

\label{s:manipulated}

\begin{itemize}
\tightlist
\item
  \textbf{Vowel}: /iː/, /əː/, /ɑː/.
\item
  \textbf{Voicing of C2}: voiceless (/p, k/), voiced (/b, g/).
\item
  \textbf{Place of articulation of C2}: labial (/p, b/), velar (/k, g/).
\item
  \textbf{Number of syllables}: monosyllabic, disyllabic.
\item
  \textbf{Frame sentence}: I'll say X this Thursday, You'll say X this
  Monday, She'll say X this Sunday, We'll say X this Friday, They'll say
  X this Tuesday.
\end{itemize}

See \Cref{s:procedure} for a list of word stimuli.

\hypertarget{measured-variables}{%
\subsection{Measured variables}\label{measured-variables}}

\begin{itemize}
\tightlist
\item
  From the acoustic signal:

  \begin{itemize}
  \tightlist
  \item
    \textbf{Duration of the release to release interval}: from the
    release of C1 to the release of C2.
  \item
    \textbf{V1 duration}: from appearance to disappearance of higher
    formant structure in the spectrogram in correspondence of V1
    \citep{machac2009}.
  \item
    \textbf{C2 closure duration}: from disappearance of higher formant
    structure in the V1C2 sequence to the release of C2
    \citep{machac2009}.
  \item
    \textbf{Speech rate}: calculated as number of syllables per second
    (number of syllables in the sentence divided by the sentence
    duration in seconds, \citealt{plug2018}).
  \end{itemize}
\end{itemize}

C1/C2 release detection will be performed with an automatic procedure in
Praat based on \citet{ananthapadmanabha2014}. The output of the
automatic force-alignment (see \Cref{s:procedure}) and the release
detection algorithm will be checked and corrected manually if needed.

\hypertarget{indices}{%
\subsection{Indices}\label{indices}}

NA.

\hypertarget{design-plan}{%
\section{Design plan}\label{design-plan}}

\hypertarget{study-type}{%
\subsection{Study type}\label{study-type}}

Experiment---A researcher randomly assigns treatments to study subjects,
this includes field or lab experiments. This is also known as an
intervention experiment and includes randomised controlled trials.

\hypertarget{blinding}{%
\subsection{Blinding}\label{blinding}}

No blinding is involved in this study.

\hypertarget{study-design}{%
\subsection{Study design}\label{study-design}}

Repeated measures, mixed design.

\hypertarget{randomisation}{%
\subsection{Randomisation}\label{randomisation}}

The sentence stimuli will be randomised within participant by means of
the built-in randomisation procedure in PsychoPy \citep{peirce2009}.

\hypertarget{analysis-plan}{%
\section{Analysis plan}\label{analysis-plan}}

\hypertarget{statistical-models}{%
\subsection{Statistical models}\label{statistical-models}}

\label{s:stats}

Bayesian linear mixed models
\citep{vasishth2018, mcelreath2015, kruschke2015} will be fitted with
brms \citep{burkner2017, burkner2018} in R \citep{r-core-team2018}.

\hypertarget{release-to-release}{%
\subsubsection{Release to release}\label{release-to-release}}

The following Bayesian regression model will be used to model the
duration of the release to release interval. As fixed effects: C2
voicing (factor, levels = `voiceless', `voiced'), number of syllables
(factor, levels = `disyllabic', `monosyllabic'), centred speech rate,
interaction between C2 voicing and number of syllables. Factors are
coded with treatment contrasts. A by-speaker and by-word random
intercept, and a by-speaker random coefficient for C2 voicing.

\begin{Shaded}
\begin{Highlighting}[]
\NormalTok{rr_}\DecValTok{1}\NormalTok{ <-}\StringTok{ }\KeywordTok{brm}\NormalTok{(}
\NormalTok{  rr }\OperatorTok{~}
\StringTok{    }\NormalTok{c2_voice }\OperatorTok{+}
\StringTok{    }\NormalTok{n_syl }\OperatorTok{+}
\StringTok{    }\NormalTok{c2_voice}\OperatorTok{:}\NormalTok{n_syl }\OperatorTok{+}
\StringTok{    }\NormalTok{speech_rate_c }\OperatorTok{+}
\StringTok{    }\NormalTok{(}\DecValTok{1} \OperatorTok{+}\StringTok{ }\NormalTok{c2_voice }\OperatorTok{|}\StringTok{ }\NormalTok{speaker) }\OperatorTok{+}
\StringTok{    }\NormalTok{(}\DecValTok{1} \OperatorTok{|}\StringTok{ }\NormalTok{word),}
  \DataTypeTok{family =} \KeywordTok{gaussian}\NormalTok{()}
\NormalTok{)}
\end{Highlighting}
\end{Shaded}

The following priors will be used. For the intercept of the release to
release interval duration, a normal distribution with mean 200 ms and SD
= 50, based on the posterior distribution of the intercept in the
Italian/Polish exploratory study. For the effect of C2 voicing, a weakly
informative prior as a normal distribution with mean 0 ms and SD = 25,
based on results from the exploratory study. For the effect of number of
syllables, a weakly informative prior as a normal distribution with mean
50 ms and SD = 25, based on differences in vowel duration between mono-
and disyllabic words, which range between 30 and 100 ms
\citep{sharf1962, klatt1973}. The same prior is chosen for the
interaction of C2 voicing and number of syllables, based on a similar
range of reported differences in vowel duration in monosyllabic words
(30-100 ms). For the effect of speech rate (centred), a normal
distribution with mean -25 and SD = 10, based on results from the
exploratory study. For the random effects, a half Cauchy distribution
(location = 0, scale = 25) for the standard deviation and a LKJ(2)
distribution for the correlation. For the residual standard deviation, a
half Cauchy distribution with location 0 ms and scale 25.

\begin{Shaded}
\begin{Highlighting}[]
\KeywordTok{c}\NormalTok{(}
  \KeywordTok{set_prior}\NormalTok{(}\StringTok{"normal(200, 50)"}\NormalTok{, }\DataTypeTok{class =} \StringTok{"Intercept"}\NormalTok{),}
  \KeywordTok{set_prior}\NormalTok{(}\StringTok{"normal(0, 25)"}\NormalTok{, }\DataTypeTok{class =} \StringTok{"b"}\NormalTok{, }\DataTypeTok{coef =} \StringTok{"c2_voicevoiced"}\NormalTok{),}
  \KeywordTok{set_prior}\NormalTok{(}\StringTok{"normal(50, 25)"}\NormalTok{, }\DataTypeTok{class =} \StringTok{"b"}\NormalTok{, }\DataTypeTok{coef =} \StringTok{"n_sylmono"}\NormalTok{),}
  \KeywordTok{set_prior}\NormalTok{(}\StringTok{"normal(50, 25)"}\NormalTok{, }\DataTypeTok{class =} \StringTok{"b"}\NormalTok{, }\DataTypeTok{coef =} \StringTok{"c2_voicevoiced:n_sylmono"}\NormalTok{),}
  \KeywordTok{set_prior}\NormalTok{(}\StringTok{"normal(-25, 10)"}\NormalTok{, }\DataTypeTok{class =} \StringTok{"b"}\NormalTok{, }\DataTypeTok{coef =} \StringTok{"speech_rate_c"}\NormalTok{),}
  \KeywordTok{set_prior}\NormalTok{(}\StringTok{"cauchy(0, 25)"}\NormalTok{, }\DataTypeTok{class =} \StringTok{"sd"}\NormalTok{),}
  \KeywordTok{set_prior}\NormalTok{(}\StringTok{"lkj(2)"}\NormalTok{, }\DataTypeTok{class =} \StringTok{"cor"}\NormalTok{),}
  \KeywordTok{set_prior}\NormalTok{(}\StringTok{"cauchy(0, 25)"}\NormalTok{, }\DataTypeTok{class =} \StringTok{"sigma"}\NormalTok{)}
\NormalTok{)}
\end{Highlighting}
\end{Shaded}

A separate model will test the effect of vowel and C2 place on the
duration of the release to release duration. As fixed effects: vowel
(factor, levels = `ee', `er', `ar'), C2 place (factor, levels =
`labial', `velar'), centred speech rate, interaction between vowel and
C2 place. Factors are coded with treatment contrasts. By-speaker and
by-word random intercepts.

\begin{Shaded}
\begin{Highlighting}[]
\NormalTok{rr_}\DecValTok{2}\NormalTok{ <-}\StringTok{ }\KeywordTok{brm}\NormalTok{(}
\NormalTok{  rr }\OperatorTok{~}
\StringTok{    }\NormalTok{vowel }\OperatorTok{+}
\StringTok{    }\NormalTok{c2_place }\OperatorTok{+}
\StringTok{    }\NormalTok{speech_rate_c }\OperatorTok{+}
\StringTok{    }\NormalTok{(}\DecValTok{1} \OperatorTok{|}\StringTok{ }\NormalTok{speaker) }\OperatorTok{+}
\StringTok{    }\NormalTok{(}\DecValTok{1} \OperatorTok{|}\StringTok{ }\NormalTok{word),}
  \DataTypeTok{family =} \KeywordTok{gaussian}\NormalTok{()}
\NormalTok{)}
\end{Highlighting}
\end{Shaded}

The following priors will be used. For the intercept of the release to
release interval duration, a normal distribution with mean 200 ms and SD
= 50, based on the posterior distribution of the intercept in the
Italian/Polish exploratory study. For the effect of vowel, a weakly
informative prior as a normal distribution with mean 0 ms and SD = 30,
based on differences in vowel duration reported in \citet{heffner1937},
\citet{house1953}, and \citet{hertrich1997}. For the effect of C2 place
of articulation, a weakly informative prior as a normal distribution
with mean 0 ms and SD = 30, based on differences in closure durations
between labial and velar stops, which range between 10 and 30 ms
\citep{sharf1962}. The remaining priors are the same as the ones of the
previous model.

\begin{Shaded}
\begin{Highlighting}[]
\KeywordTok{c}\NormalTok{(}
  \KeywordTok{set_prior}\NormalTok{(}\StringTok{"normal(200, 50)"}\NormalTok{, }\DataTypeTok{class =} \StringTok{"Intercept"}\NormalTok{),}
  \KeywordTok{set_prior}\NormalTok{(}\StringTok{"normal(0, 30)"}\NormalTok{, }\DataTypeTok{class =} \StringTok{"b"}\NormalTok{, }\DataTypeTok{coef =} \StringTok{"voweler"}\NormalTok{),}
  \KeywordTok{set_prior}\NormalTok{(}\StringTok{"normal(0, 30)"}\NormalTok{, }\DataTypeTok{class =} \StringTok{"b"}\NormalTok{, }\DataTypeTok{coef =} \StringTok{"vowelar"}\NormalTok{),}
  \KeywordTok{set_prior}\NormalTok{(}\StringTok{"normal(0, 30)"}\NormalTok{, }\DataTypeTok{class =} \StringTok{"b"}\NormalTok{, }\DataTypeTok{coef =} \StringTok{"c2_placevelar"}\NormalTok{),}
  \KeywordTok{set_prior}\NormalTok{(}\StringTok{"normal(-25, 10)"}\NormalTok{, }\DataTypeTok{class =} \StringTok{"b"}\NormalTok{, }\DataTypeTok{coef =} \StringTok{"speech_rate_c"}\NormalTok{),}
  \KeywordTok{set_prior}\NormalTok{(}\StringTok{"cauchy(0, 25)"}\NormalTok{, }\DataTypeTok{class =} \StringTok{"sd"}\NormalTok{),}
  \KeywordTok{set_prior}\NormalTok{(}\StringTok{"cauchy(0, 25)"}\NormalTok{, }\DataTypeTok{class =} \StringTok{"sigma"}\NormalTok{)}
\NormalTok{)}
\end{Highlighting}
\end{Shaded}

\hypertarget{vowel-duration}{%
\subsubsection{Vowel duration}\label{vowel-duration}}

The following Bayesian regression model will be used to model the
duration of V1. As fixed effects: C2 voicing (factor, levels =
`voiceless', `voiced'), vowel (factor, levels = `ee', `er', `ar'),
number of syllables (factor, levels = `disyllabic', `monosyllabic'),
centred speech rate, all logical interactions between C2 voicing, vowel,
and number of syllables. Factors are coded with treatment contrasts. A
by-speaker and by-word random intercept, and a by-speaker random
coefficient for C2 voicing.

\begin{Shaded}
\begin{Highlighting}[]
\NormalTok{vow_}\DecValTok{1}\NormalTok{ <-}\StringTok{ }\KeywordTok{brm}\NormalTok{(}
\NormalTok{  v1_duration }\OperatorTok{~}
\StringTok{    }\NormalTok{c2_voice }\OperatorTok{+}
\StringTok{    }\NormalTok{vowel }\OperatorTok{+}
\StringTok{    }\NormalTok{n_syl }\OperatorTok{+}
\StringTok{    }\NormalTok{c2_voice}\OperatorTok{:}\NormalTok{vowel }\OperatorTok{+}
\StringTok{    }\NormalTok{c2_voice}\OperatorTok{:}\NormalTok{n_syl }\OperatorTok{+}
\StringTok{    }\NormalTok{vowel}\OperatorTok{:}\NormalTok{n_syl }\OperatorTok{+}
\StringTok{    }\NormalTok{c2_voice}\OperatorTok{:}\NormalTok{vowel}\OperatorTok{:}\NormalTok{n_syl }\OperatorTok{+}
\StringTok{    }\NormalTok{speech_rate_c }\OperatorTok{+}
\StringTok{    }\NormalTok{(}\DecValTok{1} \OperatorTok{+}\StringTok{ }\NormalTok{c2_voice }\OperatorTok{|}\StringTok{ }\NormalTok{speaker) }\OperatorTok{+}
\StringTok{    }\NormalTok{(}\DecValTok{1} \OperatorTok{|}\StringTok{ }\NormalTok{word),}
  \DataTypeTok{family =} \KeywordTok{gaussian}\NormalTok{()}
\NormalTok{)}
\end{Highlighting}
\end{Shaded}

The following priors will be used.

\begin{Shaded}
\begin{Highlighting}[]
\KeywordTok{c}\NormalTok{(}
  \KeywordTok{set_prior}\NormalTok{(}\StringTok{"normal(145, 30)"}\NormalTok{, }\DataTypeTok{class =} \StringTok{"Intercept"}\NormalTok{),}
  \KeywordTok{set_prior}\NormalTok{(}\StringTok{"normal(50, 20)"}\NormalTok{, }\DataTypeTok{class =} \StringTok{"b"}\NormalTok{, }\DataTypeTok{coef =} \StringTok{"c2_voicevoiced"}\NormalTok{),}
  \KeywordTok{set_prior}\NormalTok{(}\StringTok{"normal(0, 30)"}\NormalTok{, }\DataTypeTok{class =} \StringTok{"b"}\NormalTok{, }\DataTypeTok{coef =} \StringTok{"voweler"}\NormalTok{),}
  \KeywordTok{set_prior}\NormalTok{(}\StringTok{"normal(0, 30)"}\NormalTok{, }\DataTypeTok{class =} \StringTok{"b"}\NormalTok{, }\DataTypeTok{coef =} \StringTok{"vowelar"}\NormalTok{),}
  \KeywordTok{set_prior}\NormalTok{(}\StringTok{"normal(50, 25)"}\NormalTok{, }\DataTypeTok{class =} \StringTok{"b"}\NormalTok{, }\DataTypeTok{coef =} \StringTok{"n_sylmono"}\NormalTok{),}
  \KeywordTok{set_prior}\NormalTok{(}\StringTok{"normal(0, 20)"}\NormalTok{, }\DataTypeTok{class =} \StringTok{"b"}\NormalTok{, }\DataTypeTok{coef =} \StringTok{"c2_voicevoiced:voweler"}\NormalTok{),}
  \KeywordTok{set_prior}\NormalTok{(}\StringTok{"normal(0, 20)"}\NormalTok{, }\DataTypeTok{class =} \StringTok{"b"}\NormalTok{, }\DataTypeTok{coef =} \StringTok{"c2_voicevoiced:vowelar"}\NormalTok{),}
  \KeywordTok{set_prior}\NormalTok{(}\StringTok{"normal(50, 25)"}\NormalTok{, }\DataTypeTok{class =} \StringTok{"b"}\NormalTok{, }\DataTypeTok{coef =} \StringTok{"c2_voicevoiced:n_sylmono"}\NormalTok{),}
  \KeywordTok{set_prior}\NormalTok{(}\StringTok{"normal(0, 30)"}\NormalTok{, }\DataTypeTok{class =} \StringTok{"b"}\NormalTok{, }\DataTypeTok{coef =} \StringTok{"voweler:n_sylmono"}\NormalTok{),}
  \KeywordTok{set_prior}\NormalTok{(}\StringTok{"normal(0, 30)"}\NormalTok{, }\DataTypeTok{class =} \StringTok{"b"}\NormalTok{, }\DataTypeTok{coef =} \StringTok{"vowelar:n_sylmono"}\NormalTok{),}
  \KeywordTok{set_prior}\NormalTok{(}\StringTok{"normal(0, 30)"}\NormalTok{, }\DataTypeTok{class =} \StringTok{"b"}\NormalTok{, }\DataTypeTok{coef =} \StringTok{"c2_voicevoiced:voweler:n_sylmono"}\NormalTok{),}
  \KeywordTok{set_prior}\NormalTok{(}\StringTok{"normal(0, 30)"}\NormalTok{, }\DataTypeTok{class =} \StringTok{"b"}\NormalTok{, }\DataTypeTok{coef =} \StringTok{"c2_voicevoiced:vowelar:n_sylmono"}\NormalTok{),}
  \KeywordTok{set_prior}\NormalTok{(}\StringTok{"normal(-25, 10)"}\NormalTok{, }\DataTypeTok{class =} \StringTok{"b"}\NormalTok{, }\DataTypeTok{coef =} \StringTok{"speech_rate_c"}\NormalTok{),}
  \KeywordTok{set_prior}\NormalTok{(}\StringTok{"cauchy(0, 25)"}\NormalTok{, }\DataTypeTok{class =} \StringTok{"sd"}\NormalTok{),}
  \KeywordTok{set_prior}\NormalTok{(}\StringTok{"lkj(2)"}\NormalTok{, }\DataTypeTok{class =} \StringTok{"cor"}\NormalTok{),}
  \KeywordTok{set_prior}\NormalTok{(}\StringTok{"cauchy(0, 25)"}\NormalTok{, }\DataTypeTok{class =} \StringTok{"sigma"}\NormalTok{)}
\NormalTok{)}
\end{Highlighting}
\end{Shaded}

\hypertarget{closure-duration}{%
\subsubsection{Closure duration}\label{closure-duration}}

\label{s:closure}

The following Bayesian regression model will be used to model the
duration of the closure of C2. As fixed effects: C2 voicing (factor,
levels = `voiceless', `voiced'), vowel (factor, levels = `ee', `er',
`ar'), number of syllables (factor, levels = `disyllabic',
`monosyllabic'), centred speech rate, all logical interactions between
C2 voicing, vowel, and number of syllables. Factors are coded with
treatment contrasts. A by-speaker and by-word random intercept, and a
by-speaker random coefficient for C2 voicing.

\begin{Shaded}
\begin{Highlighting}[]
\NormalTok{c2_}\DecValTok{1}\NormalTok{ <-}\StringTok{ }\KeywordTok{brm}\NormalTok{(}
\NormalTok{  c2_duration }\OperatorTok{~}
\StringTok{    }\NormalTok{c2_voice }\OperatorTok{+}
\StringTok{    }\NormalTok{c2_place }\OperatorTok{+}
\StringTok{    }\NormalTok{n_syl }\OperatorTok{+}
\StringTok{    }\NormalTok{c2_voice}\OperatorTok{:}\NormalTok{c2_place }\OperatorTok{+}
\StringTok{    }\NormalTok{c2_voice}\OperatorTok{:}\NormalTok{n_syl }\OperatorTok{+}
\StringTok{    }\NormalTok{vowel}\OperatorTok{:}\NormalTok{n_syl }\OperatorTok{+}
\StringTok{    }\NormalTok{c2_voice}\OperatorTok{:}\NormalTok{c2_place}\OperatorTok{:}\NormalTok{n_syl }\OperatorTok{+}
\StringTok{    }\NormalTok{speech_rate_c }\OperatorTok{+}
\StringTok{    }\NormalTok{(}\DecValTok{1} \OperatorTok{+}\StringTok{ }\NormalTok{c2_voice }\OperatorTok{|}\StringTok{ }\NormalTok{speaker) }\OperatorTok{+}
\StringTok{    }\NormalTok{(}\DecValTok{1} \OperatorTok{|}\StringTok{ }\NormalTok{word),}
  \DataTypeTok{family =} \KeywordTok{gaussian}\NormalTok{()}
\NormalTok{)}
\end{Highlighting}
\end{Shaded}

The following priors will be used. The means reported in
\citet{sharf1962} and \citet{luce1985} indicate that stop closures in
monosyllabic words are 10-30 ms shorter when the stop is voiced. A
normal distribution with mean -20 ms and SD = 10 seems informative
enough as a prior to the effect of C2 voicing on closure duration. The
same studies indicate velar stops have a closure which is 10-20ms
shorter in monosyllabic words, hence the prior
\texttt{normal(mean\ =\ -15,\ SD\ =\ 10)}. Note that the estimates for
C2 voicing and place refer to monosyllabic words, while the specified
priors refer to disyllabic words.

\begin{Shaded}
\begin{Highlighting}[]
\KeywordTok{c}\NormalTok{(}
  \KeywordTok{set_prior}\NormalTok{(}\StringTok{"normal(90, 20)"}\NormalTok{, }\DataTypeTok{class =} \StringTok{"Intercept"}\NormalTok{),}
  \KeywordTok{set_prior}\NormalTok{(}\StringTok{"normal(-20, 10)"}\NormalTok{, }\DataTypeTok{class =} \StringTok{"b"}\NormalTok{, }\DataTypeTok{coef =} \StringTok{"c2_voicevoiced"}\NormalTok{),}
  \KeywordTok{set_prior}\NormalTok{(}\StringTok{"normal(-15, 10)"}\NormalTok{, }\DataTypeTok{class =} \StringTok{"b"}\NormalTok{, }\DataTypeTok{coef =} \StringTok{"c2_placevelar"}\NormalTok{),}
  \KeywordTok{set_prior}\NormalTok{(}\StringTok{"normal(0, 25)"}\NormalTok{, }\DataTypeTok{class =} \StringTok{"b"}\NormalTok{, }\DataTypeTok{coef =} \StringTok{"n_sylmono"}\NormalTok{),}
  \KeywordTok{set_prior}\NormalTok{(}\StringTok{"normal(0, 10)"}\NormalTok{, }\DataTypeTok{class =} \StringTok{"b"}\NormalTok{, }\DataTypeTok{coef =} \StringTok{"c2_voicevoiced:c2_placevelar"}\NormalTok{),}
  \KeywordTok{set_prior}\NormalTok{(}\StringTok{"normal(0, 10)"}\NormalTok{, }\DataTypeTok{class =} \StringTok{"b"}\NormalTok{, }\DataTypeTok{coef =} \StringTok{"c2_voicevoiced:n_sylmono"}\NormalTok{),}
  \KeywordTok{set_prior}\NormalTok{(}\StringTok{"normal(0, 30)"}\NormalTok{, }\DataTypeTok{class =} \StringTok{"b"}\NormalTok{, }\DataTypeTok{coef =} \StringTok{"c2_placevelar:n_sylmono"}\NormalTok{),}
  \KeywordTok{set_prior}\NormalTok{(}\StringTok{"normal(0, 30)"}\NormalTok{, }\DataTypeTok{class =} \StringTok{"b"}\NormalTok{, }\DataTypeTok{coef =} \StringTok{"c2_placevelar:n_sylmono"}\NormalTok{),}
  \KeywordTok{set_prior}\NormalTok{(}\StringTok{"normal(0, 30)"}\NormalTok{, }\DataTypeTok{class =} \StringTok{"b"}\NormalTok{, }\DataTypeTok{coef =} \StringTok{"c2_voicevoiced:c2_placevelar:n_sylmono"}\NormalTok{),}
  \KeywordTok{set_prior}\NormalTok{(}\StringTok{"normal(0, 30)"}\NormalTok{, }\DataTypeTok{class =} \StringTok{"b"}\NormalTok{, }\DataTypeTok{coef =} \StringTok{"c2_voicevoiced:c2_placevelar:n_sylmono"}\NormalTok{),}
  \KeywordTok{set_prior}\NormalTok{(}\StringTok{"normal(-25, 10)"}\NormalTok{, }\DataTypeTok{class =} \StringTok{"b"}\NormalTok{, }\DataTypeTok{coef =} \StringTok{"speech_rate_c"}\NormalTok{),}
  \KeywordTok{set_prior}\NormalTok{(}\StringTok{"cauchy(0, 25)"}\NormalTok{, }\DataTypeTok{class =} \StringTok{"sd"}\NormalTok{),}
  \KeywordTok{set_prior}\NormalTok{(}\StringTok{"lkj(2)"}\NormalTok{, }\DataTypeTok{class =} \StringTok{"cor"}\NormalTok{),}
  \KeywordTok{set_prior}\NormalTok{(}\StringTok{"cauchy(0, 25)"}\NormalTok{, }\DataTypeTok{class =} \StringTok{"sigma"}\NormalTok{)}
\NormalTok{)}
\end{Highlighting}
\end{Shaded}

\hypertarget{transformations}{%
\subsection{Transformations}\label{transformations}}

Speech rate will be centred to make the estimate of the intercept
interpretable. Centring will be obtained with the standard formula
(speech rate - mean speech rate).

\hypertarget{follow-up-analyses}{%
\subsection{Follow-up analyses}\label{follow-up-analyses}}

NA.

\hypertarget{inference-criteria}{%
\subsection{Inference criteria}\label{inference-criteria}}

Bayesian posterior distributions, rather than point estimates, will be
employed for inference.

\hypertarget{data-exclusion}{%
\subsection{Data exclusion}\label{data-exclusion}}

Residual data with mistakes and speech errors will be discarded.

\hypertarget{missing-data}{%
\subsection{Missing data}\label{missing-data}}

Individual missing observations (due to excluded data or annotation
difficulties) will be dropped.

\hypertarget{exploratory-analysis}{%
\subsection{Exploratory analysis}\label{exploratory-analysis}}

An exploratory analysis will look into differences in closure durations
of voiceless and voiced stops in mono- and disyllabic words (see
\Cref{s:closure}).

\hypertarget{script-optional}{%
\section{Script (Optional)}\label{script-optional}}

\hypertarget{analysis-scripts-optional}{%
\subsection{Analysis scripts
(Optional)}\label{analysis-scripts-optional}}

See \Cref{s:stats}.

\hypertarget{other}{%
\section{Other}\label{other}}

NA.

\bibliography{linguistics.bib}


\end{document}
