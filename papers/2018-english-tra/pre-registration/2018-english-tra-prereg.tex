\documentclass[11pt,]{article}
\usepackage{lmodern}
\usepackage{amssymb,amsmath}
\usepackage{ifxetex,ifluatex}
\usepackage{fixltx2e} % provides \textsubscript
\ifnum 0\ifxetex 1\fi\ifluatex 1\fi=0 % if pdftex
  \usepackage[T1]{fontenc}
  \usepackage[utf8]{inputenc}
\else % if luatex or xelatex
  \ifxetex
    \usepackage{mathspec}
  \else
    \usepackage{fontspec}
  \fi
  \defaultfontfeatures{Ligatures=TeX,Scale=MatchLowercase}
    \setmainfont[]{DejaVu Sans}
\fi
% use upquote if available, for straight quotes in verbatim environments
\IfFileExists{upquote.sty}{\usepackage{upquote}}{}
% use microtype if available
\IfFileExists{microtype.sty}{%
\usepackage{microtype}
\UseMicrotypeSet[protrusion]{basicmath} % disable protrusion for tt fonts
}{}
\usepackage[margin=1in]{geometry}
\usepackage{hyperref}
\hypersetup{unicode=true,
            pdftitle={Pre-registration of a study on tongue root advancement},
            pdfauthor={Stefano Coretta},
            pdfborder={0 0 0},
            breaklinks=true}
\urlstyle{same}  % don't use monospace font for urls
\usepackage{natbib}
\bibliographystyle{unified.bst}
\usepackage{color}
\usepackage{fancyvrb}
\newcommand{\VerbBar}{|}
\newcommand{\VERB}{\Verb[commandchars=\\\{\}]}
\DefineVerbatimEnvironment{Highlighting}{Verbatim}{commandchars=\\\{\}}
% Add ',fontsize=\small' for more characters per line
\usepackage{framed}
\definecolor{shadecolor}{RGB}{248,248,248}
\newenvironment{Shaded}{\begin{snugshade}}{\end{snugshade}}
\newcommand{\KeywordTok}[1]{\textcolor[rgb]{0.13,0.29,0.53}{\textbf{#1}}}
\newcommand{\DataTypeTok}[1]{\textcolor[rgb]{0.13,0.29,0.53}{#1}}
\newcommand{\DecValTok}[1]{\textcolor[rgb]{0.00,0.00,0.81}{#1}}
\newcommand{\BaseNTok}[1]{\textcolor[rgb]{0.00,0.00,0.81}{#1}}
\newcommand{\FloatTok}[1]{\textcolor[rgb]{0.00,0.00,0.81}{#1}}
\newcommand{\ConstantTok}[1]{\textcolor[rgb]{0.00,0.00,0.00}{#1}}
\newcommand{\CharTok}[1]{\textcolor[rgb]{0.31,0.60,0.02}{#1}}
\newcommand{\SpecialCharTok}[1]{\textcolor[rgb]{0.00,0.00,0.00}{#1}}
\newcommand{\StringTok}[1]{\textcolor[rgb]{0.31,0.60,0.02}{#1}}
\newcommand{\VerbatimStringTok}[1]{\textcolor[rgb]{0.31,0.60,0.02}{#1}}
\newcommand{\SpecialStringTok}[1]{\textcolor[rgb]{0.31,0.60,0.02}{#1}}
\newcommand{\ImportTok}[1]{#1}
\newcommand{\CommentTok}[1]{\textcolor[rgb]{0.56,0.35,0.01}{\textit{#1}}}
\newcommand{\DocumentationTok}[1]{\textcolor[rgb]{0.56,0.35,0.01}{\textbf{\textit{#1}}}}
\newcommand{\AnnotationTok}[1]{\textcolor[rgb]{0.56,0.35,0.01}{\textbf{\textit{#1}}}}
\newcommand{\CommentVarTok}[1]{\textcolor[rgb]{0.56,0.35,0.01}{\textbf{\textit{#1}}}}
\newcommand{\OtherTok}[1]{\textcolor[rgb]{0.56,0.35,0.01}{#1}}
\newcommand{\FunctionTok}[1]{\textcolor[rgb]{0.00,0.00,0.00}{#1}}
\newcommand{\VariableTok}[1]{\textcolor[rgb]{0.00,0.00,0.00}{#1}}
\newcommand{\ControlFlowTok}[1]{\textcolor[rgb]{0.13,0.29,0.53}{\textbf{#1}}}
\newcommand{\OperatorTok}[1]{\textcolor[rgb]{0.81,0.36,0.00}{\textbf{#1}}}
\newcommand{\BuiltInTok}[1]{#1}
\newcommand{\ExtensionTok}[1]{#1}
\newcommand{\PreprocessorTok}[1]{\textcolor[rgb]{0.56,0.35,0.01}{\textit{#1}}}
\newcommand{\AttributeTok}[1]{\textcolor[rgb]{0.77,0.63,0.00}{#1}}
\newcommand{\RegionMarkerTok}[1]{#1}
\newcommand{\InformationTok}[1]{\textcolor[rgb]{0.56,0.35,0.01}{\textbf{\textit{#1}}}}
\newcommand{\WarningTok}[1]{\textcolor[rgb]{0.56,0.35,0.01}{\textbf{\textit{#1}}}}
\newcommand{\AlertTok}[1]{\textcolor[rgb]{0.94,0.16,0.16}{#1}}
\newcommand{\ErrorTok}[1]{\textcolor[rgb]{0.64,0.00,0.00}{\textbf{#1}}}
\newcommand{\NormalTok}[1]{#1}
\usepackage{graphicx,grffile}
\makeatletter
\def\maxwidth{\ifdim\Gin@nat@width>\linewidth\linewidth\else\Gin@nat@width\fi}
\def\maxheight{\ifdim\Gin@nat@height>\textheight\textheight\else\Gin@nat@height\fi}
\makeatother
% Scale images if necessary, so that they will not overflow the page
% margins by default, and it is still possible to overwrite the defaults
% using explicit options in \includegraphics[width, height, ...]{}
\setkeys{Gin}{width=\maxwidth,height=\maxheight,keepaspectratio}
\IfFileExists{parskip.sty}{%
\usepackage{parskip}
}{% else
\setlength{\parindent}{0pt}
\setlength{\parskip}{6pt plus 2pt minus 1pt}
}
\setlength{\emergencystretch}{3em}  % prevent overfull lines
\providecommand{\tightlist}{%
  \setlength{\itemsep}{0pt}\setlength{\parskip}{0pt}}
\setcounter{secnumdepth}{5}
% Redefines (sub)paragraphs to behave more like sections
\ifx\paragraph\undefined\else
\let\oldparagraph\paragraph
\renewcommand{\paragraph}[1]{\oldparagraph{#1}\mbox{}}
\fi
\ifx\subparagraph\undefined\else
\let\oldsubparagraph\subparagraph
\renewcommand{\subparagraph}[1]{\oldsubparagraph{#1}\mbox{}}
\fi

%%% Use protect on footnotes to avoid problems with footnotes in titles
\let\rmarkdownfootnote\footnote%
\def\footnote{\protect\rmarkdownfootnote}

%%% Change title format to be more compact
\usepackage{titling}

% Create subtitle command for use in maketitle
\newcommand{\subtitle}[1]{
  \posttitle{
    \begin{center}\large#1\end{center}
    }
}

\setlength{\droptitle}{-2em}

  \title{Pre-registration of a study on tongue root advancement}
    \pretitle{\vspace{\droptitle}\centering\huge}
  \posttitle{\par}
    \author{Stefano Coretta}
    \preauthor{\centering\large\emph}
  \postauthor{\par}
      \predate{\centering\large\emph}
  \postdate{\par}
    \date{05/10/2018}

\usepackage{cleveref}

\begin{document}
\maketitle

\section{Study information}\label{study-information}

\subsection{Title}\label{title}

\subsection{Authorship}\label{authorship}

Stefano Coretta (The University of Manchester, UK), Steven M. Lulich
(Indiana University, USA).

\subsection{Research questions}\label{research-questions}

An exploratory study of articulatory properties of link between
consonant voicing and vowel duration \citep{coretta2018f, coretta2018d}
showed that tongue root advancement is initiated and executed during the
production of vowels preceding both voiced and voiceless stops in
Italian and
Polish.\footnote{Tongue root displacement data was gathered from the stressed vowel in CV́CV words. C1 is /p/, C2 is either /t, d, k, g/, and V1 = V2 = /a, o, u/.}
\citet{coretta2018d} proposes that root advancement might have
originally been a mechanical consequence of tongue body (dorsum/tip)
raising, then evolutionarily coopted for cavity enlargement in the
context of voiced stops.

It is also known that during the production of labial stops (both
voiceless and voiced) the tongue body lowers
\citep{vazquez-alvarez2007}. A cooptation mechanism parallel to that of
tongue root advancement might have operated with tongue body lowering in
the context of voiced labial stops.

The following research questions derive from the cooptation hypothesis:

\begin{enumerate}
\def\labelenumi{\arabic{enumi}.}
\tightlist
\item
  Does the tongue root advances during vowels before voiceless and
  voiced velar stops?
\item
  Does the tongue body lowers during vowels before voiceless and voiced
  labial stops?
\item
  Are tongue root advancement and tongue body dorsum mechanically
  linked?
\item
  What is the relationship of timing and velocity of the advancing
  gesture in vowels before voiceless and voiced velar stop contexts?
\end{enumerate}

\subsection{Hypotheses}\label{hypotheses}

The following specific hypotheses are tested:

\begin{enumerate}
\def\labelenumi{\arabic{enumi}.}
\item
  The tongue root advances during the production of vowels before
  voiceless and voiced velar stops.
\item
  The tongue body lowers during the production of vowels before
  voiceless and voiced velar stops.
\item
  Tongue root advancement and tongue dorsum raising are synchronous.
\item
\end{enumerate}

\section{Sampling plan}\label{sampling-plan}

\subsection{Existing data}\label{existing-data}

\subsection{Explanation of existing
data}\label{explanation-of-existing-data}

\subsection{Data collection
procedures}\label{data-collection-procedures}

\subsubsection{Participants}\label{participants}

Inclusion rule: Native speakers of American English, 18+ yo, with no
reported hearing or speaking disorders, with normal or corrected to
normal vision.

\subsubsection{Procedure}\label{procedure}

Ultrasound data will be collected with a Philips EPIQ 7G system using an
xMatrix 117 x6-1 digital 3D/4D transducer. An Articulate Instrument
headset will be used for probe stabilisation. Audio data will be
recorded with a SHURE KSM32 microphone (sample rate 48 kHz, 16-bit). The
participant will read the target words (\Cref{s:manipulated}) embedded
in a frame sentence (\emph{I said a ``X'' again}).

\subsection{Sample size}\label{sample-size}

20 participants, 8 words, 10 repetitions. Grand total: 1600
observations.

\subsection{Sample size rationale}\label{sample-size-rationale}

There is a general consensus in ultrasound tongue imaging studies that a
number of participants between 10 and 20 is acceptable, given the time
resources necessary for processing the data.

\subsection{Stopping rule}\label{stopping-rule}

Data collection will be terminated if the 20 participants target is not
reached by the October 29th 2018.

\section{Variables}\label{variables}

\subsection{Manipulated variables}\label{manipulated-variables}

\label{s:manipulated}

\begin{itemize}
\tightlist
\item
  \textbf{Place of articulation}: velar (/k, g/), labial (/p, b/)
\item
  \textbf{Voicing}: voiceless (/k, p/), voiced (/g, b/)
\item
  \textbf{Syllables}: monosyllabic, disyllabic
\end{itemize}

The word stimuli:

\begin{tabular}{ll}
pop & pob  \\
caulk & cog \\
popper & pobber \\
cocker & cogger \\
\end{tabular}

\subsection{Measured variables}\label{measured-variables}

\begin{itemize}
\tightlist
\item
  \textbf{Tongue root displacement} during the vowel (TRD): displacement
  of the tongue root along a vector line measured along the entire
  duration of the vowel (in mm)
\item
  \textbf{Tongue dorsum displacement} during the vowel (TDD):
  displacement of the tongue dorsum along a vector line measured along
  the entire duration of the vowel (in mm)
\item
  \textbf{Tongue body displacement} during the vowel (TBD): displacement
  of the tongue body along a vector line measured along the entire
  duration of the vowel (in mm)
\item
  \textbf{Gesture onset of tongue root advancement} (TR-GONS):
  difference between the time of maximum tongue dorsum displacement of
  C1 and the time of tongue root gesture onset during V1 (in ms)
\item
  \textbf{Gesture onset of tongue dorsum raising} (TD-GONS): difference
  between the time of maximum tongue dorsum displacement of C1 and the
  time of tongue root gesture onset during V1 (in ms)
\item
  \textbf{TR/TD GONS difference}: difference of TR-GONS and TD-GONS (in
  ms)
\item
  \textbf{TR velocity}: first derivative of tongue root displacement
  during the vowel
\item
  \textbf{Speech rate}: calculated as syllables per second
  \citep[\texttt{n\ of\ syllables\ /\ sentence\ duration},][]{plug2018}
\end{itemize}

Tongue dynamic measures are obtained following the methods described in
\citet{strycharczuk2015}.

\subsection{Indices}\label{indices}

NA.

\section{Design plan}\label{design-plan}

\subsection{Study type}\label{study-type}

Experiment---A researcher randomly assigns treatments to study subjects,
this includes field or lab experiments. This is also known as an
intervention experiment and includes randomized controlled trials.

\subsection{Blinding}\label{blinding}

No blinding is involved in this study.

\subsection{Study design}\label{study-design}

Repeated measures, mixed design.

\subsection{Randomisation}\label{randomisation}

The word stimuli will be randomised across each block and across
speakers.

\section{Analysis plan}\label{analysis-plan}

\subsection{Statistical models}\label{statistical-models}

\label{s:stats}

\begin{Shaded}
\begin{Highlighting}[]
\KeywordTok{bam}\NormalTok{(}
\NormalTok{  trd }\OperatorTok{~}
\StringTok{    }\NormalTok{voicing_place }\OperatorTok{+}
\StringTok{    }\KeywordTok{s}\NormalTok{(proportion, }\DataTypeTok{bs =} \StringTok{"cr"}\NormalTok{) }\OperatorTok{+}
\StringTok{    }\KeywordTok{s}\NormalTok{(proportion, }\DataTypeTok{bs =} \StringTok{"cr"}\NormalTok{, }\DataTypeTok{by =}\NormalTok{ voicing_place) }\OperatorTok{+}
\StringTok{    }\KeywordTok{s}\NormalTok{(proportion, speaker, }\DataTypeTok{bs =} \StringTok{"fs"}\NormalTok{, }\DataTypeTok{m =} \DecValTok{1}\NormalTok{)}
\NormalTok{)}
\end{Highlighting}
\end{Shaded}

\begin{Shaded}
\begin{Highlighting}[]
\KeywordTok{bam}\NormalTok{(}
\NormalTok{  tbd }\OperatorTok{~}
\StringTok{    }\NormalTok{voicing_place }\OperatorTok{+}
\StringTok{    }\KeywordTok{s}\NormalTok{(proportion, }\DataTypeTok{bs =} \StringTok{"cr"}\NormalTok{) }\OperatorTok{+}
\StringTok{    }\KeywordTok{s}\NormalTok{(proportion, }\DataTypeTok{bs =} \StringTok{"cr"}\NormalTok{, }\DataTypeTok{by =}\NormalTok{ voicing_place) }\OperatorTok{+}
\StringTok{    }\KeywordTok{s}\NormalTok{(proportion, speaker, }\DataTypeTok{bs =} \StringTok{"fs"}\NormalTok{, }\DataTypeTok{m =} \DecValTok{1}\NormalTok{)}
\NormalTok{)}
\end{Highlighting}
\end{Shaded}

\begin{Shaded}
\begin{Highlighting}[]
\NormalTok{priors <-}\StringTok{ }\KeywordTok{c}\NormalTok{(}
  \KeywordTok{set_prior}\NormalTok{(}\StringTok{"normal(0, ...)"}\NormalTok{, }\DataTypeTok{class =} \StringTok{"Intercept"}\NormalTok{),}
  \KeywordTok{set_prior}\NormalTok{(}\StringTok{"normal(0, ...)"}\NormalTok{, }\DataTypeTok{class =} \StringTok{"b"}\NormalTok{, }\DataTypeTok{coef =} \StringTok{"syllablesdisyllabic"}\NormalTok{),}
  \KeywordTok{set_prior}\NormalTok{(}\StringTok{"normal(0, ...)"}\NormalTok{, }\DataTypeTok{class =} \StringTok{"b"}\NormalTok{, }\DataTypeTok{coef =} \StringTok{"speech_rate"}\NormalTok{),}
  \KeywordTok{set_prior}\NormalTok{(}\StringTok{"normal(0, ...)"}\NormalTok{, }\DataTypeTok{class =} \StringTok{"sd"}\NormalTok{),}
  \KeywordTok{set_prior}\NormalTok{(}\StringTok{"normal(0, ...)"}\NormalTok{, }\DataTypeTok{class =} \StringTok{"sigma"}\NormalTok{),}
  \KeywordTok{set_prior}\NormalTok{(}\StringTok{"lkj(2)"}\NormalTok{, }\DataTypeTok{class =} \StringTok{"cor"}\NormalTok{)}
\NormalTok{)}

\KeywordTok{brms}\NormalTok{(}
\NormalTok{  tr_td }\OperatorTok{~}
\StringTok{    }\NormalTok{syllables }\OperatorTok{+}
\StringTok{    }\NormalTok{speech_rate }\OperatorTok{+}
\StringTok{    }\NormalTok{(}\DecValTok{1}\OperatorTok{|}\NormalTok{speaker) }\OperatorTok{+}
\StringTok{    }\NormalTok{(}\DecValTok{1}\OperatorTok{|}\NormalTok{word),}
  \DataTypeTok{prior =}\NormalTok{ priors}
\NormalTok{)}
\end{Highlighting}
\end{Shaded}

\subsection{Transformations}\label{transformations}

NA.

\subsection{Follow-up analyses}\label{follow-up-analyses}

NA.

\subsection{Inference criteria}\label{inference-criteria}

\emph{P}-values will be used for the effect of vowel height on vowel
duration, VOT, and RVoffT, with \(\alpha\) = 0.05. \emph{P}-values below
0.05 will be deemed significant. Bayes factors will be employed for the
effect of vowel height on voicing interval duration, since I am
interested in testing both the null and the alternative hypotheses. The
recommendations in \citet[139]{raftery1995} for the interpretation of
the Bayes factors will be followed.

\subsection{Data exclusion}\label{data-exclusion}

\subsection{Missing data}\label{missing-data}

NA.

\subsection{Exploratory analysis}\label{exploratory-analysis}

\section{Script (Optional)}\label{script-optional}

\subsection{Analysis scripts
(Optional)}\label{analysis-scripts-optional}

See \Cref{s:stats}.

\section{Other}\label{other}

\label{s:other}

\bibliography{linguistics.bib}


\end{document}
