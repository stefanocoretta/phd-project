%%%%%%%%%%%%%%%%%%%%%%%%%%%%%%%%%%%%%%%%%%%%%%%%%%
%	JASA LaTeX Template File
%  For use in making articles using JASAnew.cls
% July 26, 2017
%%%%%%%%%%%%%%%%%%%%%%%%%%%%%%%%%%%%%%%%%%%%%%%%%%

%% Step 1:
%% Uncomment the style that you want to use:

%%%%%%% For Preprint
%% For manuscript, 12pt, one column style

%% Comment this out if you'd rather use another style:
\documentclass[preprint]{JASAnew}

%%%%% Preprint Options %%%%%
%% The track changes option allows you to mark changes
%% and will produce a list of changes, their line number
%% and page number at the end of the article.
%\documentclass[preprint,trackchanges]{JASAnew}

%% authaffil option will make affil immediately
% follow author, otherwise authors are grouped, and affiliations
% are stacked underneath all the authors.
%\documentclass[preprint,authaffil]{JASAnew}

%% NumberedRefs is used for numbered bibliography and citations.
%% Default is Author-Year style.
%% \documentclass[preprint,NumberedRefs]{JASAnew}

%%%%%%% For Reprint
%% For appearance of finished article; 2 columns, 10 pt fonts

% \documentclass[reprint]{JASAnew}

%%%%% Reprint Options %%%%%

%% For testing to see if author has exceeded page length request, use 12pt option
%\documentclass[reprint,12pt]{JASAnew}

% authaffil option will make affil immediately
% follow author, otherwise authors are grouped, and affiliations
% are stacked underneath all the authors.
%\documentclass[reprint,authaffil]{JASAnew}

%% NumberedRefs is used for numbered bibliography and citations.
%% Default is Author-Year style.
% \documentclass[reprint,NumberedRefs]{JASAnew}

%% TurnOnLineNumbers
%% Make lines be numbered in reprint style:
% \documentclass[reprint,TurnOnLineNumbers]{JASAnew}

\usepackage{fontspec}

    \setmainfont[]{Times New Roman}

\usepackage{natbib}



\begin{document}
%% the square bracket argument will send term to running head in
%% preprint, or running foot in reprint style.

\title[A subtitle goes on another line]{This is a title and this is too}

% ie
%\title[JASA/Sample JASA Article]{Sample JASA Article}

%% repeat as needed

\author{Stefano Coretta}
% ie
%\affiliation{Department1,  University1, City, State ZipCode, Country}
\affiliation{The University of Manchester}
%% for corresponding author
\email{stefano.coretta@manchester.ac.uk}
%% for additional information
\thanks{other info}
\author{Steven M. Lulich}
% ie
%\affiliation{Department1,  University1, City, State ZipCode, Country}
\affiliation{Indiana University}
%% for corresponding author

%% for additional information


% ie
% \author{Author Four}
% \email{author.four@university.edu}
% \thanks{Also at Another University, City, State ZipCode, Country.}

%% For preprint only,
%  optional, if you want want this message to appear in upper left corner of title page
% \preprint{}

%ie
%\preprint{Author, JASA}

% optional, if desired:
%\date{\today}

\begin{abstract}
% Put your abstract here. Abstracts are limited to 200 words for
% regular articles and 100 words for Letters to the Editor. Please no
% personal pronouns, also please do not use the words ``new'' and/or
% ``novel'' in the abstract. An article usually includes an abstract, a
% concise summary of the work covered at length in the main body of the
% article.
The abstract.
\end{abstract}

%% pacs numbers not used

\maketitle

%  End of title page for Preprint option --------------------------------- %

%% See preprint.tex/.pdf or reprint.tex/.pdf for many examples


%  Body of the article
\section{Introduction}\label{introduction}

The position of the tongue root during the production of voiced stops
plays a fundamental role in ensuring that voicing can be sustained. The
realisation of vocal fold vibration (i.e.~voicing) requires a difference
in air pressure between the cavities below and above the glottis.
Specifically, the sub-glottal pressure needs to be higher than the
supra-glottal pressure for voicing to be maintained. This property of
voicing is formally known as the Aerodynamic Voicing Constraint
\citep{ohala2011}. When the oral tract is completely occluded during the
production of a stop closure, the supra-glottal pressure quickly
increases, due to the incoming airstream from the lungs. Such pressure
increase can hinder the ability to sustain vocal fold vibration during
closure, to the point in which voicing ceases.

An articulatory solution to counterbalance the increased pressure is to
enlarge the supra-glottal cavity by advancing the root of the tongue. It
has been repeatedly observed that the tongue root is in a more front
position in voiced stops compared to voiceless stops
\citep{kent1969, perkell1969, westbury1983}. \citet{rothenberg1967}
calculates that the walls of the supraglottal cavity can absorb the
incoming airflow for 20 to 30 ms by passive expansion, after which the
sub- and supraglottal pressures would equalise and voicing cease.
\citet{rothenberg1967} thus argues that a passive expansion of the
pharyngeal walls is not sufficient.

According to \citet{rothenberg1967}, the active forward gesture of the
tongue root would have a time constant of 70 to 90 ms. Given that stop
closures are generally much shorter than that, it is natural that
advancement is initiated during the vowel, so that an appreciable amount
of advancement is obtained when closure is achieved. Furthermore,
\citet{westbury1983} finds that tongue root advancement is initiated
before full closure is achieved and that there is a forward movement
even in the context of voiceless stops, which is counterintuitive given
that tongue root advancement is generally considered to be a feature of
voiced stops.

However, the relationship between tongue root advancement and voicing is
a complex one. First, tongue root advancement is not the only mechanism
for sustaining voicing during a stop
\citep{rothenberg1967, westbury1983, ohala2011} and it has a certain
level of idiosyncrasy \citep{ahn2016}. Other solutions include expansion
of the lateral walls of the pharynx {[}{]}, larynx lowering
\citep{riordan1980}, opening of the velopharyngeal port
\citep{yanagihara1966}, producing a retroflex occlusion
\citep{sprouse2008}. Second, implementation of tongue root advancement
can be decoupled from the presence of actual vocal fold vibration.
\citet{ahn2015}; \citet{ahn2016} look at word-initial stops in American
English. The stops were phonologically voiceless or voiced. They find
that the tongue root is more advanced in the phonologically voiced stops
independent of whether they actually show vocal fold vibration or not.

In an exploratory study of the link between voicing and vowel duration,
\citet{coretta2018f}; \citet{coretta2018d} looks at the dynamics of
tongue root position during the production of vowels before voiceless
and voiced stops in Italian and Polish. Coretta finds that the advancing
gesture of the tongue root is initiated at around 50\% into the duration
of vowel and that the advancing gesture is present in vowels before both
voiced and voiceless stops in both languages. These findings are in
agreement with \citet{rothenberg1967} and \citet{westbury1983}. The
presence of an advancing gesture---relative to the position of the root
at the onset of the vowel---in voiceless stops could be a mechanical
consequence of tongue body raising.

The place of articulation of the consonant and the vowel type also have
an effect on tongue root advancement. Voiced labial stops do not
generally show tongue root advancement but rather tongue body lowering
\citep{svirsky1997, vazquez-alvarez2007}. Tongue body lowering, however,
is also a general property of labial stops (whether voiced or not), such
that during the production of labial stops, the tongue body lowers
relative to the preceding and following vocalic segment, phenomenon
known as the trough effect.

\section{Methods}\label{methods}

\subsection{Participants}\label{participants}

20 native speakers of American English participated in the experiment.
The participants received a monetary compensation of \$10.

\subsection{Equipment set-up}\label{equipment-set-up}

The system set-up of the Speech Production Laboratory of the Department
of Speech and Hearing Sciences at Indiana University, USA
(\citet{lulich2017}; \citet{charles2018}). The ultrasonic data was
acquired with a Philips EPIQ 7G system using an xMatrix 117 x6-1 digital
3D/4D transducer (). Stabilisation of the ultrasonic transducer was
ensured with the Articulate Instruments Ltd™ headset
(\citeyear{articulate2008}). Synchronised audio was recorded with a
SHURE KSM32 microphone, sampled at 48 kHz (16-bit).

\subsection{Materials}\label{materials}

For this study we have chosen mono- and disyllabic nonce words as target
words. The monosyllabic words are C\textsubscript{1}VC\textsubscript{2}
words (\emph{pop}, \emph{pob}, \emph{caulk} {[}kʰɒkʰ{]}, \emph{cog}).
The disyllabic words have a
C\textsubscript{1}VC\textsubscript{2}-\emph{er} structure
(\emph{popper}, \emph{pobber}, \emph{cocker}, \emph{cogger}). The place
of articulation of C\textsubscript{1} and C\textsubscript{2} was kept
constant within each word to facilitate measuring tongue displacement
and locating gestural landmarks. Only one vowel (/ɒ/) was included in
the study to keep the number of stimuli low, and hence the duration of
the task short. Moreover, back low vowels like /ɒ/ are easier to image
with ultrasound given the proximity of tongue to the transducer.

\subsection{Procedure}\label{procedure}

The data was collected in a sound-attenuated booth in the Speech
Production Laboratory at Indiana University. The stabilisation headset
was fitted on the participant head before recording started. The hard
palate was imaged by asking the participant to swallow water
\citep{epstein2005}. The participants then read the sentence stimuli
which were displayed on a screen via the WASL software. WASL was
developed by Steven M. Lulich and the Indiana University Speech
Production Laboratory, \url{http://www.indiana.edu/~spliu/WASL.htm}.
Each participant read the list of 8 stimuli 10 times. The order of the
stimuli was randomised both across repetitions and across speakers. A
total of 1600 tokens were recorded (8 stimuli per 10 repetitions per 20
speakers).

\subsection{Data processing and
analysis}\label{data-processing-and-analysis}

\begin{acknowledgments}
Thanks to...
\end{acknowledgments}

\appendix
\section{Optional appendix}

%% before appendix (optional) and bibliography:
% \begin{acknowledgments}
%This research was supported by  ...
% \end{acknowledgments}

% -------------------------------------------------------------------------------------------------------------------
%   Appendix  (optional)

%\appendix
%\section{Appendix title}

%If only one appendix, please use
%\appendix*
%\section{Appendix title}


%=======================================================
%IMPORTANT

%Use \bibliography{<name of your .bib file>}+
%to make your bibliography with BibTeX.

%Once you have used BibTeX you
%should open the resulting .bbl file and cut and paste the entire contents
%into the end of your article.
%=======================================================

\bibliography{linguistics.bib}


\end{document}
