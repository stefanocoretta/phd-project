\documentclass[]{elsarticle}
\usepackage{lmodern}
\usepackage{amssymb,amsmath}
\usepackage{ifxetex,ifluatex}
\usepackage{fixltx2e} % provides \textsubscript
\ifnum 0\ifxetex 1\fi\ifluatex 1\fi=0 % if pdftex
  \usepackage[T1]{fontenc}
  \usepackage[utf8]{inputenc}
\else % if luatex or xelatex
  \ifxetex
    \usepackage{mathspec}
  \else
    \usepackage{fontspec}
  \fi
  \defaultfontfeatures{Ligatures=TeX,Scale=MatchLowercase}
\fi
% use upquote if available, for straight quotes in verbatim environments
\IfFileExists{upquote.sty}{\usepackage{upquote}}{}
% use microtype if available
\IfFileExists{microtype.sty}{%
\usepackage{microtype}
\UseMicrotypeSet[protrusion]{basicmath} % disable protrusion for tt fonts
}{}
\usepackage[unicode=true]{hyperref}
\hypersetup{
            pdftitle={The link between tongue root advancement and the voicing effect: an ultrasound study of Italian and Polish},
            pdfauthor={Stefano Coretta},
            pdfborder={0 0 0},
            breaklinks=true}
\urlstyle{same}  % don't use monospace font for urls
\usepackage{natbib}
\bibliographystyle{plainnat}
\IfFileExists{parskip.sty}{%
\usepackage{parskip}
}{% else
\setlength{\parindent}{0pt}
\setlength{\parskip}{6pt plus 2pt minus 1pt}
}
\setlength{\emergencystretch}{3em}  % prevent overfull lines
\providecommand{\tightlist}{%
  \setlength{\itemsep}{0pt}\setlength{\parskip}{0pt}}
\setcounter{secnumdepth}{5}
% Redefines (sub)paragraphs to behave more like sections
\ifx\paragraph\undefined\else
\let\oldparagraph\paragraph
\renewcommand{\paragraph}[1]{\oldparagraph{#1}\mbox{}}
\fi
\ifx\subparagraph\undefined\else
\let\oldsubparagraph\subparagraph
\renewcommand{\subparagraph}[1]{\oldsubparagraph{#1}\mbox{}}
\fi

% set default figure placement to htbp
%\makeatletter
%\def\fps@figure{htbp}
%\makeatother


\title{The link between tongue root advancement and the voicing effect: an
ultrasound study of Italian and Polish}
\author{Stefano Coretta}
\date{08/09/2017}

\begin{document}
\maketitle

\section{Introduction}\label{introduction}

It is known that tongue root plays a role in maintaining voicing in
voiced stops in English. Recent ultrasound tongue imaging work has
confirmed that the tongue root is advanced in voiced consonants. Tongue
root advancement has been shown to be present also when vocal fold
vibration is not present \citep{ahn2016}. An interesting question
arising from this is weather tongue root advancement might be correlated
to other factors like, the focus of this paper, vowel duration. Several
studies showed that vowels followed by voiced stops are longer than
vowels followed by voiceless stops. Different languages show different
magnitudes of such durational differential and some languages do not
show it at all. Given the connection between voicing and tongue root
advancement, it is natural to ask whether tongue root advancement is
also linked to vowel duration. If this is indeed the case, then one
would expect tongue root advancement to play a role in language that
have the voicing effect, but not in languages that do not show it.

To test this hypothesis, I conducted an acoustic and articulatory study
that looked at vowel duration and tongue contours to assess the possible
link between consonantal and vocalic tongue gestures.

Although several attempts have been put forward to explain the effect of
voicing on vowel durations, to date no consensus has been reached. A
recurrent theme focusses on the difference in gestural implementation
that characterises voiced stops in comparison with voiceless
stops.\footnote{However, see ... for a perceptually inclined account.}
One of the first accounts that attributed the voicing effect to a
difference in production is that of \citet{halle1967}, subsequently
reiterated by \citet{chomsky1968}. According to this account, voicing in
vowels is produced with a state of the glottis that diverges from the
configuration necessary to produce voiced consonants, due to the
aerodynamics of the vocal tract. On the contrary, it is claimed that
voiceless stops do not require any specific glottal configuration and
thus the voicing perpetuated during the vowel naturally ceases at
closure. The authors thus hypothesise that, to allow the glottal state
to change in voiced stops from sonorant voicing to obstruent voicing,
the vowel is lengthen so that enough time is available for the change to
happen without compromising the quality of voicing during the vowel.
Although such account seemed promising at that time, later studies could
not demonstrate that obstruent voicing is any different from sonorant
voicing.

Tongue root advancement differences seems like a promising area of
enquiry since its link to voicing has been already confirmed. On the
same line of the laryngeal hypothesis, I put forward a similar account
in which it is tongue root advancement rather than fold configuration
that requires extra time during the vowel to be implemented. If tongue
root advancement plays a role in determining the duration of preceding
vowels through the extension mechanism described before, than one
expects languages with the voicing effect to show a systematic
advancement of the tongue root in voiced stops. On the contrary, tongue
root advancement in languages without the voicing effect should be
absent or less systematic.

\section{Methodology}\label{methodology}

\subsection{Participants}\label{participants}

Native speakers of Italian and Polish have been recruited in Manchester
and in Italy. Four speakers per language participated in the experiment.
The participants received a compensation of £10 (or equivalent in Euro).

\subsection{Equipment set-up}\label{equipment-set-up}

An Articulate Instruments set-up was used for this study. This is
constituted by a TELEMED C3.5/20/128Z-3 ultrasonic transducer plugged
into a TELEMED Echo Blaster 128 unit. A synchronisation unit (P-Stretch)
was plugged into the Echo Blaster unit and used for automatic
audio/ultrasound synchronisation. A Movo LV4-O2 Lavalier microphone
plugged into a \ldots{} was used for audio recording. Articulate
Assistant Advanced v2.17.1 running on a Windows laptop was used for the
acquisition of audio and ultrasonic signal. A stabilisation headset
produced by Articulate Instruments was used for probe stabilisation
{[}{]}.

\subsection{Materials}\label{materials}

Disyllabic words of the form
C\textsubscript{1}V\textsubscript{1}C\textsubscript{2}V\textsubscript{2}
were used as targets, where C\textsubscript{1} = /p/, V\textsubscript{1}
= /a, o, u/, and C\textsubscript{2} = /t, d, k, g/ (e.g. /pata/, /pada/,
/poto/, etc.). A bilabial stop /p/ was chosen as the first consonant to
reduce influence on the following vowel (although cf.
\citet{vazquez-alvarez2007}). Only coronal and velar stops were used as
target consonants since labial consonants cannot be imaged with
ultrasonography. All possible combinations were employed, yielding to a
total of 12 target words. The words were embedded in medial position
within a frame sentence. Prosodically similar sentences were used to
ensure comparability between languages. The frame sentence was
\emph{Dico X lentamente} `I say X slowly' for Italian, and \emph{Mówię X
teraz} `I say X now' for Polish.

\subsection{Procedure}\label{procedure}

The stimuli were randomised for each participant and repeated six times
(the order was kept the same in each of the six repetitions due to
software constraints). Before recording the stimuli, the occlusal plane
of the participant was imaged using a bite plate (), and the palate
asking the participant to swallow water.

\subsection{Data processing}\label{data-processing}

Durational measurements were obtained from acoustics. A first rough
annotation was achieved through force alignment using the SPASS force
aligner. The criteria for the annotation of the acoustic data were the
following:

\begin{itemize}
\tightlist
\item
  the target vowel onset was marked by visual inspection at the time of
  appearance of higher formants in the spectrogram following the burst
  of /p/
\item
  the target vowel offset was marked by visual inspection at the time of
  disappearance of the higher formants in the spectrogram; this point
  also corresponded with the consonantal onset
\item
  closure offset was marked by visual inspection at the time of formants
  onsets of the following vowel
\item
  burst onset was determined automatically by a Praat implementation of
  the algorithm described in \citet{ananthapadmanabha2014}
\end{itemize}

Finally, vowel duration, consonant duration (including burst and VOT),
and closure duration (from vowel offset to burst onset) were also
automatically extracted with a Praat script.

\subsection{Analysis}\label{analysis}

The spline data was normalised within speaker by coordinate offsetting
and rotation based on the occlusal plane. Generalised additive mixed
effects models (GAMM) were used for statistical analysis in R
\citep{r-core-team2017}. Duration measurements were subject to linear
mixed effects model using \texttt{lme4} in R.

\section{Results}\label{results}

\renewcommand\refname{Discussion}
\bibliography{linguistics.bib}

\end{document}
