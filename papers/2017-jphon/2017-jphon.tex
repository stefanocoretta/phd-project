\documentclass[]{elsarticle}
\usepackage{lmodern}
\usepackage{amssymb,amsmath}
\usepackage{ifxetex,ifluatex}
\usepackage{fixltx2e} % provides \textsubscript
\ifnum 0\ifxetex 1\fi\ifluatex 1\fi=0 % if pdftex
  \usepackage[T1]{fontenc}
  \usepackage[utf8]{inputenc}
\else % if luatex or xelatex
  \ifxetex
    \usepackage{mathspec}
  \else
    \usepackage{fontspec}
  \fi
  \defaultfontfeatures{Ligatures=TeX,Scale=MatchLowercase}
\fi
% use upquote if available, for straight quotes in verbatim environments
\IfFileExists{upquote.sty}{\usepackage{upquote}}{}
% use microtype if available
\IfFileExists{microtype.sty}{%
\usepackage{microtype}
\UseMicrotypeSet[protrusion]{basicmath} % disable protrusion for tt fonts
}{}
\usepackage[unicode=true]{hyperref}
\hypersetup{
            pdftitle={The link between tongue root advancement and the voicing effect: an ultrasound study of Italian and Polish},
            pdfauthor={Stefano Coretta},
            pdfborder={0 0 0},
            breaklinks=true}
\urlstyle{same}  % don't use monospace font for urls
\usepackage{natbib}
\bibliographystyle{plainnat}
\IfFileExists{parskip.sty}{%
\usepackage{parskip}
}{% else
\setlength{\parindent}{0pt}
\setlength{\parskip}{6pt plus 2pt minus 1pt}
}
\setlength{\emergencystretch}{3em}  % prevent overfull lines
\providecommand{\tightlist}{%
  \setlength{\itemsep}{0pt}\setlength{\parskip}{0pt}}
\setcounter{secnumdepth}{5}
% Redefines (sub)paragraphs to behave more like sections
\ifx\paragraph\undefined\else
\let\oldparagraph\paragraph
\renewcommand{\paragraph}[1]{\oldparagraph{#1}\mbox{}}
\fi
\ifx\subparagraph\undefined\else
\let\oldsubparagraph\subparagraph
\renewcommand{\subparagraph}[1]{\oldsubparagraph{#1}\mbox{}}
\fi

% set default figure placement to htbp
%\makeatletter
%\def\fps@figure{htbp}
%\makeatother


\title{The link between tongue root advancement and the voicing effect: an
ultrasound study of Italian and Polish}
\author{Stefano Coretta}
\date{08/09/2017}

\begin{document}
\maketitle

\section{Introduction}\label{introduction}

It is known that tongue root plays a role in maintaining voicing in
voiced stops in English. Recent ultrasound tongue imaging work has
confirmed that the tongue root is advanced in voiced consonants. Tongue
root advancement has been shown to be present also when vocal fold
vibration is not present \citep{ahn2016}. An interesting question
arising from this is weather tongue root advancement might be correlated
to other factors like, the focus of this paper, vowel duration. Several
studies showed that vowels followed by voiced stops are longer than
vowels followed by voiceless stops. Different languages show different
magnitudes of such durational differential and some languages do not
show it at all. Given the connection between voicing and tongue root
advancement, it is natural to ask whether tongue root advancement is
also linked to vowel duration. If this is indeed the case, then one
would expect tongue root advancement to play a role in language that
have the voicing effect, but not in languages that do not show it.

To test this hypothesis, I conducted an acoustic and articulatory study
that looked at vowel duration and tongue contours to assess the possible
link between consonantal and vocalic tongue gestures.

Although several attempts have been put forward to explain the effect of
voicing on vowel durations, to date no consensus has been reached. A
recurrent theme focusses on the difference in gestural implementation
that characterises voiced stops in comparison with voiceless
stops.\footnote{However, see ... for a perceptually inclined account.}
One of the first accounts that attributed the voicing effect to a
difference in production is that of \citet{halle1967}, subsequently
reiterated by \citet{chomsky1968}. According to this account, voicing in
vowels is produced with a state of the glottis that diverges from the
configuration necessary to produce voiced consonants, due to the
aerodynamics of the vocal tract. On the contrary, it is claimed that
voiceless stops do not require any specific glottal configuration and
thus the voicing perpetuated during the vowel naturally ceases at
closure. The authors thus hypothesise that, to allow the glottal state
to change in voiced stops from sonorant voicing to obstruent voicing,
the vowel is lengthen so that enough time is available for the change to
happen without compromising the quality of voicing during the vowel.
Although such account seemed promising at that time, later studies could
not demonstrate that obstruent voicing is any different from sonorant
voicing.

Tongue root advancement differences seems like a promising area of
enquiry since its link to voicing has been already confirmed. On the
same line of the laryngeal hypothesis, I put forward a similar account
in which it is tongue root advancement rather than fold configuration
that requires extra time during the vowel to be implemented. If tongue
root advancement plays a role in determining the duration of preceding
vowels through the extension mechanism described before, than one
expects languages with the voicing effect to show a systematic
advancement of the tongue root in voiced stops. On the contrary, tongue
root advancement in languages without the voicing effect should be
absent or less systematic.

Italian has been reported to have the voicing effect.
\citet{farnetani1986}, in a study assessing general properties on
segmental durations of spoken Italian, found for the pair of nonce words
/lata/ \textasciitilde{} /lada/ that the vowel /a/ was on average 35
msec longer when followed by the voiced stop /d/ (/lata/ 223 msec, sd =
18; /lada/ 258 msec, sd = 13, p.~26). \citet{esposito2002} extended the
research to all vowels and stops and found that vowels were longer when
followed by a voiced stop. Although the estimates from the statistical
models are not reported in the paper, judging from the figures in the
graphs, an average of 30-35 msec difference was found, which is similar
to what has been reported by \citet{farnetani1986}.

On the other hand, \citet{keating1984} reported that vowels in Polish
are not affected by the voicing of the following consonant. She tested
vowel duration in the word pair /rata/ \textasciitilde{} /rada/ and
found that there was a durational difference of 2 milliseconds. As for
the some of the Italian studies, statistical analysis was not performed,
although judging by the size of the difference, it can be assumed that
it is not significant.

\section{Methodology}\label{methodology}

\subsection{Participants}\label{participants}

Native speakers of Italian and Polish have been recruited in Manchester
and in Italy. Four speakers per language participated in the experiment.
The participants received a compensation of £10 (or equivalent in Euro).

\subsection{Equipment set-up}\label{equipment-set-up}

An Articulate Instruments set-up was used for this study. This is
constituted by a TELEMED C3.5/20/128Z-3 ultrasonic transducer plugged
into a TELEMED Echo Blaster 128 unit. A synchronisation unit (P-Stretch)
was plugged into the Echo Blaster unit and used for automatic
audio/ultrasound synchronisation. A Movo LV4-O2 Lavalier microphone
plugged into a \ldots{} was used for audio recording. Articulate
Assistant Advanced v2.17.1 running on a Windows laptop was used for the
acquisition of audio and ultrasonic signal. A stabilisation headset
produced by Articulate Instruments was used for probe stabilisation
{[}{]}.

\subsection{Materials}\label{materials}

Disyllabic words of the form
C\textsubscript{1}V\textsubscript{1}C\textsubscript{2}V\textsubscript{2}
were used as targets, where C\textsubscript{1} = /p/, V\textsubscript{1}
= /a, o, u/, and C\textsubscript{2} = /t, d, k, g/ (e.g. /pata/, /pada/,
/poto/, etc.). A bilabial stop /p/ was chosen as the first consonant to
reduce influence on the following vowel (although cf.
\citet{vazquez-alvarez2007}). Only coronal and velar stops were used as
target consonants since labial consonants cannot be imaged with
ultrasonography. All possible combinations were employed, yielding to a
total of 12 target words. The words were embedded in medial position
within a frame sentence. Prosodically similar sentences were used to
ensure comparability between languages. The frame sentence was
\emph{Dico X lentamente} `I say X slowly' for Italian, and \emph{Mówię X
teraz} `I say X now' for Polish.

\subsection{Procedure}\label{procedure}

The stimuli were randomised for each participant and repeated six times
(the order was kept the same in each of the six repetitions due to
software constraints). Before recording the stimuli, the occlusal plane
of the participant was imaged using a bite plate \citep{scobbie2011},
and the palate asking the participant to swallow water.

\subsection{Data processing}\label{data-processing}

Durational measurements were obtained from acoustics. A first rough
annotation was achieved through force alignment using the SPASS force
aligner \citep{bigi2015}. The criteria for the annotation of the
acoustic data were the following:

\begin{itemize}
\tightlist
\item
  the target vowel onset was marked by visual inspection at the time of
  appearance of higher formants in the spectrogram following the burst
  of /p/
\item
  the target vowel offset was marked by visual inspection at the time of
  disappearance of the higher formants in the spectrogram; this point
  also corresponded with the consonantal onset
\item
  closure offset was marked by visual inspection at the time of formants
  onsets of the following vowel
\item
  burst onset was determined automatically by a Praat implementation of
  the algorithm described in \citet{ananthapadmanabha2014}
\end{itemize}

Finally, vowel duration, consonant duration (including burst and VOT),
and closure duration (from vowel offset to burst onset) were also
automatically extracted with a Praat script.

Tongue contours were extracted from the ultrasonic data using Articulate
Assistant Advanced \citep[AAA,][]{articulate2011}. Spline lines were
first fitted to the visible contours using the batch tracking function
natively included in AAA. Manual correction was applied in the cases
where the automatic tracking showed clear errors. Tongue displacement
along time was thus calculated in AAA using a custom module provided by
Dr.~Patrycja Strycharczuk \citep{Strycharczuk2015}. The displacement was
calculated based on the movement of the splines along a given fan line.
For each speaker, two relevant fan lines were chosen: one for the tongue
tip and one for the tongue dorsum. The selection of the fan lines was
based on the inspection of the standard deviation of the splines in the
relevant area of the tongue. The fan line that showed the highest
standard deviation was then chosen as the relevant fan line for the
calculation of displacement. Velocity and tangential velocity were also
calculated. The time of maximum displacement based on tangential
velocity was finally obtained using a modified version of a search
set-up provided by Dr.~Patrycja Strycharczuk.

\subsection{Analysis}\label{analysis}

The coordinates of the splines were exported at acoustic closure and at
maximum displacement. The contours were normalised by applying
offsetting and rotation relative to the participant's occlusal plane
\citep{scobbie2011}. Generalised additive mixed effects models
\citep{wood2006} were used for the statistical analysis of tongue
contour data in R \citep{r-core-team2017}. Duration measurements were
subject to linear mixed effects models using \texttt{lme4} in R
\citep{bates2015}.

\section{Results}\label{results}

\subsection{Vowel duration and
voicing}\label{vowel-duration-and-voicing}

A linear mixed effect regression model was fit on the Italian vowel
durations with duration as the outcome variable, vowel quality (/a, o,
u/), voicing and place of articulation of the following consonant,
sentence duration as fixed effects, random intercepts by speaker and
word, and random slopes for voicing by speaker. An interaction between
voicing and vowel quality was also included, which turned out to be
significant. P-values were obtained through likelihood ration tests
comparing a model including voicing with a null model without voicing.
According to the model, Italian vowels are 19.5 msec (±5.5) longer if
followed by a voiced stop (\(\chi^2\)(3) = 18.5, p = 0.000337).

For Polish, the same model structure was used, excluding the
voicing-vowel interaction (which was not significant). Surprisingly, the
model reported a partially significant 8 msec (±3) effect of consonantal
voicing on the preceding vowel (\(\chi^2\)(1) = 5.4, p = 0.02). The
exploration of the random slopes by speaker showed that one speaker
showed a particularly higher slope for voicing, meaning that the effect
of voicing was stronger in his data. This observation will come handy
while discussing about the results of the tongue contour data.

\subsection{Tongue contours}\label{tongue-contours}

The Italian tongue contour analysis showed that voiced stops are
produced with advancement of the root of the tongue. Generalised
additive mixed effect models were fitted for each speaker: the
y-coordinates of the contours were the outcome variable; the
x-coordinates the parametric term; a reference smooth term for X, with
difference smooths for X by voicing, vowel quality, and place of
articulation of following consonant; random smooths by word. A
first-order autoregressive model was included, given the high
auto-correlation residuals as obtained by visual inspection of the
residuals. For two participants out of four, the root was significantly
more front in voiced stops in both vocalic contexts (/a, o/). On the
other hand, one participant had significant tongue root advancement only
following /a/, while the fourth participant didn't show advancement at
all. For Polish, three out of four speakers did not have tongue root
advancement, while the fourth speaker showed significant advancement in
both vocalic contexts.

Further contour analysis was carried out at the point of acoustic
closure for Italian and for the Polish speaker showing advancement.
Surprisingly, the tongue root in the voiced condition was found to be
already in advanced position at closure. Both the Italian (4 speakers)
and the Polish data (1 speaker) show that the root of the tongue at the
time of consonantal closure is advanced in both languages. Models fitted
on the individual speakers comparing tongue contours at closure and at
maximum displacement further confirmed that root advancement was bigger
at maximum displacement for Italian, but not for Polish.

\section{Discussion}\label{discussion}

The presence of tongue root advancement in Italian but not in Polish
initially supports the idea that vowels are longer if followed by voiced
stops to allow time for the root to reach a position suitable for
consonantal voicing. The reported absence of the voicing effect in
Polish could then be ascribed to the absence of tongue root advancement
in the voiced consonants of Polish. However, such statement is
complicated by two facts that emerged from the data analysed in this
study. First, tongue root advancement was found in one of the Polish
speakers and it was absent from one of the Italian speakers. Second,
even though the effect was not as big as for Italian, the Polish
duration data nonetheless showed a difference of 8 msec. However, as
mentioned above, one Polish speaker had a particularly higher slope for
voicing, and incidentally this is also the same and only Polish speaker
who showed root advancement. Moreover, a difference of 8 milliseconds
could be considered to be quite small in any case, although
statistically
detectable.\footnote{A model fitted on the data excluding the outlier speaker leads in fact to a smaller estimate of about 6 msec.}

The data also showed raising of the tongue dorsum concomitant to the
advancement of the root. Although such gesture was not prospected on the
basis of the tested hypothesis, it makes sense from an anatomical stand
point. Given the anteriorisation of the tongue root, other things being
equal, it derives that the tongue mass gets compressed. Such compression
can be counterbalanced (entirely or partially) by allowing the dorsum of
the tongue to raise. It is not thus surprising to observe a raised
dorsum in voiced stops accompanying root advancement.

\bibliography{linguistics.bib}

\end{document}
