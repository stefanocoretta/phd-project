%%%%%%%%%%%%%%%%%%%%%%%%%%%%%%%%%%%%%%%%%%%%%%%%%%
%	JASA LaTeX Template File
%  For use in making articles using JASAnew.cls
% July 26, 2017
%%%%%%%%%%%%%%%%%%%%%%%%%%%%%%%%%%%%%%%%%%%%%%%%%%

%% Step 1:
%% Uncomment the style that you want to use:

%%%%%%% For Preprint
%% For manuscript, 12pt, one column style

%% Comment this out if you'd rather use another style:
\documentclass[]{JASAnew}

%%%%% Preprint Options %%%%%
%% The track changes option allows you to mark changes
%% and will produce a list of changes, their line number
%% and page number at the end of the article.
%\documentclass[preprint,trackchanges]{JASAnew}

%% authaffil option will make affil immediately
% follow author, otherwise authors are grouped, and affiliations
% are stacked underneath all the authors.
%\documentclass[preprint,authaffil]{JASAnew}

%% NumberedRefs is used for numbered bibliography and citations.
%% Default is Author-Year style.
%% \documentclass[preprint,NumberedRefs]{JASAnew}

%%%%%%% For Reprint
%% For appearance of finished article; 2 columns, 10 pt fonts

% \documentclass[reprint]{JASAnew}

%%%%% Reprint Options %%%%%

%% For testing to see if author has exceeded page length request, use 12pt option
%\documentclass[reprint,12pt]{JASAnew}

% authaffil option will make affil immediately
% follow author, otherwise authors are grouped, and affiliations
% are stacked underneath all the authors.
%\documentclass[reprint,authaffil]{JASAnew}

%% NumberedRefs is used for numbered bibliography and citations.
%% Default is Author-Year style.
% \documentclass[reprint,NumberedRefs]{JASAnew}

%% TurnOnLineNumbers
%% Make lines be numbered in reprint style:
% \documentclass[reprint,TurnOnLineNumbers]{JASAnew}

\usepackage{natbib}


\usepackage{cleveref}
\usepackage{ctable}

\begin{document}
%% the square bracket argument will send term to running head in
%% preprint, or running foot in reprint style.

\title[]{Vowel duration and tongue root advancement: Results from an exploratory
study of the relation between voicing and vowel duration}

% ie
%\title[JASA/Sample JASA Article]{Sample JASA Article}

%% repeat as needed

\author{Stefano Coretta}
% ie
%\affiliation{Department1,  University1, City, State ZipCode, Country}
\affiliation{The University of Manchester}
%% for corresponding author
\email{stefano.coretta@manchester.ac.uk}
%% for additional information


% ie
% \author{Author Four}
% \email{author.four@university.edu}
% \thanks{Also at Another University, City, State ZipCode, Country.}

%% For preprint only,
%  optional, if you want want this message to appear in upper left corner of title page
% \preprint{}

%ie
%\preprint{Author, JASA}

% optional, if desired:
%\date{\today}

\begin{abstract}
% Put your abstract here. Abstracts are limited to 200 words for
% regular articles and 100 words for Letters to the Editor. Please no
% personal pronouns, also please do not use the words ``new'' and/or
% ``novel'' in the abstract. An article usually includes an abstract, a
% concise summary of the work covered at length in the main body of the
% article.
Put your abstract here. Abstracts are limited to 200 words for regular
articles and 100 words for Letters to the Editor. Please no personal
pronouns, also please do not use the words
\texttt{new\textquotesingle{}\textquotesingle{}\ and/or}novel'' in the
abstract. An article usually includes an abstract, a concise summary of
the work covered at length in the main body of the article.
\end{abstract}

%% pacs numbers not used

\maketitle

%  End of title page for Preprint option --------------------------------- %

%% See preprint.tex/.pdf or reprint.tex/.pdf for many examples


%  Body of the article
\hypertarget{introduction}{%
\section{Introduction}\label{introduction}}

This paper reports a previously undocumented correlation between vowel
duration and degree of tongue root advancement. In an exploratory study
of the articulatory correlates of stop voicing, it has been found that
tongue root advancement---a known mechanism that facilitates voicing
during stop closure---can be implemented not only during the closure of
a stop, but even during the production of the vowel preceding a stop.
Moreover, vowel acoustic duration turned out to be linearly correlated
with degree of tongue root advancement, such that longer vowels show
greater tongue root advancement.

One of the known differences in supra-glottal articulation between
voiced and voiceless stops concerns the position of the tongue root
relative to the front-back axis of the oral tract. It has been observed
that the tongue root is more advanced in voiced than in voiceless stops
\citep{kent1969,perkell1969,westbury1983}. This gesture has been
interpreted as a mechanism to ensure voicing can be maintained during
the closure of the stop. The realisation of vocal fold vibration
(i.e.~voicing) requires the air pressure in the supra-glottal cavity to
be lower than the air pressure below the glottis. During the production
of voiced obstruents, the supra-glottal pressure increases due to the
immittance of air in the supra-glottal cavity. Such pressure increase
can hinder the ability to sustain voicing during closure, to the point
that voicing ceases if the supra-glottal pressure is higher than the
sub-glottal pressure \citep{ohala2011}. One of the possible articulatory
solution to counterbalance the increase in pressure during the closure
of a voiced stop is to expand the supra-glottal cavity by advancing the
root of the tongue.

Tongue root advancement has also been reported as a mechanism for
ensuring a short voice onset time \citep{ahn2016}.

An extensive number of studies show that, cross-linguistically, vowels
tend to be longer when followed by voiced obstruents than when they are
followed by voiceless obstruents
\citep{house1953, peterson1960, chen1970, klatt1973, lisker1974, farnetani1986, fowler1992, hussein1994, esposito2002, lampp2004, durvasula2012}.
This phenomenon, know as the voicing effect, has been reported in a
variety of languages, including (but not limited to) English, German,
Hindi, Russian, Italian, Arabic, and Korean (see \citealt{maddieson1976}
for a more comprehensive list). A common stance in the literature is
that the magnitude of the voicing effect differs depending on the
language (although see \citealt{laeufer1992} ), and that this phenomenon
is not a universal tendency, since the duration of vowels is not
affected by the voicing of the following obstruents in some languages,
like Polish and Czech \citep{keating1984}. Although several attempts
have been made to explain the voicing effect, an account that survives
all empirical data is still lacking \citep{durvasula2012,soskuthy2013}.

To summarise, tongue root advancement, shorter VOT duration, and longer
vowel durations are all correlates of voicing. Moreover, tongue root
advancement and shorter VOT show a link. In this paper, I will report
the results from an exploratory study which fill the gap in this picture
of correlation, by showing that tongue root advancement is also linked
to longer vowel durations.

\hypertarget{methodology}{%
\section{Methodology}\label{methodology}}

\hypertarget{participants}{%
\subsection{Participants}\label{participants}}

Eleven native speakers of Italian (5 females, 6 males) and 6 native
speakers of Polish (3 females, 3 males) were recorded in the Phonetics
Laboratory at the University of Manchester and in a private location in
Italy (see \Cref{t:participants}). The Italian speakers of this study
are from Northern Italy (three from the north-west and one from
north-east). The Polish group was more heterogeneous, with two speakers
from the West (Poznań), and two from the East (Warsaw and Przasnysz).
Ethical clearance was obtained for this study from the University of
Manchester (REF 2016-0099-76). The participants received a monetary
compensation.

\hypertarget{materials}{%
\subsection{Materials}\label{materials}}

Disyllabic words of the form
C\textsubscript{1}V\textsubscript{1}C\textsubscript{2}V\textsubscript{2}
were used as targets, where C\textsubscript{1} = /p/, V\textsubscript{1}
= /a, o, u/, C\textsubscript{2} = /t, d, k, g/, and V\textsubscript{2} =
V\textsubscript{1} (e.g. \emph{pata}, \emph{pada}, \emph{poto}, etc.),
giving a total of 12 target words, used both for Italian and Polish.
Most of these words were nonce words in both languages, with a few
exceptions (see table). The words were presented using the respective
writing conventions (see table). A labial stop was chosen as the first
consonant to reduce possible coarticulation with the following
vowel.\footnote{Although there is a tendency in the articulatory literature to suggest that labial consonants do not affect lingual articulations, \citealt{vazquez-alvarez2007} report tongue body lowering in the context of labial stops.}
At low power settings and high frame rates, high and mid front vowel
have the double disadvantage of being often not clearly visible in the
ultrasound image (given their greater distance from the probe) and of
producing less displacement of the tongue (essential for the closing
gesture identification, see XXX) in the movement from the vowel itself
to the following consonant. For this reason, only central/back vowels
(low /a/, mid /o/, and high /u/) were included in the target words. The
use of back and central vowels (with the exclusion of mid/high front
vowels) had the advantage of facilitating the identification of the
consonantal gesture of C2. Since the original motive for the exploratory
study was to study possible difference in the articulation of closure in
voiceless vs.~voiced stops, only coronal and velar stops were chosen as
target consonants since, of course, the closure of labial consonants
cannot be imaged with ultrasonography. The target words were embedded in
a frame sentence, \emph{Dico X lentamente} `I say X slowly' for Italian,
and \emph{Mówię X teraz} `I say X now' for Polish. The similarity of
prosodic structure of these sentences ensured better comparability
between the two languages.

\hypertarget{equipment-and-procedure}{%
\subsection{Equipment and procedure}\label{equipment-and-procedure}}

\label{s:equipment}

An Articulate Instruments Ltd™ system was used for this study. The
system is made of a TELEMED Echo Blaster 128 unit, an Articulate
Instruments Ltd™ P-Stretch synchronisation unit, and a FocusRight
Scarlett Solo pre-amplifier (see \Cref{f:uti-setup}). A TELEMED
C3.5/20/128Z-3 ultrasonic transducer (20mm radius, 2-4 MHz) and a Movo
LV4-O2 Lavalier microphone were used respectively the acquisition of
ultrasonic and audio data. Stabilisation of the ultrasonic transducer
was ensured by means of a metallic headset designed by Articulate
Instruments Ltd™ (\citeyear{articulate2008}). The transducer was placed
in contact with the sub-mental triangle, aligned with the mid-sagittal
plane. The headset holds the transducer and it keeps it in a constant
position and inclination relative to the sub-mental triangle, thus
allowing head movements without the need for post-processing correction.
The acquisition of the mid-sagittal ultrasonic and audio signals was
achieved with the software Articulate Assistant Advanced (AAA, v2.17.2)
running on a Hawlett-Packard ProBook 6750b laptop with Microsoft Windows
7. The synchronisation of the ultrasonic and audio signals was performed
by AAA after recording by means of a synchronisation signal produced by
the P-Stretch unit. The ranges of the ultrasonic settings were: frames
per second = 43-68, number of scan lines = 88--114, pixel per scan line
= 980--988, field of view = 71--93°, pixel offset = 109--263, depth (mm)
= 75--180. The audio signal was recorded at 22050 Hz (16-bit).

The head set with the prob was fitted at the beginning of the
experimental session. Then the hard palate was imaged by recording the
participant swallowing water \citep{epstein2005}. A between-speaker
reference coordinate system was derived from imaging the occlusal plane,
which corresponds to the trace of a metallic bite plate inserted in the
mouth of the participant \citep{scobbie2011}. The participant gently
bites on the bite plate while pressing the tongue against it. The
participant then started reading six repetitions of the sentences with
the target words. These were presented on the screen in a randomised
order across participants, although the order was kept the same for each
repetition within participant due to design constraints of the AAA
software.

\hypertarget{data-processing-and-analysis}{%
\subsection{Data processing and
analysis}\label{data-processing-and-analysis}}

\hypertarget{acoustic-data}{%
\subsubsection{Acoustic data}\label{acoustic-data}}

The audio data was subject to force alignment using the SPeech
Phonetisation Alignment and Syllabification software (SPPAS)
\citep{bigi2015}. The outcome of the automatic alignment was then
manually corrected, according to the criteria in \Cref{t:dur-measures},
which were mainly based on properties of the the spectrogram. The
release C2 was detected automatically in Praat \citep{boersma2016} by
means of the burst-detection algorithm described in
\citet{ananthapadmanabha2014}. The durations of the following intervals
were extracted from the annotated acoustic landmarks using a scripted
procedure in Praat: vowel duration (V1 onset to V1 offset), consonant
duration (V1 offset to V2 onset), and closure duration (V1 offset to C2
release).

\ctable[caption = List of measurements extracted from the acoustic data.,
label = t:dur-measures,
width=\textwidth,
star
]{ll>{\raggedright}p{9cm}}{}{
\FL
\textbf{landmarks}               &                  & \textbf{criteria}                                                                                    \ML
vowel onset           & (V1 onset)         & appearance of higher formants in the spectrogram following the release of /p/ (C1)            \NN
vowel offset          & (V1 offset)        & disappearance of the higher formants in the spectrogram preceding the target consonant (C2) \NN
consonant onset       & (C2 onset)         & corresponds to V1 offset                                                                    \NN
closure onset         & (C2 closure onset) & corresponds to V1 offset                                                                    \NN
consonant offset      & (C2 offset)        & appearance of higher formants of the vowel following C2 (V2); corresponds to V2 onset                                \NN
consonant release & (C2 release)         & automatic detection \citep{ananthapadmanabha2014}                                           \LL
}

\hypertarget{ultrasonic-data}{%
\subsubsection{Ultrasonic data}\label{ultrasonic-data}}

Mid-sagittal tongue contours were obtained from the ultrasonic data
according to the following method. Smoothing splines were automatically
fitted to the visible tongue contours in AAA. Manual correction was then
applied in cases of clear tracking errors. The time of maximum tongue
displacement within consonant closure was then calculated in AAA
following the method described in \citet{strycharczuk2015}, which is
based on the velocity of tongue displacement along a vector. To obtain
the time of maximum displacement, a fan-like frame consisting of 42
equidistant radial lines is used as the coordinate system. The origin of
the 42 fan-lines coincides with the centre of the ultrasonic transducer,
such that each fan-line is parallel to the direction of the ultrasonic
scan lines. Tongue displacement was thus calculated as the displacement
of the fitted splines along the fan-line vectors. The time of maximum
tongue displacement was the time of greater displacement along the
vector that showed the greatest standard deviation of displacement. The
fan-lines from which the relevant vector was chosen were restricted to
the fan-lines corresponding to the tongue tip for coronal consonants,
and to the fan-lines corresponding to the tongue dorsum for velar
consonants.

The polar coordinates of the tongue contours were exported from two time
points: the onset of C2 closure, and the time of maximum tongue
displacement (which is always within C2 closure). The contours were
normalised within speaker by applying offsetting and rotation relative
to the participant's occlusal plane \citep{scobbie2011}.

%% before appendix (optional) and bibliography:
% \begin{acknowledgments}
%This research was supported by  ...
% \end{acknowledgments}

% -------------------------------------------------------------------------------------------------------------------
%   Appendix  (optional)

%\appendix
%\section{Appendix title}

%If only one appendix, please use
%\appendix*
%\section{Appendix title}


%=======================================================
%IMPORTANT

%Use \bibliography{<name of your .bib file>}+
%to make your bibliography with BibTeX.

%Once you have used BibTeX you
%should open the resulting .bbl file and cut and paste the entire contents
%into the end of your article.
%=======================================================

\bibliography{linguistics.bib}


\end{document}
