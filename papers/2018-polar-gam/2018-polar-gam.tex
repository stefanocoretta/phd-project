\documentclass[12pt,]{article}
\usepackage{lmodern}
\usepackage{amssymb,amsmath}
\usepackage{ifxetex,ifluatex}
\usepackage{fixltx2e} % provides \textsubscript
\ifnum 0\ifxetex 1\fi\ifluatex 1\fi=0 % if pdftex
  \usepackage[T1]{fontenc}
  \usepackage[utf8]{inputenc}
\else % if luatex or xelatex
  \ifxetex
    \usepackage{mathspec}
  \else
    \usepackage{fontspec}
  \fi
  \defaultfontfeatures{Ligatures=TeX,Scale=MatchLowercase}
\fi
% use upquote if available, for straight quotes in verbatim environments
\IfFileExists{upquote.sty}{\usepackage{upquote}}{}
% use microtype if available
\IfFileExists{microtype.sty}{%
\usepackage{microtype}
\UseMicrotypeSet[protrusion]{basicmath} % disable protrusion for tt fonts
}{}
\usepackage[margin=1in]{geometry}
\usepackage{hyperref}
\hypersetup{unicode=true,
            pdftitle={Assessing midsaggital tongue contours in polar coordinates using generalised additive (mixed) models},
            pdfauthor={Stefano Coretta},
            pdfborder={0 0 0},
            breaklinks=true}
\urlstyle{same}  % don't use monospace font for urls
\usepackage{natbib}
\bibliographystyle{unified.bst}
\usepackage{color}
\usepackage{fancyvrb}
\newcommand{\VerbBar}{|}
\newcommand{\VERB}{\Verb[commandchars=\\\{\}]}
\DefineVerbatimEnvironment{Highlighting}{Verbatim}{commandchars=\\\{\}}
% Add ',fontsize=\small' for more characters per line
\usepackage{framed}
\definecolor{shadecolor}{RGB}{248,248,248}
\newenvironment{Shaded}{\begin{snugshade}}{\end{snugshade}}
\newcommand{\AlertTok}[1]{\textcolor[rgb]{0.94,0.16,0.16}{#1}}
\newcommand{\AnnotationTok}[1]{\textcolor[rgb]{0.56,0.35,0.01}{\textbf{\textit{#1}}}}
\newcommand{\AttributeTok}[1]{\textcolor[rgb]{0.77,0.63,0.00}{#1}}
\newcommand{\BaseNTok}[1]{\textcolor[rgb]{0.00,0.00,0.81}{#1}}
\newcommand{\BuiltInTok}[1]{#1}
\newcommand{\CharTok}[1]{\textcolor[rgb]{0.31,0.60,0.02}{#1}}
\newcommand{\CommentTok}[1]{\textcolor[rgb]{0.56,0.35,0.01}{\textit{#1}}}
\newcommand{\CommentVarTok}[1]{\textcolor[rgb]{0.56,0.35,0.01}{\textbf{\textit{#1}}}}
\newcommand{\ConstantTok}[1]{\textcolor[rgb]{0.00,0.00,0.00}{#1}}
\newcommand{\ControlFlowTok}[1]{\textcolor[rgb]{0.13,0.29,0.53}{\textbf{#1}}}
\newcommand{\DataTypeTok}[1]{\textcolor[rgb]{0.13,0.29,0.53}{#1}}
\newcommand{\DecValTok}[1]{\textcolor[rgb]{0.00,0.00,0.81}{#1}}
\newcommand{\DocumentationTok}[1]{\textcolor[rgb]{0.56,0.35,0.01}{\textbf{\textit{#1}}}}
\newcommand{\ErrorTok}[1]{\textcolor[rgb]{0.64,0.00,0.00}{\textbf{#1}}}
\newcommand{\ExtensionTok}[1]{#1}
\newcommand{\FloatTok}[1]{\textcolor[rgb]{0.00,0.00,0.81}{#1}}
\newcommand{\FunctionTok}[1]{\textcolor[rgb]{0.00,0.00,0.00}{#1}}
\newcommand{\ImportTok}[1]{#1}
\newcommand{\InformationTok}[1]{\textcolor[rgb]{0.56,0.35,0.01}{\textbf{\textit{#1}}}}
\newcommand{\KeywordTok}[1]{\textcolor[rgb]{0.13,0.29,0.53}{\textbf{#1}}}
\newcommand{\NormalTok}[1]{#1}
\newcommand{\OperatorTok}[1]{\textcolor[rgb]{0.81,0.36,0.00}{\textbf{#1}}}
\newcommand{\OtherTok}[1]{\textcolor[rgb]{0.56,0.35,0.01}{#1}}
\newcommand{\PreprocessorTok}[1]{\textcolor[rgb]{0.56,0.35,0.01}{\textit{#1}}}
\newcommand{\RegionMarkerTok}[1]{#1}
\newcommand{\SpecialCharTok}[1]{\textcolor[rgb]{0.00,0.00,0.00}{#1}}
\newcommand{\SpecialStringTok}[1]{\textcolor[rgb]{0.31,0.60,0.02}{#1}}
\newcommand{\StringTok}[1]{\textcolor[rgb]{0.31,0.60,0.02}{#1}}
\newcommand{\VariableTok}[1]{\textcolor[rgb]{0.00,0.00,0.00}{#1}}
\newcommand{\VerbatimStringTok}[1]{\textcolor[rgb]{0.31,0.60,0.02}{#1}}
\newcommand{\WarningTok}[1]{\textcolor[rgb]{0.56,0.35,0.01}{\textbf{\textit{#1}}}}
\usepackage{graphicx,grffile}
\makeatletter
\def\maxwidth{\ifdim\Gin@nat@width>\linewidth\linewidth\else\Gin@nat@width\fi}
\def\maxheight{\ifdim\Gin@nat@height>\textheight\textheight\else\Gin@nat@height\fi}
\makeatother
% Scale images if necessary, so that they will not overflow the page
% margins by default, and it is still possible to overwrite the defaults
% using explicit options in \includegraphics[width, height, ...]{}
\setkeys{Gin}{width=\maxwidth,height=\maxheight,keepaspectratio}
\setlength{\emergencystretch}{3em}  % prevent overfull lines
\providecommand{\tightlist}{%
  \setlength{\itemsep}{0pt}\setlength{\parskip}{0pt}}
\setcounter{secnumdepth}{5}
% Redefines (sub)paragraphs to behave more like sections
\ifx\paragraph\undefined\else
\let\oldparagraph\paragraph
\renewcommand{\paragraph}[1]{\oldparagraph{#1}\mbox{}}
\fi
\ifx\subparagraph\undefined\else
\let\oldsubparagraph\subparagraph
\renewcommand{\subparagraph}[1]{\oldsubparagraph{#1}\mbox{}}
\fi

%%% Use protect on footnotes to avoid problems with footnotes in titles
\let\rmarkdownfootnote\footnote%
\def\footnote{\protect\rmarkdownfootnote}

%%% Change title format to be more compact
\usepackage{titling}

% Create subtitle command for use in maketitle
\newcommand{\subtitle}[1]{
  \posttitle{
    \begin{center}\large#1\end{center}
    }
}

\setlength{\droptitle}{-2em}

  \title{Assessing midsaggital tongue contours in polar coordinates using
generalised additive (mixed) models}
    \pretitle{\vspace{\droptitle}\centering\huge}
  \posttitle{\par}
    \author{Stefano Coretta}
    \preauthor{\centering\large\emph}
  \postauthor{\par}
    \date{}
    \predate{}\postdate{}
  
\usepackage{cleveref}

\begin{document}
\maketitle

\hypertarget{introduction}{%
\section{Introduction}\label{introduction}}

Since the publication of the seminal paper by \citet{davidson2006},
statistical modelling of whole tongue contours obtained with ultrasound
imaging has been dominated by the use of Smoothing Splines Analysis of
Variance (SSANOVA, \citealt{gu2013}). These models have greatly advanced
our understanding of tongue articulation and speech modelling. On the
other hand, a variety of research disciplines is witnessing an increased
use of Generalised Additive Models (GAMs) and their mixed-effects
counterpart (GAMMs, \citealt{wood2006}), especially when dealing with
complex data. GAMs have being increasingly adopted in linguistics as a
means to model dynamic speech data. This paper introduces an
implementation of GAMs with tongue contours using polar coordinates. The
use of polar GAMs is illustrated with ultrasound tongue imaging data
comparing voiceless and voiced stops. The R package \texttt{rticulate}
has been developed to facilitate the use of the model, and it is briefly
introduced here.

\hypertarget{ultrasound-tongue-imaging}{%
\subsection{Ultrasound tongue imaging}\label{ultrasound-tongue-imaging}}

Ultrasound imaging is a non-invasive technique for obtaining an image of
internal organs and other body tissues. 2D ultrasound imaging has been
successfully used for imaging sections of the tongue surface (for a
review and applications in field settings, see \citealt{gick2002} and
\citealt{lulich2018}). An image of the (2D) tongue surface can be
obtained by placing the transducer in contact with the sub-mental
triangle (the area under the chin), aligned either with the mid-sagittal
or the coronal plane. The ultrasonic waves propagate from the transducer
in a radial fashion through the aperture of the mandible and get
reflected when they hit the air above the tongue surface. This `echo' is
captured by the transducer and translated into an image like the one
shown in \Cref{f:uti}.

\begin{figure}
  \centering
  \includegraphics{./img/uti.png}
  \caption{An ultrasound image showing a mid-sagittal view of the tongue. The white curved stripe in the image indicates where the ultrasonic waves have been reflected by the air above the tongue. The tongue surface corresponds to the lower edge of the white stripe. In this image, the tongue tip is located on the right. The green curve approximates the location of the palate.}
  \label{f:uti}
\end{figure}

\hypertarget{generalised-additive-models}{%
\subsection{Generalised Additive
models}\label{generalised-additive-models}}

Generalised additive models, or GAMs, are a more general form of
non-parametric modelling that allows fitting non-linear as well as
linear effects, and combine properties of linear and additive modelling
\citep{hastie1986}. GAMs are built with smoothing splines (like SSANOVA,
see \citealt{helwig2016}), which are defined piecewise with a set (the
\emph{basis}) of polynomial functions (the \emph{basis functions}). When
fitting GAMs, the smoothing splines try to maximise the fit to the data
while being constrained by a smoothing penalty (usually estimated from
the data itself). Such penalisation constitutes a guard against
overfitting. GAMs are thus powerful and flexible models that can deal
with non-linear data efficiently. Moreover, GAMs have a mixed-effect
counterpart, Generalised Additive Mixed Models (GAMMs), in which random
effects can be included (for a technical introduction to GAM(M)s, see
\citet{zuur2012} and \citet{wood2017}). GAMs can offer relief from
issues of autocorrelation between points of the contour (given that
points close to each other are not independent from one another). For
example, GAMs can fit separate smooths to individual contours, or a
first-order autoregression model can be included which tries to account
for the autocorrelation between each point in the contour and the one
following it. Tongue contours obtained from ultrasound imaging lend
themselves to be efficiently modelled using GAM(M)s.

\hypertarget{polar-coordinates}{%
\subsection{Polar coordinates}\label{polar-coordinates}}

\citet{mielke2015} and \citet{heyne2015a, heyne2015} have shown that
using polar coordinates of tongue contours rather than cartesian
coordinates brings several benefits, among which reduced variance at the
contour edges. Points in a polar coordinate system are defined by pairs
of radial and angular values. The point is describes with a radius,
which corresponds to the radial distance from the origin, and the angle
from the reference radius. Tongue contours, due to their shape, tend to
have increasing slope at the left and right edges, in certain cases
tending to become almost completely vertical. The almost verticality of
the contours has the effect of increasing the variance of the fitted
contours (and hence increased confidence intervals), and in some cases
it can even generate uninterpretable curves.

This issue is illustrated in \Cref{f:smooths}. The \emph{x} and \emph{y}
axes are the \emph{x} and \emph{y} cartesian coordinates in millimeters.
The plot shows LOESS smooths superimposed on the points of the
individual tongue contours of an Italian speaker (IT01, see
\Cref{s:data}). These contours refer to the mid-sagittal shape of the
tongue during the closure of four consonants (/t, d, k, g/) preceded by
one of three vowels (/a, o, u/). The tip of the tongue is on the
right-end side of each panel. Focussing on the smooths, it can be
noticed that the smooths in the contexts of the vowel /u/ diverge
substantially from the true contours (as inferred by the points). In the
contexts of velar consonants and the other two vowels, one could say
that the back/root of the tongue is somewhat flatted out relative to the
actual contours. These artefacts of smoothing happen because, especially
in the right-edge of these particular contours, the slope of the curve
increases in such a way that at times the curve bends under itself (see
for example the context /ug/, when \emph{x} is between -30 and -20).
Since those points on the bend share the \emph{x} value, the smooth just
averages across the \emph{y} values of those points.

\begin{figure}

{\centering \includegraphics[width=\linewidth]{2018-polar-gam_files/figure-latex/smooths-1} 

}

\caption{Estimated tongue contours of IT01 depending on C2 place, vowel and C2 voicing.}\label{f:smooths}
\end{figure}

\Cref{f:paths} shows a better way of representing individual tongue
contours. In these plots, the points of each contour are connected
sequentially by a line, rather than smoothed over. The parts in which
the contours bends over are kept and visualised correctly

\begin{figure}

{\centering \includegraphics[width=\linewidth]{2018-polar-gam_files/figure-latex/paths-1} 

}

\caption{Estimated tongue contours of IT01 depending on C2 place, vowel and C2 voicing.}\label{f:paths}
\end{figure}

These figures illustrate that using cartesian coordinates for modelling
can introduce smoothing artefacts which in turn can negatively affect
the model output. When tongue contours are expressed with polar
coordinates, on the other hand, the variance is reduced and the fitted
contours generally reflect more closely the underlying tongue shape.
Mielke has implemented a series of R \citep{r-core-team2018} functions
for fitting polar SSANOVAs to tongue contours. While model fitting is
achieved using polar coordinates, plotting of the model output is
subsequently obtained by reconverting the coordinates to cartesian. The
method introduce in this paper is based on Mielke's implementation
applied to GAMs.

\hypertarget{polar-gamms}{%
\section{Polar GAM(M)s}\label{polar-gamms}}

GAMs fitted to tongue contours in polar coordinates are introduced here.
A polar GAM is constructed as follows. The outcome variable of the model
are the radial coordinates, while a smooth term over the angular
coordinates is the predictor which takes care of modelling the curved
shape of the contour. Other predictors, such as consonant or vowel type,
speech rate, or random effects, can be also included. The model returns
fitted smooths with polar coordinates as units. The predicted polar
coordinates can be derived from the fitted smooths and converted into a
cartesian coordinate system (centred on the origin of the polar system)
for plotting. A simple example with data from one speaker will
illustrate how to fit polar GAMs with the R package \texttt{rticulate}.
The following section gives information on the ultrasonic system used
for data collection and on how the data has been processed, before
moving onto model fitting itself.

\hypertarget{data-collection-and-processing}{%
\subsection{Data collection and
processing}\label{data-collection-and-processing}}

\label{s:data}

Synchronised audio and ultrasound tongue imaging data have been recorded
from 4 speakers of Italian while reading a series of controlled
sentences. An Articulate Instruments Ltd™ set-up was used for this
study. The ultrasonic data was collected through a TELEMED Echo Blaster
128 unit with a TELEMED C3.5/20/128Z-3 ultrasonic transducer (20mm
radius, 2-4 MHz). A synchronisation unit (P-Stretch) was plugged into
the Echo Blaster unit and used for automatic audio/ultrasound
synchronisation. A FocusRight Scarlett Solo pre-amplifier and a Movo
LV4-O2 Lavalier microphone were used for audio recording. The
acquisition of the ultrasonic and audio signals was achieved with the
software Articulate Assistant Advanced (AAA, v2.17.2) running on a
Hawlett-Packard ProBook 6750b laptop with Microsoft Windows 7.
Stabilisation of the ultrasonic transducer was ensured by using the
metallic headset produced by Articulate Instruments Ltd™
(\citeyear{articulate2008}).

Before the reading task, the participant's occlusal plane was obtained
using a bite plate \citep{scobbie2011}. The participants read nonce
words embedded in the frame sentence \emph{Dico \_\_\_ lentamente} `I
say \_\_\_ slowly'. The words follow the structure
C\textsubscript{1}V́\textsubscript{1}C\textsubscript{2}V\textsubscript{2},
where C\textsubscript{1} = /p/, V\textsubscript{1} = /a, o, u/,
C\textsubscript{2} = /t, d, k, g/, and V\textsubscript{2} =
V\textsubscript{1}. Each speaker repeated the stimuli six times.

Spline curves were fitted to the visible tongue contours using the AAA
automatic tracking function. Manual correction was applied in those
cases that showed clear tracking errors. The time of maximum tongue
displacement within consonant closure was then calculated in AAA
following the method in \citet{strycharczuk2015}. A fan-like frame
consisting of 42 equidistant radial lines was used as the coordinate
system. The origin of the 42 fan-lines coincides with the centre of the
ultrasonic probe, such that each fan-line is parallel to the direction
of the ultrasonic signal. Tongue displacement was thus calculated as the
displacement of the fitted splines along the fan-line vectors. The time
of maximum tongue displacement was the time of greater displacement
along the fan-line vector that showed the greatest standard deviation
(as assessed manually). The vector standard deviation search area was
restricted to the portion of the contour corresponding to the tongue tip
for coronal consonants, and to the portion corresponding to the tongue
dorsum for velar consonants.

The cartesian coordinates of the tongue contours were extracted from the
ultrasonic data at the time of maximum tongue displacement (always
within C2 closure). The contours were subsequently normalised within
speaker by applying offsetting and rotation relative to the
participant's occlusal plane \citep{scobbie2011}. Each participants'
dataset is thus constituted by \emph{x} and \emph{y} coordinates of the
tongue contours that define respectively the horizontal and vertical
axes. The horizontal plane is parallel to the speaker's occlusal plane.

\hypertarget{fitting-a-polar-gam}{%
\subsection{Fitting a polar GAM}\label{fitting-a-polar-gam}}

GAMs can be fitted in R with the \texttt{gam()} function from package
\texttt{mgcv} \citep{wood2011, wood2017}. \texttt{bam()} is a more
efficient function when the datasets has several hundreds observations.
The package \texttt{rticulate} has been developed as a wrapper of the
\texttt{bam()} function to be used with tongue contours. The special
function \texttt{polar\_gam()} can fit any specified GAM model to tongue
contours coordinates, using the same syntax of \texttt{mgcv}. The
function accepts tongue contours either in cartesian or polar
coordinates. In the first case, the coordinates can be transformed into
polar before fitting. If the data is in the AAA fan-like coordinate
system, the origin is automatically estimated with the method in
\citet{heyne2015a}. If the data is not exported from AAA, the user can
specify the known coordinates of probe origin. The function
\texttt{plot\_polar\_smooths()}, used for plotting the estimated
contours, converts the coordinates back into cartesian using the same
origin as with GAM fitting.

A GAM in R can be specified with a formula that uses the same syntax of
\texttt{lme4}, a commonly used package for linear mixed-effects models
\citep{bates2015}. The \texttt{mgcv} package allows to specify smoothing
spline terms with the function \texttt{s()}. This function takes the
term along which a spline is created (for example, time in a time
series, or \emph{x}-coordinates in a cartesian system). Among the
arguments of \texttt{s()}, the user can select the type of spline (the
\texttt{bs} argument) and the grouping factor used for comparison (the
\texttt{by} argument). For a more in-depth introduction to GAMs in R for
linguistics, see \citet{soskuthy2017} and \citet{wieling2017}.

As means of illustration, the following paragraphs will show how to fit
a polar GAM with data from one of the 4 Italian speakers. Due to
differences in the placement of the probe and in the speakers' anatomy,
different portions of the tongue are likely to be imaged across
speakers, so that scaling might not be possible (or wise). For this
reason, it is recommended to fit separate models for each participant,
rather than aggregate all of the data in a single model.

We can start from a simple model in which we test the effect of C2
place, vowel, and voicing on tongue contours. \texttt{vc\_voicing} is an
ordered factor that specifies the combination of C2 place, vowel, and
voicing. Modelling different contours for each combination of the three
predictors can be achieved by using \texttt{vc\_voicing} with the
\texttt{by} argument of the difference smooth, and by including
\texttt{vc\_voicing} as a parametric term. The following code fits the
specified model to the contour data of IT01. When running the code, the
coordinates of the estimated origin used for the conversion to polar
coordinates are returned. The model is fitted by Maximum Likelihood (ML)
here to allow model comparison below).

\begin{Shaded}
\begin{Highlighting}[]
\NormalTok{it01_gam <-}\StringTok{ }\KeywordTok{polar_gam}\NormalTok{(}
\NormalTok{  Y }\OperatorTok{~}
\StringTok{    }\NormalTok{vc_voicing }\OperatorTok{+}\StringTok{            }\CommentTok{# parametric term}
\StringTok{    }\KeywordTok{s}\NormalTok{(X) }\OperatorTok{+}\StringTok{                  }\CommentTok{# reference smooth}
\StringTok{    }\KeywordTok{s}\NormalTok{(X, }\DataTypeTok{by =}\NormalTok{ vc_voicing),  }\CommentTok{# difference smooth}
  \DataTypeTok{data =}\NormalTok{ tongue_it01,}
  \DataTypeTok{method =} \StringTok{"ML"}
\NormalTok{)}
\end{Highlighting}
\end{Shaded}

\begin{verbatim}
## The origin is x = 14.3900999664996, y = -65.2314226131983.
\end{verbatim}

The function \texttt{plot\_polar\_smooths()} can be used to plot the
estimated contours. The shaded areas around the estimated contours are
95\% confidence intervals. Note that, differently from SSANOVA,
statistical significance can't be assessed from the overlapping (or lack
thereof) of the confidence intervals.

\begin{Shaded}
\begin{Highlighting}[]
\KeywordTok{plot_polar_smooths}\NormalTok{(}
\NormalTok{  it01_gam,}
\NormalTok{  X,}
\NormalTok{  voicing,}
  \DataTypeTok{facet_terms =}\NormalTok{ c2_place }\OperatorTok{+}\StringTok{ }\NormalTok{vowel,}
  \CommentTok{# the following splits the factor interaction in the individual terms,}
  \CommentTok{# so that they can called in the plotting arguments}
  \DataTypeTok{split =} \KeywordTok{list}\NormalTok{(}\DataTypeTok{vc_voicing =} \KeywordTok{c}\NormalTok{(}\StringTok{"vowel"}\NormalTok{, }\StringTok{"c2_place"}\NormalTok{, }\StringTok{"voicing"}\NormalTok{))}
\NormalTok{) }\OperatorTok{+}
\StringTok{  }\KeywordTok{coord_fixed}\NormalTok{() }\OperatorTok{+}
\StringTok{  }\KeywordTok{theme}\NormalTok{(}\DataTypeTok{legend.position =} \StringTok{"top"}\NormalTok{)}
\end{Highlighting}
\end{Shaded}

\begin{figure}

{\centering \includegraphics[width=\linewidth]{2018-polar-gam_files/figure-latex/it01-gam-plot-1} 

}

\caption{Estimated tongue contours of IT01 depending on C2 place, vowel and C2 voicing.}\label{f:it01-gam-plot}
\end{figure}

One way to assess significance of model terms is to compare the ML score
of the full model against one without the relevant predictor, using the
function \texttt{compareML()} from the \texttt{itsadug} package. Both
the parametric term and the difference smooth need to be removed in the
null model.

\begin{Shaded}
\begin{Highlighting}[]
\NormalTok{it01_gam_}\DecValTok{0}\NormalTok{ <-}\StringTok{ }\KeywordTok{polar_gam}\NormalTok{(}
\NormalTok{  Y }\OperatorTok{~}
\StringTok{    }\CommentTok{# vc_voicing +            # remove parametric term}
\StringTok{    }\KeywordTok{s}\NormalTok{(X),                     }\CommentTok{# keep reference smooth}
    \CommentTok{# s(X, by = vc_voicing),  # remove difference smooth}
  \DataTypeTok{data =}\NormalTok{ tongue_it01,}
  \DataTypeTok{method =} \StringTok{"ML"}
\NormalTok{)}
\end{Highlighting}
\end{Shaded}

\begin{verbatim}
## The origin is x = 14.3900999664996, y = -65.2314226131983.
\end{verbatim}

\begin{Shaded}
\begin{Highlighting}[]
\KeywordTok{compareML}\NormalTok{(it01_gam_}\DecValTok{0}\NormalTok{, it01_gam)}
\end{Highlighting}
\end{Shaded}

\begin{verbatim}
## it01_gam_0: Y ~ s(X)
## 
## it01_gam: Y ~ vc_voicing + s(X) + s(X, by = vc_voicing)
## 
## Chi-square test of ML scores
## -----
##        Model     Score Edf Difference     Df  p.value Sig.
## 1 it01_gam_0 12395.227   3                                
## 2   it01_gam  7423.356  36   4971.871 33.000  < 2e-16  ***
## 
## AIC difference: 10258.65, model it01_gam has lower AIC.
\end{verbatim}

To check which part of the contour differs among conditions, the method
recommended in \citet{soskuthy2017} is to plot the difference smooth and
check the confidence interval. The parts of confidence interval that
don't include 0 indicate that the difference between contours in that
part is significant. \Cref{f:diff-it01} illustrates the use of
difference smooths with the difference smooths of voiceless vs.~voiced
coronal stops when the vocalic context is /a/ or /u/. As per usual, the
tongue tip is on the right-end side of each plot. The difference smooths
indicate that there is a significant difference along the most anterior
part of the tongue (the root and dorsum). Based on the predicted smooths
sown in \Cref{f:it01-gam-plot}, we can argue that, in the context of
coronal consonants, the root is more advanced in voiced relateive to
voiceless stops (when the vowel is either /a/ or /u/), and that the
dorsum is also somewhat retracted in voiced stops if the vowel is /u/.

\begin{figure}

{\centering \includegraphics[width=.7\linewidth,height=5cm]{2018-polar-gam_files/figure-latex/diff-it01-1} \includegraphics[width=.7\linewidth,height=5cm]{2018-polar-gam_files/figure-latex/diff-it01-2} 

}

\caption{Difference smooth of voiceless vs. voiced stops in the context of /a/ (left) and /u/ (right).}\label{f:diff-it01}
\end{figure}

As mentioned in the introduction, autocorrelation in the data can
produce unwanted patterns in the residuals, which in turn can affect the
estimated smooths (and falsely increase certainty about them). A
first-order autoregressive (AR1) model can be included to reduce
autocorrelation at lag 1. \Cref{f:it01-acf} show the autocorrelations in
the residuals without and with an AR1 model. The GAM model with the AR1
correction has lower values of autocorrelations. In this case, it is
thus advisable to perform ML comparison and smooths plotting with models
in which an AR1 model has been included. For a more in-depth treatment
of issues related to autocorrelation, see \citet{soskuthy2017}.

\begin{figure}

{\centering \includegraphics[width=.7\linewidth,height=5cm]{2018-polar-gam_files/figure-latex/it01-acf-1} \includegraphics[width=.7\linewidth,height=5cm]{2018-polar-gam_files/figure-latex/it01-acf-2} 

}

\caption{Autocorrelation plots of a model fitted without (left) and with (right) a first-order autoregressive model (AR1).}\label{f:it01-acf}
\end{figure}

\hypertarget{comparing-tongue-contours-in-voiceless-and-voiced-stops}{%
\section{Comparing tongue contours in voiceless and voiced
stops}\label{comparing-tongue-contours-in-voiceless-and-voiced-stops}}

Mid-sagittal tongue contours at maximum tongue displacement of voiceless
and voiced stops have been compared using polar GAMs.
\Cref{f:tongues-it01} to \Cref{f:tongues-it04} show an appreciable
degree of variation across speakers and phonological contexts in
relation to the differences in tongue shapes between voiceless and
voiced stops. In some speakers and contexts, the tongue root (the left
of the tongue contours) is more advanced in voiced stops than in
voiceless stops, especially in the context of a coronal C2 and /a/.
Tongue root advancement is a well known mechanism employed to maintain
intra-oral pressure below the threshold required for voicing
\citep{ohala2011, kent1969, perkell1969, westbury1983, ahn2018}.

The magnitude of the difference in tongue root position in the data
reported here is about 2 millimetres. \citet{kirkham2017} find that the
tongue root in +ATR vowels is on average 4 millimetres more advanced
than the respective −ATR vowels. \citet{rothenberg1967} argues, based on
modelling, that the tongue root can move forward by a maximum of about 5
mm mid-sagittally. This movement corresponds to an average volume
increase of 18 cm\textsuperscript{2}. Given these estimates, it can be
argued that a 2 mm change reasonably contributes to an appreciable
volume increase, also considering that other volume expansion mechanisms
can operate along with the advancement of the tongue root (like larynx
lowering, slack oral walls, etc.).

\begin{figure}

{\centering \includegraphics[width=.8\textwidth]{2018-polar-gam_files/figure-latex/tongues-it01-1} 

}

\caption{Tongue contours of voiceless and voiced stops in IT01.}\label{f:tongues-it01}
\end{figure}

\begin{figure}

{\centering \includegraphics[width=.8\textwidth]{2018-polar-gam_files/figure-latex/tongues-it02-1} 

}

\caption{Tongue contours of voiceless and voiced stops in IT02.}\label{f:tongues-it02}
\end{figure}

\begin{figure}

{\centering \includegraphics[width=.8\textwidth]{2018-polar-gam_files/figure-latex/tongues-it03-1} 

}

\caption{Tongue contours of voiceless and voiced stops in IT03.}\label{f:tongues-it03}
\end{figure}

\begin{figure}

{\centering \includegraphics[width=.8\textwidth]{2018-polar-gam_files/figure-latex/tongues-it04-1} 

}

\caption{Tongue contours of voiceless and voiced stops in IT04.}\label{f:tongues-it04}
\end{figure}

\hypertarget{conclusions}{%
\section{Conclusions}\label{conclusions}}

Generalised additive (mixed) models (GAMs) can be efficiently used to
statistically assess differences in tongue contour shapes as obtained
from ultrasound tongue imaging. This paper showed how GAMs can be fitted
to tongue contours in polar coordinates in R with the specialised
package \texttt{rticulate}. An example of how GAMs can help modelling
differences in tongue contours has been illustrated with data from 4
speakers of Italian in which the mid-sagittal tongue contours of
voiceless and voiced stops where compared. The same general advantages
and issues noted in \citet{davidson2006} for SSANOVA apply to polar
GAMs. In particular, while within-speaker normalisation can be achieved
by rotation and offsetting of the data relative to a bite plate (as done
here), across-speaker normalisation represents a bigger challenge. Since
it can't be deduced with sufficient certainty from the ultrasonic image
which part of the tongue is being actually imaged, it is not possible to
define fixed anatomical landmarks across speakers that can be used in
normalisation. For this reason it has been recommended here to fit
separate models for each speaker. Future work will explore ways of
allowing the user to use data aggregated from multiple speakers while
accounting for the uncertainty in which parts of the tongue are imaged.
To conclude, polar GAMs can also be extended to model whole tongue
contours differences over time (in other words, how the sectional shape
of the tongue changes over time) and 3D tongue surfaces.

\bibliography{linguistics.bib}


\end{document}
