\documentclass[]{article}
\usepackage{lmodern}
\usepackage{amssymb,amsmath}
\usepackage{ifxetex,ifluatex}
\usepackage{fixltx2e} % provides \textsubscript
\ifnum 0\ifxetex 1\fi\ifluatex 1\fi=0 % if pdftex
  \usepackage[T1]{fontenc}
  \usepackage[utf8]{inputenc}
\else % if luatex or xelatex
  \ifxetex
    \usepackage{mathspec}
  \else
    \usepackage{fontspec}
  \fi
  \defaultfontfeatures{Ligatures=TeX,Scale=MatchLowercase}
    \setmainfont[]{Brill}
\fi
% use upquote if available, for straight quotes in verbatim environments
\IfFileExists{upquote.sty}{\usepackage{upquote}}{}
% use microtype if available
\IfFileExists{microtype.sty}{%
\usepackage{microtype}
\UseMicrotypeSet[protrusion]{basicmath} % disable protrusion for tt fonts
}{}
\usepackage[margin=1in]{geometry}
\usepackage{hyperref}
\hypersetup{unicode=true,
            pdftitle={Response to reviews},
            pdfborder={0 0 0},
            breaklinks=true}
\urlstyle{same}  % don't use monospace font for urls
\usepackage{graphicx,grffile}
\makeatletter
\def\maxwidth{\ifdim\Gin@nat@width>\linewidth\linewidth\else\Gin@nat@width\fi}
\def\maxheight{\ifdim\Gin@nat@height>\textheight\textheight\else\Gin@nat@height\fi}
\makeatother
% Scale images if necessary, so that they will not overflow the page
% margins by default, and it is still possible to overwrite the defaults
% using explicit options in \includegraphics[width, height, ...]{}
\setkeys{Gin}{width=\maxwidth,height=\maxheight,keepaspectratio}
\IfFileExists{parskip.sty}{%
\usepackage{parskip}
}{% else
\setlength{\parindent}{0pt}
\setlength{\parskip}{6pt plus 2pt minus 1pt}
}
\setlength{\emergencystretch}{3em}  % prevent overfull lines
\providecommand{\tightlist}{%
  \setlength{\itemsep}{0pt}\setlength{\parskip}{0pt}}
\setcounter{secnumdepth}{0}
% Redefines (sub)paragraphs to behave more like sections
\ifx\paragraph\undefined\else
\let\oldparagraph\paragraph
\renewcommand{\paragraph}[1]{\oldparagraph{#1}\mbox{}}
\fi
\ifx\subparagraph\undefined\else
\let\oldsubparagraph\subparagraph
\renewcommand{\subparagraph}[1]{\oldsubparagraph{#1}\mbox{}}
\fi

%%% Use protect on footnotes to avoid problems with footnotes in titles
\let\rmarkdownfootnote\footnote%
\def\footnote{\protect\rmarkdownfootnote}

%%% Change title format to be more compact
\usepackage{titling}

% Create subtitle command for use in maketitle
\providecommand{\subtitle}[1]{
  \posttitle{
    \begin{center}\large#1\end{center}
    }
}

\setlength{\droptitle}{-2em}

  \title{Response to reviews}
    \pretitle{\vspace{\droptitle}\centering\huge}
  \posttitle{\par}
    \author{}
    \preauthor{}\postauthor{}
      \predate{\centering\large\emph}
  \postdate{\par}
    \date{21/01/2019}

\usepackage{xcolor}
\usepackage{changepage}

\begin{document}
\maketitle

\hypertarget{editor}{%
\section{Editor}\label{editor}}

I have one major concern about your paper. In the main part of your
paper, you present results pooled across subjects. These results show
some agreement with your hypotheses. However, in Section C. Individual
differences, you show that several individuals do not follow the general
trends that you have found in the pooled results. So, why not only
present the individual results, since they suggest variability across
subjects?

I also think that you need to say something about the resolution of the
ultrasound images, since, as noted by Reviewer 1, some movements are
very small. In addition, I would like to see some general comments on
tongue movements during speech, in particular when you discuss tongue
root movement and vowel duration.

I elaborate on these and other issues below.

Line 17 Fowler, 1992 and Lisker 1974, d not present any data on tongue
root movements.

\begin{adjustwidth}{1cm}{} \textit{
They referred to vowel duration. I've expanded the list as per reviewer's request and moved the references to the next sentence.
} \end{adjustwidth}

Lines 21 and 24 Here, you make claims about earlier works that need to
be referenced.

\begin{adjustwidth}{1cm}{} \textit{
Line 21 refers to this study, so I changed to "In this exploratory study". I added references on line 24.
} \end{adjustwidth}

Line 72 \ldots{}is generally longer than that of voiceless
stops\ldots{}" \textgreater{} \ldots{}is generally shorter than that of
voiceless stops\ldots{}"

\begin{adjustwidth}{1cm}{} \textit{
Fixed.
} \end{adjustwidth}

Page 6, second paragraph Since you bring up possible voicing differences
between Italian and Polish, I think it would help if you made clear
whether the voiced stops in your material were actually produced with
glottal vibrations., If they were not, there would be no need for tongue
movements to maintain voicing. Possibly, some of the individual
differences may be due to voicing variability.

\begin{adjustwidth}{1cm}{} \textit{
The paragraph mentions differences in the effect of voicing on vowel duration, rather than differences of implementation of consonantal voicing. The EGG data of the study (not discussed here) indicates that virtually all voiced tokens were uttered with vocal fold vibration. I added a footnote in Section I.B which mentions this: "Simultaneous electroglottographic data (not discussed here) was also collected during the experiment. This data indicates that virtually all tokens of voiced stops were uttered with vocal fold vibration, with just a few exceptions (4 tokens were voiceless in the speaker PL02)."
} \end{adjustwidth}

Page 11, E, Data processing and statistical analysis. I would suggest
saying something about the spatial resolution of the ultrasound system
and the reliability of the splines. From Figure 1, it is not immediately
clear that you are actually measuring the tongue root.

\begin{adjustwidth}{1cm}{} \textit{
I have added the following sentence in Section II.A: "The mean pixel size as used by the automatic tracker was 0.47 mm (SD = 0.16), so that differences in tongue position smaller than that would not be captured."
As for measuring tongue root displacement, see answer to Reviewer 1 L216-8
} \end{adjustwidth}

Page 17, C. Correlation between tongue rot position and V1 duration Why
is this interesting and relevant? Since the tongue is always moving in
speech, would it not move longer during a long vowel? The tongue does
not appear to have a point target for vowels.

\begin{adjustwidth}{1cm}{} \textit{
As far as I know, correlation between vowel duration and tongue root position at V1 offset have not been reported before. One possible obvious explanation is that a longer vowel allows for more tongue root advancement (as I argue in the paper). But it could have been that tongue root advancement is initiated at different times during the production of vowels in different vowel duration contexts, so that the same or similar amount of displacement is reached at the end of the vowel independent of its duration. Rather, what the positive correlation interestingly suggests is that the onset of the advancement gesture during the production of the vowel is initiated at a stable temporal distance from the release of C1/onset of V1.
} \end{adjustwidth}

Line 302 Can you explain wat you mean with ``curvature''?

\begin{adjustwidth}{1cm}{} \textit{
I have removed this term throughout the paper and used simpler terminology to describe the movement of the tongue root along the duration of the vowel.
} \end{adjustwidth}

Line 333 ``\ldots{}the onset of the forward gesture of the root is timed
relative to a landmark preceding the closure, independent of the
duration of the vowel.'' I don't understand what this means. What kind
of landmark are you talking about?

\begin{adjustwidth}{1cm}{} \textit{
I am sorry, the wording was most unclear. I have updated the paragraph to hopefully make it clearer. What I am try to say is that the data suggests the following. If we take the release of C1 as a temporal landmark, then the onset of the tongue root advancement gesture would happen at a fixed temporal distance from C1 release, independent of the duration of the vowel. Of course, C1 release is just chose for convenience, rather than for a theoretical reason (as discussed in Coretta 2018).
} \end{adjustwidth}

Line 391 The duration of 70-90 ms suggested ny Rothenberg (1967) is not
based on actual measurements.

\begin{adjustwidth}{1cm}{} \textit{
Updated the sentence to "Rothenberg 1967 hypothesised, after an informal investigation, that a maximal ballistic expansion movement of the tongue root to increase the size of the lower pharynx would take 70--90 ms" as suggested by one reviewer below.
} \end{adjustwidth}

Figure 6 The labels and the y-axis values are identical in the left and
right plots. The two plots appear to be identical.

\begin{adjustwidth}{1cm}{} \textit{
That was a coding error, now fixed.
} \end{adjustwidth}

References

Ahn. 2015, is available here:
\url{http://www.ultrafest2015.hku.hk/docs/S_Ahn_ultrafest.pdf}

Ahn \& Davidson, 2016, is an abstract, thus not very useful and should
be left out.

Halle \& Stevens, 1967, is an abstract, thus not very useful and should
be left out,. The

Keating, 1984. Page numbers are 35-49, and the text is available here:
\url{https://escholarship.org/uc/item/2497n8jq}

Line 567 ``English''

Lindblom, 1967, is available here:
\url{http://www.speech.kth.se/prod/publications/files/qpsr/1967/1967_8_4_001-029.pdf}

Maddieson \& Gandor, 1976, is availanlde here:
\url{https://escholarship.org/uc/item/31f5j8m7}

Malisz \& Klessa, 2008, is available here:
\url{http://www.isle.illinois.edu/sprosig/sp2008/papers/id182.pdf}

Line 624 ``Phonetics''

\begin{adjustwidth}{1cm}{} \textit{
All changes to references have been applied.
} \end{adjustwidth}

\hypertarget{reviewer-1}{%
\section{Reviewer 1}\label{reviewer-1}}

Title: Longer vowel duration correlates with greater tongue root
displacement: Acoustic and articulatory data from Italian and Polish

Summary -- this is an articulatory (ultrasound tongue imaging) study of
tongue-root position in the production of CV.CV nonsense words in
Italian and Polish, focusing on tongue-root position during V1 and at
the onset of C2, while varying whether C2 is a phonemically voiced or
voiceless consonant. The main aim of the study is to determine whether
tongue root advancement is more extreme during the vowel and at C2 onset
when a C2 is voiced than when it is voiceless. It is hypothesized that
presence of increased tongue-root advancement when C2 is voiced is a
mechanism for increasing the supraglottal cavity and prolonging the
transglottal airflow needed to maintain voicing during the C2 closure.
Mixed effects modelling is used to identify the impact of C2
voiced/voiceless status (and other factors) on tongue root position at
C2 onset. C2 voicing is found to have a significant effect on
tongue-root position at C2 onset, with tongue-root onset 0.77mm fronter
when C2 is a voiced consonant. A generalised additive mixed model is
used to assess the impact of C2 voicing (and other factors) on
tongue-root position across V1. The model shows a significant effect of
C2 voicing, particularly during the latter half of the vowel. Root
advancement is present during the vowel before both voiced and voiceless
C2s, but to a greater degree before voiced C2s. A further mixed effects
model showed a significant effect of V1 duration on tongue-root
advancement, with longer vowels resulting in greater tongue-root
advancement. I think it is argued that having a later voiced C2 onset
and, concomitantly, a longer preceding vowel, allows for more extreme
tongue-root advancement at C2 onset and that therefore the shorter
voiced C2 closed phase permits greater expansion of the supralaryngeal
cavity to aid transglottal airflow and maintain voicing, while also
ensuring that voicing needs to be maintained for a shorter time period.

General comments -- the findings of this paper are interesting and seem
to show a plausible 2- fold effect of later consonant onset on the
maintenance of coronal plosive voicing contrasts. I did find this paper
a bit convoluted to read and I found that I didn't have a clear idea of
what the author was trying to say. Additionally, raw differences in
tongue-root advancement at voiced and voiceless consonant onset, for all
they are statistically significant, are extremely subtle \textless{}1mm,
while UTI images provide a somewhat blurred image of the tongue surface,
and there is potential for error during (automatic) spline fitting due
to the fuzziness of the image. I also worry about the measurement
protocol itself. The ultrasound probe does not seem to have been angled
towards the pharynx and yet the probe angles are quite narrow (71-93°).
The example given in FIG. 1 does not contain a hyoid shadow, therefore
it is difficult to determine whether movement along the selected radial
fan axis shown is really capturing root retraction, or some point
further up the back of the tongue. Another issue is that an increase in
the value of the point at which the tongue spline intersects this chosen
radial fan axis is taken to indicate greater tongue-root advancement,
whereas tongue-root advancement should result in a lower value -- as the
tongue root moves forwards, it should move closer to the origin of the
radial fan line. It is possible that the author reversed the
radial-fan-line intersect values, as in (Kirkham and Nance, 2017), in
which case they should state this explicitly. I believe the issues with
measurement can be cleared up with some better figures and some
assurances about how the measures were taken and transformed, so I
suggest revise and resubmit.

Specific comments:

I think some of the content of the section l67-88 should be moved closer
to the beginning of the introduction §A, to give context to the ATR
focus of the study i.e.~there should be an acknowledgement that there
are a range of articulatory strategies associated with plosive voicing
and then you can say that you are going to focus on one of these
features in this study.

\begin{adjustwidth}{1cm}{} \textit{
This suggestion has been implemented in Section I.A. Now the section starts with a general overview of some of the mechanisms for voicing maintenance and then focusses on tongue root advancement.
} \end{adjustwidth}

1-2 ``Voiced stops tend to be preceded by longer vowels and produced
with a more advanced tongue root than voiceless stops.'' Citation
missing.

\begin{adjustwidth}{1cm}{} \textit{
Citations to a long list of publications is given at multiple points within the manuscript, so I think it would not be practical to include them in the abstract.
} \end{adjustwidth}

2 Is ``modulated'' the right word here? ``modified/affected''. It
doesn't seem to be the right word, given that the voicing you are
discussing is phonemic and not necessarily present at the phonetic
level, as you point out later.

\begin{adjustwidth}{1cm}{} \textit{
I changed modulated to "modified/affected".
} \end{adjustwidth}

3-4 ``in many languages vowels are longer when followed by voiced
stops.'' Examples and citations missing.

\begin{adjustwidth}{1cm}{} \textit{
See comment above.
} \end{adjustwidth}

4-6 ``Tongue root advancement is known to be an articulatory mechanism
which ensures the right pressure conditions for the maintenance of
voicing during closure as dictated by the Aerodynamic Voicing
Constraint.'' Citation missing.

\begin{adjustwidth}{1cm}{} \textit{
See comment above.
} \end{adjustwidth}

7 ``enter in a direct statistical relation''. This phrase sounds odd.
Rephrase as ``have a direct statistical relationship''?, or later in
lines 25\&26, you say ``acoustic duration of the vowel is positively
correlated with tongue root position''. Can you not say that here?

\begin{adjustwidth}{1cm}{} \textit{
Changed to "have a direct statistical relationship".
} \end{adjustwidth}

8-9 ``17 speakers of Italian and Polish (in total)''

\begin{adjustwidth}{1cm}{} \textit{
Added "in total".
} \end{adjustwidth}

12-13 ``in a temporally stable interval''. Perhaps it would be better to
say ``comparatively later closure onset of voiced stops'', because it is
not clear without context what ``stable interval'' you mean -- C1
release to C2 release.

\begin{adjustwidth}{1cm}{} \textit{
Changed as suggested.
} \end{adjustwidth}

12-14 It feels as if these lines belong earlier in the paragraph to make
the proposed chain of causal relationships clear, i.e.~ATR allows
maintenance of voicing for voiced stops, ATR correlates with vowel
length.

\begin{adjustwidth}{1cm}{} \textit{
Changed as suggested.
} \end{adjustwidth}

16 ``characterised'' is not the right word here, given that ATR is
covert, ``associated with''?

\begin{adjustwidth}{1cm}{} \textit{
Changed as suggested.
} \end{adjustwidth}

18-19 ``a lot of work'' citations missing. ``aspects'', change to
``phonetic features''? ``phonetic correlates''? Also l19
``relationship'', rather than ``relation''.

\begin{adjustwidth}{1cm}{} \textit{
A selection of citations from later sections has been included here.
} \end{adjustwidth}

21 ``in an exploratory'' -- change to ``in this exploratory''.

\begin{adjustwidth}{1cm}{} \textit{
Changed as suggested.
} \end{adjustwidth}

24 ``this replicates previous work on tongue root position'' --
citation(s) missing.

\begin{adjustwidth}{1cm}{} \textit{
Citations included.
} \end{adjustwidth}

25 ``\ldots{} indicate that the acoustic duration of the vowel is
positively correlated with the tongue root position (at vowel offset/
postvocalic consonant onset)\ldots{}''

\begin{adjustwidth}{1cm}{} \textit{
Changed as suggested.
} \end{adjustwidth}

24-8 There is something oddly circular about this section. Compare to
the way you talk about the trade-off between vowel and consonant length
in lines 121-2. To avoid a circular argument, I would rephrase as:
``Furthermore, the results of this study indicate that a comparatively
later C2 onset for voiced consonants results in a longer preceding vowel
duration which, in turn, results in greater tongue-root advancement
during C2 onset. Both the shorter closed phase of the voiced consonant
and the more advanced tongue root, which expands the supra-glottal
cavity, have the potential to maintain voicing throughout C2 and
preserve the voicing contrast.''

\begin{adjustwidth}{1cm}{} \textit{
Changed as suggested.
} \end{adjustwidth}

L32 ``relative to the front-back dimension'' -- ``across/in the
front-back dimension''?

\begin{adjustwidth}{1cm}{} \textit{
Fixed.
} \end{adjustwidth}

L33 ``This has'' -- change to ``This finding has\ldots{}''

\begin{adjustwidth}{1cm}{} \textit{
Fixed.
} \end{adjustwidth}

L43 I think it would be worth mentioning here that there are other
articulatory strategies employed to increase supralaryngeal space,
e.g.~larynx lowering (Rothenberg, 1967). See first comment above about
moving content of lines 67-88 to beginning of introductory section.

\begin{adjustwidth}{1cm}{} \textit{
Changed as suggested.
} \end{adjustwidth}

L52 imply

\begin{adjustwidth}{1cm}{} \textit{
Fixed.
} \end{adjustwidth}

L57 ``closure of a stop'' -- ``closure of a lingual stop''.

\begin{adjustwidth}{1cm}{} \textit{
Fixed.
} \end{adjustwidth}

L157-58 Rephrase as: "Rothenberg (1967) hypothesized after an informal
investigation, that a maximal ballistic expansion movement of the tongue
root to increase the size of the lower pharynx would take 70-90ms
(Rothenberg, 1967: 99).

\begin{adjustwidth}{1cm}{} \textit{
Fixed.
} \end{adjustwidth}

P65-66 ``While the tongue body is more involved in labials''. Can you be
more specific? What does the tongue body do?

\begin{adjustwidth}{1cm}{} \textit{
I specified that tongue body lowering is common with labials.
} \end{adjustwidth}

L92 ``it can be'' -\textgreater{} ``it is sometimes''

\begin{adjustwidth}{1cm}{} \textit{
Fixed.
} \end{adjustwidth}

L104 typo ``to study of the''

\begin{adjustwidth}{1cm}{} \textit{
Fixed.
} \end{adjustwidth}

L105-6 ``given their reported differences in magnitude/presence of the
effect and the relative ease of comparison.'' -- you need to unpack this
a bit.

\begin{adjustwidth}{1cm}{} \textit{
I've expanded with "Moreover, the segmental phonologies of these languages facilitate the design of sufficiently comparable experimental material (see Coretta 2018 for a more thorough discussion)".
} \end{adjustwidth}

L104-106 I suggest this edit: ``Italian and Polish offer an opportunity
to study of the articulatory aspects of the voicing effect, given that
the former has been consistently reported as a voicing- effect language,
while voicing effect in the latter is more complex, with some studies
finding an effect and others not.''

\begin{adjustwidth}{1cm}{} \textit{
Changed.
} \end{adjustwidth}

L110-12 What were the mean voicing differences for the two languages?
Were they comparable?

\begin{adjustwidth}{1cm}{} \textit{
I've added details on the raw mean differences of the voicing effect in the two languages and mentioned that such difference is not significant in the linear model.
} \end{adjustwidth}

L112-114 Can you explain further? ``The high degree of intra-speaker
variation, backed up by statistical modelling, also indicates that these
languages possibly behave similarly in regards to the voicing effect.''

\begin{adjustwidth}{1cm}{} \textit{
I've corrected "intra-speaker" with "inter-speaker" and expanded on this statment by indicating that independent of language speakers show a range of possible magnitudes of the effect.
} \end{adjustwidth}

L124 Can you come up with a better section heading?

\begin{adjustwidth}{1cm}{} \textit{
Changed to "Rationale of the current study".
} \end{adjustwidth}

L126 Citations missing.

\begin{adjustwidth}{1cm}{} \textit{
I inserted here some of the citations from above for clarity.
} \end{adjustwidth}

L125-131 Suggested rephrasing ``Previous research has established that
longer preconsonantal vowel durations and greater tongue root
advancement are associated with voicing in postvocalic plosives. We know
that voicing during plosive closure can be sustained by advancing the
tongue root during the production of voiced plosives and that tongue
root advancement probably begins before the closure onset (i.e.~during
the preceding vowel). We also know that vowels followed by voiced
plosives tend to be longer than vowels followed by voiceless plosives.
Acoustic analysis of the current dataset confirmed an apparent
compensatory relationship between the duration of the plosive closure
and the duration of the preceding vowel; the shorter the plosive
closure, the longer the preceding vowel.''

\begin{adjustwidth}{1cm}{} \textit{
Rephrased as suggested.
} \end{adjustwidth}

L133 ``insights on the link'' -- ``insights into the link'', ``between
closure'' -- ``between closure duration''?? or ``link between closure
and vowel durations''.

\begin{adjustwidth}{1cm}{} \textit{
Changed.
} \end{adjustwidth}

L134 I am not comfortable with the term ``modulates''. I think it
implies a causal relationship that has not yet been proved. Can you say
``correlates''? or ``covaries with''?

\begin{adjustwidth}{1cm}{} \textit{
Changed.
} \end{adjustwidth}

L135-6 Suggested rephrasing ``resulting in a three-way relationship
between stop consonant duration, vowel duration and tongue-root
advancement.''

\begin{adjustwidth}{1cm}{} \textit{
Changed.
} \end{adjustwidth}

L136-8 ``More specifically, the timing of the closure onset within the
release-to-release interval decides the duration of the vowel\ldots{}''
Again, is this not a given, see comments l27-28.

\begin{adjustwidth}{1cm}{} \textit{
Yes The sentence now says "More specifically, the timing of the closure onset within the release-to-release interval determines not only the duration of the vowel and that of the closure (as discussed in Coretta 2018), but also the degree of tongue root advancement at V1 offset/C2 onset."
} \end{adjustwidth}

L159 Can you change to ``TELEMED Echo Blaster 128 unit, with a
C3.5/20/128Z-3 ultrasonic transducer (20mm radius, 2-4 MHz).'' Grouping
these items together makes sense. Otherwise you are just writing a
random list of items that you used in your recording set up.

\begin{adjustwidth}{1cm}{} \textit{
Changed.
} \end{adjustwidth}

L160-1 The pre-amp is part of the audio recording set up, not the
ultrasound recording system. You need to move it further down the list
of equipment, i.e.~``a Movo LV4-O2 Lavalier microphone with a FocusRight
Scarlett Solo pre-amplifier.''

\begin{adjustwidth}{1cm}{} \textit{
Changed.
} \end{adjustwidth}

L163 Please gloss ``sub-mental triangle'', and in what way was the probe
``aligned with the midsagittal plane''?

\begin{adjustwidth}{1cm}{} \textit{
Now the line reads "The ultrasonic probe was placed in contact with the flat area below the chin, aligned along the participant's mid-sagittal plane so that the mid-sagittal profile of the tongue could be imaged."
} \end{adjustwidth}

L169 "by means of a synchronisation signal produced by the ultrasound
unit and amplified by the P-Stretch unit.

\begin{adjustwidth}{1cm}{} \textit{
Changed.
} \end{adjustwidth}

P170-2 Does this mean that the ultrasound recording settings varied in
every recording session, or between the Italian and Polish recording
settings? If the latter, can you give the separate setting by language,
rather than a range. Can you also say why you think this variation won't
affect what you are measuring? The range of probe angles seem quite
narrow.

L184 ``explore timing/ articulatory? differences in the closing gesture
of voiceless and voiced stops\ldots{}''

\begin{adjustwidth}{1cm}{} \textit{
Changed.
} \end{adjustwidth}

L202-9 Can you add a figure showing an example of annotation?

\begin{adjustwidth}{1cm}{} \textit{
Added.
} \end{adjustwidth}

L216-8 ``Tongue root displacement was thus calculated as the
displacement of the fitted spline along a selected vector.'' Based on
the figure provided (FIG. 1) I am not convinced that tongue- root
advancement is being captured. The probe angle is very narrow. The
tongue root does not appear to have been imaged, as there is no visible
hyoid shadow. I am not sure that a measure of displacement along a
radial fan axis shown would capture tongue-root advancement. Could you
provide information on the range of radial axes that were used as
measurement axes?

\begin{adjustwidth}{1cm}{} \textit{
The hyoid shadow is not visible because during UTI set up the angle is adjusted so that the leftmost edge of the ultrasound image corresponds to the edge of the shadow (this is necessary for the automatic tracker to work more consistently). It is true that an ideal measure of tongue root position should be taken along a vector which is perpendicular to the movement. Such vector could for example be chosen so that it is parallel to the genioglossus, which can be made visible on UTI images with the right settings. Unfortunately, the genioglossus was not always visible in the data. Moreover, since the data was collected without tongue root position in mind, I did not make sure to see whole of the actual tongue root all the time. Note though that the measured displacement must be related to tongue root displacement, since the fan-lines from which it was obtained were always more posterior than the fan-lines which corresponded to velar occlusion. This is the best I could do with the data at hand, although I understand it is not perfectly ideal. I'm happy to mention these issues in the paper if the reviewers think the paper might benefit from it.
} \end{adjustwidth}

L218-20 Can you specify the range of radial axes (i.e.~their numbers)
that were used as measurement axes across all speakers?

\begin{adjustwidth}{1cm}{} \textit{
Added.
} \end{adjustwidth}

L225 Remove the word ``real''.

\begin{adjustwidth}{1cm}{} \textit{
Removed.
} \end{adjustwidth}

L225 I am still confused about why there was no standard frame rate,
although I don't think it will make a difference to the results given
that the minimum frame rate is 43fps.

\begin{adjustwidth}{1cm}{} \textit{
I added a footnote "The frame rate is adjusted by the system depending on other settings, so there is no standard frame rate."
} \end{adjustwidth}

L255 ``\ldots{}scaled (z-scored) tongue root position.''

\begin{adjustwidth}{1cm}{} \textit{
Added.
} \end{adjustwidth}

L257-258 I might have missed this part of the data transformation. Fig.
2's caption says that ``Higher values indicate advancement.'' and
appears to show greater ATR values at closure onset where a voiced
consonant follows. I would have expected the opposite if advanced tongue
root is being measured. With increased ATR, the tongue-root section of
the tongue spline should intersect the radial fan axis closer to its
origin, obtaining a smaller value. Kirkham and Nance (2017) reversed the
sign of the z-scored tongue root distance measures, so that greater
distance scores were associated with greater ATR (p71). Was that also
carried out here? If it was, I think you should be explicit about that,
rather than just saying you followed Kirkham and Nance's method.

\begin{adjustwidth}{1cm}{} \textit{
I've expanded by saying that the sign has been flipped to facilitate interpretation (greater values = greater tongue root advancement).
} \end{adjustwidth}

282 ``A second linear mixed regression was fitted to tongue root
position ?at V1offset/C2onset?\ldots{}''

\begin{adjustwidth}{1cm}{} \textit{
Yes. I have now specified that the time point is V1 offset/C2 onset.
} \end{adjustwidth}

288-289 ``V1 duration and tongue root position ?at
V1offset/C2onset?\ldots{}''

\begin{adjustwidth}{1cm}{} \textit{
Yes. I have now specified that the time point is V1 offset/C2 onset.
} \end{adjustwidth}

378 Change to ``It is worth briefly discussing\ldots{}''

\begin{adjustwidth}{1cm}{} \textit{
Changed
} \end{adjustwidth}

384 phonemic symbol missing

\begin{adjustwidth}{1cm}{} \textit{
Fixed.
} \end{adjustwidth}

383-388 Perhaps this needs to be rephrased. It looks as if you are
implying that the acoustic difference between /e/ and /ɛ/ is entirely
due to differences in tongue-root position. Also phonemic contrast can
occur due to quite subtle acoustic cues, so phonemic difference does not
necessarily mean a huge difference in acoustics or underlying
articulation.

390-3 See comments l57-58 Rothenberg's estimated duration for
tongue-root advancement.

\begin{adjustwidth}{1cm}{} \textit{
Rephrased as "If a maximal ballistic forward movement of the tongue root takes between 70 and 90 ms as suggested by the informal investigation by Rothenberg (1967), we can calculate the maximum displacement plausible to be between 4.55 to 5.85 mm (0.065 mm times 70–90 ms).".
} \end{adjustwidth}

393-4 Does this refer to: ``Lacking further evidence we might assume it
is possible to produce a vertical elongation of the pharynx in plosive
production of about 0.5cm.'' (Rothenberg 1967: 98)?

\begin{adjustwidth}{1cm}{} \textit{
This refers to a few lines above that: "If we consider the lateral and posterior walls of the pharynx to be held fixed, the anterior-posterior dimension of the pharynx can be increased by a forward motion of the larynx and/or base of the tongue. Forward motions of the base of the tongue of the order of magnitude of 0.5 cm have been reported by PERKELL (1965). The main activation is most likely supplied by the most inferior-posterior fibers of the genioglossus muscle to the base of the tongue, the geniohyoid muscles, and the infrahyoid muscles. The forward motion of the anterior wall of the pharynx might well be more of a pivoting around an axis somewhere near the cricoid cartilage than a uniform forward motion; however, it is difficult to find empirical evidence bearing on this question."
} \end{adjustwidth}

FIG. 6 caption ``tornge''. Why are there two of each plot?

\begin{adjustwidth}{1cm}{} \textit{
Fixed. The plot duplication was a coding error and it's now fixed.
} \end{adjustwidth}

L431 ``shows'', or ``these plots show''

\begin{adjustwidth}{1cm}{} \textit{
Fixed.
} \end{adjustwidth}

L442 ``The mean difference in tongue root position (at the onset of?)
voiceless vs.~voiced stops\ldots{}''

\begin{adjustwidth}{1cm}{} \textit{
Yes, added "at the onset of".
} \end{adjustwidth}

L464 and 465 citations missing.

\begin{adjustwidth}{1cm}{} \textit{
I inserted here some of the citations from above for clarity.
} \end{adjustwidth}

Ll468 ``Similarly to what (was) previously found for English
(citation)\ldots{}''

\begin{adjustwidth}{1cm}{} \textit{
I inserted here some of the citations from above for clarity and I added "was".
} \end{adjustwidth}

L485 ``replicate'' -\textgreater{} ``are replicable''

\begin{adjustwidth}{1cm}{} \textit{
Changed.
} \end{adjustwidth}

\hypertarget{reviewer-2}{%
\section{Reviewer 2}\label{reviewer-2}}

The aim of this study is to show the relation between longer vowel
duration and tongue root advancement (which is related to the voicing of
the following stop). This manuscript contains some very interesting and
original/novel data, which was not studied before. Also, it has a
potential to be extended to other contexts as well as other languages.
However, there are some concerns about the interpretation of the
results. Here are my main concerns:

\begin{enumerate}
\def\labelenumi{\arabic{enumi}.}
\item
  The result the author argues to be the main point of the manuscript
  (how a longer vowel duration corresponds to greater tongue root
  advancement) is not very well explained. The author shows some results
  (Figure 5), but does not explain enough why tongue root advancement is
  related to the longer vowel duration. This study explains why tongue
  root should be more advanced at the vowel offset of voiced stops, but
  not necessarily why more tongue root advancement is found with the
  longer vowel duration. The only part that explains the relation
  between the vowel duration and tongue root advancement is line 333
  (p.21), but more explanations will be necessary---considering the main
  argument (and the title) of this paper is how longer vowel duration
  correlates with greater tongue root advancement)---or I wonder if the
  title purposefully used the word `displacement' instead of
  `advancement'.
\item
  Also, I'm concered about inter-speaker variation. In fact, out of 17
  speakers, 7 speakers did not show more tongue root advancement at
  closure onset of voiced stops (Figure 7 and Figure 8). Considering the
  difficulty of analyzing everyone's data collectively when doing an
  ultrasound study (due to the differences in each speaker's anatomy),
  the statistical model(s) the author reports in the manuscript can be
  questionable especially with the high variability found among
  speakers.
\item
  On p.~19, Figure 5, it shows that when the vowel is longer, the tongue
  root position at the onset of the vowel is more retracted. The reason
  for this seems very important, but it's not explained why that's the
  case other than saying "the trajectory curvature increases with vowel
  duration (line 302).
\item
  The author reports the data on two different languages, Italian and
  Polish, but it is less clear why these two languages are studied,
  especially since as the author argues, two languages did not show any
  difference. Also, since the data were collected in the UK, it's worth
  mentioning the length of stay in the UK and English proficiency of
  each speaker. It is well known that English voiced/voiceless stops
  behave differently from other languages, so English may have some
  effects on participants' native languages. Regarding the materials
  used for the data collection, I wonder if there was any effect of
  including the real words. Also, why /b, p/ was not included (at least
  for a comparison to other lingual consonants?)
\end{enumerate}

And, here are some parts I found unclear/need more explanations:

p.~3, 2nd paragraph: it's unclear what the purpose of the last sentence
is.

\begin{adjustwidth}{1cm}{} \textit{
This sentence clarifies the use of "active gesture" in the paper, to avoid confusion with the more general reading of active gesture as a gestures implemented for a specific purpose.
} \end{adjustwidth}

p.~13, line 235: Generalized additive mixed models (GAMMs). Including
more explanations on this model will be very helpful to understand how
this model's prediction works in Section III B and Fig. 3.

\begin{adjustwidth}{1cm}{} \textit{
I have included more details on what each term contributes with to the model fit in Section III.B.
} \end{adjustwidth}

p.~20, line 330: ``Said correlation exists independent of the voicing
status of the consonant following the vowel'' --\textgreater{} I
especially found it difficult to understand the paragraph including this
sentence and the following two paragraphs, and I believe these
paragraphs are main arguments of this manuscript. More specifically, why
do we see a greater tongue root advancement when the vowel is longer? Is
it simply because voiced stops show more tongue root advancement and a
longer vowel duration? More importantly, why do we see a greater
curvature in longer vowels? Why does a longer vowel show a more
retracted tongue root at its vowel onset?

\begin{adjustwidth}{1cm}{} \textit{
I have updated the relevant paragraphs in Section IV.A to reflect changes in Section III.D and expanded them to clarify the point being made.
} \end{adjustwidth}

p.~21, 3rd paragraph: this paragraph suddenly mentions speech rate and
its paradox, and it is not very clear what the argument here is.
Considering inter-speaker variation. I think at least with speech rate
and tongue root advancement of vowel onset mentioned here, it may be
worth looking at individual speaker's results instead of just use speech
rate as a factor in the statistical model.

\begin{adjustwidth}{1cm}{} \textit{
I have expanded and moved the relevant paragraphs to after the discussion of speakers' variation.
} \end{adjustwidth}

p.~23-24, Section IV B: I think this section can be either condensed or
deleted since estimates of tongue root displacement will be highly
variable depending on speaker. Since (1) Speaker variation was reported
in this study, (2) only two languages are studied, and (3) the author
mentions that ``the correlation between tongue root position and vowel
duration needs to be replicated by expanding the enquired contexts to
other types of consonants and vowels, and with other languages'' (p.~24,
line 410), I recommend to condense this section.

\begin{adjustwidth}{1cm}{} \textit{
It is generally desirable to discuss in some details the estimates of the sought effects, as recommended by recent literature on statistical power. It is true that the estimates will be highly variable depending on speaker, but since such estimates refer to a population of speakers it can still be useful, for example, for the determination of Bayesian priors in future work. The discussion of the estimates also links the results of this study back to theoretical argumentations from previous work, and it thus constitute a grounding point.
} \end{adjustwidth}

There are some parts where the tables and figures may be necessary
and/or more explanations on figures may be helpful:

First of all, in the results, tables with these acoustic data are
necessary: e.g.~mean V1 duration, mean speech rate, mean
closure-to-closure interval

p.~11, line 206: what is ``higher formant''? Either give a very specific
example, or provide a sample spectrogram.

\begin{adjustwidth}{1cm}{} \textit{
I have specified what is meant by "higher formants structure", taken from Machač 2009.
} \end{adjustwidth}

Figure 1: please indicate the front and back of the tongue

\begin{adjustwidth}{1cm}{} \textit{
Added indication of back/front.
} \end{adjustwidth}

Figure 2: How was the data of each speaker scaled? Also, it will be
helpful if this figure is in color.

\begin{adjustwidth}{1cm}{} \textit{
I've added that those are z-scores and I made the figure in colour.
} \end{adjustwidth}

Figure 4: Maybe using color and indicating voiced/voiceless stops may be
helpful.

\begin{adjustwidth}{1cm}{} \textit{
I've added colour by vowel in the plot. The model from which the regression line are obtained does not contain C2 voicing. This is because a separate model indicated that voicing nor its interaction with vowel duration is significant, so it was dropped from the final model. I've added a mention to this in Section III.C.
} \end{adjustwidth}

Figure 5: Why 145ms was used instead of 150ms? (while other V1 duration
values used are 50, 100, and 200). Also, why 145 ms and 200 ms starts
from more retracted tongue root position? It's not explained in the
text.

\begin{adjustwidth}{1cm}{} \textit{
The figure has now 150 ms rather than 145 ms. I have also increased the number of basis functions in the smooth for vowel duration, since I realised it was too low according to "gam.check()". This has partially change the starting values for the different vowel durations, so that now there is an increase in starting tongue root position with the exception of 200 ms which decreases. I do not know why this is the case. I've added in the text: "I have no explanation for why the advancement of the root seemingly increases with increasing vowel duration except when the duration goes from 150 to 200 ms."
} \end{adjustwidth}

The following other minor suggestions are noted:

p.~2, line 18: ``a lot of work has been done on each of these aspects
separately,\ldots{}'' but actual references are not mentioned; also,
line 24---references

\begin{adjustwidth}{1cm}{} \textit{
I have added here some of the references from later sections for clarity.
} \end{adjustwidth}

p.~3, line 39: ``Aerodynamic Voicing Constraint'': since this concept is
mentioned as a very important concept of this manuscript, more detailed
information about it will be helpful

\begin{adjustwidth}{1cm}{} \textit{
I am not sure what is asked here. The constraint simply states that there must be a positive trans-glottal air pressure differential, as discussed in lines 33-42.
} \end{adjustwidth}

p.~4, line 58: \ldots{}root ``advancement'' about\ldots{}.

\begin{adjustwidth}{1cm}{} \textit{
The sentence has been rephrased as suggested by one of the other reviewers and moved at the beginning of the paragraph.
} \end{adjustwidth}

p.~4, line 72: typo--`longer' --\textgreater{} `shorter'

\begin{adjustwidth}{1cm}{} \textit{
Fixed.
} \end{adjustwidth}

p.~4, line72-73: please indicate what language(s) each paper worked on

\begin{adjustwidth}{1cm}{} \textit{
Added.
} \end{adjustwidth}

p.~5, line 79: (Ahn, 2018)--\textgreater{} Ahn (2018)

\begin{adjustwidth}{1cm}{} \textit{
Fixed.
} \end{adjustwidth}

p.~5, line 98: here as well, please specify what language(s) each paper
worked on

\begin{adjustwidth}{1cm}{} \textit{
Added.
} \end{adjustwidth}

p.~6, line 106: ``relative ease of comparison'' --\textgreater{} not
sure what exactly it means

\begin{adjustwidth}{1cm}{} \textit{
I've explained this now "Moreover, the segmental phonologies of these languages facilitate the design of sufficiently comparable experimental material (see Coretta 2018 for a more thorough discussion)."
} \end{adjustwidth}

p.~6, line 111: ``the stressed vowels of disyllabic
words''--\textgreater{} indicate which vowel is stressed

\begin{adjustwidth}{1cm}{} \textit{
Stress is indicated with a diacritic although it is not properly rendered due to restrictions of the submission system, but it should be fixed in the final form. I have now specified that the stressed vowel is the first.
} \end{adjustwidth}

p.~8, line 147: missing `.' After Verbania (Italy)

\begin{adjustwidth}{1cm}{} \textit{
Fixed.
} \end{adjustwidth}

p.~10, line 182: ``high and front vowels usually produce less tongue
displacement from and to a stop consonant'' --\textgreater{}
reference(s)?

\begin{adjustwidth}{1cm}{} \textit{
This fact hasn't been systematically studied, but since the tongue is already quite high because of the production of high-front vowels, the displacement between the position of the tongue during high-front vowels and that of the tongue during stop closure is generally quite small and it makes it difficult to detect gestural landmarks from UTI displacement data.
} \end{adjustwidth}

p.~13, line 231: the calculated speech rate values---here, please
explain how the values were calculated

\begin{adjustwidth}{1cm}{} \textit{
I have included there a condensed version of the formula from above for clarity.
} \end{adjustwidth}

p.~13, line 235: Generalised additive mixed models---since it's first
time mentioning this model, show the abbreviation here (GAMMs?); line
238: GAMs--\textgreater{} GAMMs??

\begin{adjustwidth}{1cm}{} \textit{
Added abbreviation and fixed line 238.
} \end{adjustwidth}

P. 21, line 349 and 351: Two ``howevers'' are confusing. maybe change
the second however to ``nevertheless''?

\begin{adjustwidth}{1cm}{} \textit{
I have fixed the repetition of "however" by rephrasing.
} \end{adjustwidth}

P. 22, line 357: `previous work' \textgreater{} reference(s)

\begin{adjustwidth}{1cm}{} \textit{
I have added a cross-reference to the patterns described in the Introduction section.
} \end{adjustwidth}

P. 23, line 384: a +ATR vowel, a -ATR vowel,

\begin{adjustwidth}{1cm}{} \textit{
Fixed.
} \end{adjustwidth}

P. 23, line 384: also missing IPA symbol in the second / /

\begin{adjustwidth}{1cm}{} \textit{
Fixed.
} \end{adjustwidth}

P. 25, Figure 6: typo (torngue--\textgreater{} tongue)

\begin{adjustwidth}{1cm}{} \textit{
Fixed.
} \end{adjustwidth}

P. 26, line 447: 11 speakers--\textgreater{} 10 speakers?

\begin{adjustwidth}{1cm}{} \textit{
Fixed.
} \end{adjustwidth}

\hypertarget{reviewers-responses-to-questions}{%
\section{Reviewers' Responses to
Questions}\label{reviewers-responses-to-questions}}

Is the manuscript of good scientific quality, free from errors,
misconceptions or ambiguities; does it present original work; and does
it contain sufficient new results, new applications or new developments
of reasonable enough significance to warrant its publication in JASA?
Please indicate in your report (in detailed comments, below) any points
which are objectionable or which need attention.

Reviewer 1: There are some ambiguities and potential issues with
measurement that need to be addressed.

Reviewer 2: 1. The manuscript contains some very interesting and
original/new data. However, there are some concerns about the
interpretation of the results. The result the author argues to be the
main point of the manuscript (how a longer vowel duration corresponds to
greater tongue root advancement) is not very well explained throught the
manuscript. The author shows some results, but does not explain why that
is the case. Also, I'm concered about inter-speaker variation. In fact,
out of 17 speakers, 7 speakers did not show more tongue root advancement
at closure onset of voiced stops (Figure 7 and Figure 8). Considering
the difficulty of analyzing different speakers' data in one statistical
model when doing an ultrasound study (due to the differences in each
speaker's anatomy), the statistical model(s) the author reports in the
manuscript can be questionable especially with the high variability
found among speakers.

\begin{enumerate}
\def\labelenumi{\arabic{enumi}.}
\setcounter{enumi}{1}
\item
  On p.~19, Figure 5, it shows that when the vowel is longer, the tongue
  root position at the onset of the vowel is more retracted. The reason
  for this seems very important and worth some attention. However, the
  author does not explain why that's the case other than saying "the
  trajectory curvature increases with vowel duration (line 302).
\item
  There aren't adquate explanations on why tongue root advancement will
  be related to the longer vowel duration. This study explains why
  tongue root should be more advanced at the vowel offset of voiced
  stops, but not necessarily why more tongue root advancement is found
  with the longer vowel duration. The only part that explains the
  relation between the vowel duration and tongue root advancement is
  line 333 (p.21), but that part is not very well explained.
\item
  The author reports the data on two different languages, Italian and
  Polish, but it is less clear why these two languages are studied,
  especially since as the author argues, two languages did not show any
  difference.
\item
  Since the data were collected in the UK, it's worth mentioning the
  length of stay in the UK and English proficiency of each speaker. It
  is well known that English voiced/voiceless stops behave differently
  from other languages, so English may have some effects on
  participants' native languages.
\item
  Regarding the materials used for the data collection, including both
  nonce words and real words may not have been the best choice. I wonder
  if there was any effect of including the real words.
\item
  Also, this is not the crucial point, but some errors/typos in terms of
  writing sometimes makes it difficult to understand the writer's
  intention sometimes. (e.g.~p.~4, line 72, `longer' should be
  `shorter', which is very important). More detailed line-by-line
  comments will be provided to the author.
\end{enumerate}

\par

\noindent

\rule{\textwidth}{0.4pt}

Is JASA an appropriate journal in which to present this work? In this
regard, please consider carefully the commitment of JASA to publish work
that is within the scope of Acoustics. Does the content of the
manuscript, including terminology and the references cited, meet this
criterion?

Reviewer 1: Yes, this work is appropriate for publication in JASA.

Reviewer 2: This work includes both articulatory and acoustic data,
which I believe is suitable to be published in JASA. Also, it includes
some important phonetic work (both articulatory and acoustic) in the
field as relevant references.

\par

\noindent

\rule{\textwidth}{0.4pt}

Is the manuscript a clear, concise, reasonably self-contained
presentation of the material, giving adequate references to related
work? Is the English satisfactory? Please indicate needed changes in
your report.

Reviewer 1: The central argument of the manuscript could be made
clearer. The style of the manuscript can be overly complex.

Reviewer 2: 1. The manuscript contains interesting and new results, but
the writing itself is not the easiest to read. There were some sections
I had to read several times to understand. The statistical models in the
results will be benefited by more explanations, and the discussion
section is also not very clear. More detailed comments are below:

p.~3, 2nd paragraph: it's unclear what the purpose of the last sentence
is

p.~4, 2nd paragraph: the typo in line 72 (`longer' should be `shorter')
made the whole paragraph (and the following paragraphs) difficult to
understand

p.~13, line 235: Generalized additive mixed models (GAMMs). It is very
difficult to understand what this model does, and how this model's
prediction works in Section III B, which makes it difficult to
understand Fig. 3.

p.~20, line 330: ``Said correlation exists independent of the voicing
status of the consonant following the vowel'' --\textgreater{} this part
made the whole paragraph and the manuscript's argument a bit confusing.
I also found it difficult to understand the paragraph including this
sentence and the following paragraph.

p.~21, 3rd paragraph: this paragraph suddenly mentions speech rate and
its paradox, and it is not very convincing, especially considering
individual variation. I think at least with speech rate and tongue root
advancement of vowel onset, it may be worth looking at individual
speaker's results. It may not be best to use speech rate just as a
factor in the statistical model.

p.~23-24, Section IV B: I think this section can be either condensed or
deleted.

p.~25, Section IV C: I actually think individual differences are very
important. Since some speakers show very different patterns, I think
combining everyone's data at once and running the statistical model may
not be ideal. In fact, Figure 7 shows out of 17 speakers, 7 speakers
either didn't show the expected results or showed the opposite results.

\begin{enumerate}
\def\labelenumi{\arabic{enumi}.}
\setcounter{enumi}{1}
\tightlist
\item
  This manuscript gives some references to related work, but there are
  some parts where references are missing when necessary. More detailed
  comments are below:
\end{enumerate}

p.~2, line 18: ``a lot of work has been done on each of these aspects
separately,\ldots{}'' but actual references are not mentioned

p.~3, line 39: ``Aerodynamic Voicing Constraint'': since this concept is
mentioned as a very important concept of this manuscript, more detailed
information about it will be helpful

p.~4, line72-73: please indicate what language(s) each paper worked on

p.~5, line 98: here as well, please specify what language(s) each paper
worked on

p.~10, line 182: ``high and front vowels usually produce less tongue
displacement from and to a stop consonant'' --\textgreater{}
reference(s)?

\par

\noindent

\rule{\textwidth}{0.4pt}

Are the tables and figures clear and relevant, and are the captions
adequate? Are there either too many or too few? If any of the figures
are in color, is the color essential for conveying the scientific point?

Reviewer 1: Tables and figures are clear and relevant. Captions are
adequate. More figures could be included to illustrate measurement.

Reviewer 2: Some figures are not easy to understand, and there are some
parts where the tables and figures are necessary.

First of all, in the results, tables with these acoustic data are
necessary: mean V1 duration, mean speech rate, mean closure-to-closure
interval

p.~11, line 206: what is ``higher formant''? Either give a very specific
example, or provide a sample spectrogram

Figure 1: please indicate the front and back of the tongue

Figure 2: this figure is difficult to understand. How was the data of
each speaker scaled? Also, it will be helpful if this figure is in
color.

Figure 4: I don't understand this figure---maybe using color and
indicating voiced/voiceless stops may be helpful.

Figure 5: Why 145ms was used instead of 150ms? (while other V1 duration
values used are 50, 100, and 200) Also, why 145 ms and 200 ms starts
from the backer tongue root position? It's not explained in the text.

\par

\noindent

\rule{\textwidth}{0.4pt}

If Supplementary material was submitted, is it relevant to the
manuscript and should it be deposited in the Supplemental Depository for
reference to the manuscript?

Reviewer 1: NA

Reviewer 2: N/A

\par

\noindent

\rule{\textwidth}{0.4pt}

Does the paper make effective use of journal space, or are parts
unnecessary, unimportant, or subject to condensation? If so, which?

Reviewer 1: I believe the manuscript could be shortened and made more
concise, particularly the discussion section.

Reviewer 2: p.~23, Section IV B can be condensed/removed since estimates
of tongue root displacement will be highly variable depending on
speaker. Since (1) Speaker variation was reported in this study, (2)
only two languages are studied, and (3) the author mentions that ``the
correlation between tongue root position and vowel duration needs to be
replicated by expanding the enquired contexts to other types of
consonants and vowels, and with other languages'' (p.~24, line 410), I
recommend to condense this section.

\par

\noindent

\rule{\textwidth}{0.4pt}

Is the title appropriate and the abstract adequate for verbatim
reproduction in abstract journals? IMPORTANT: The lead paragraph should
advertise the main points of the article and must describe in terms
accessible to the general reader the context and significance or the
research problem studied and the importance of the results.

Reviewer 1: Yes, the title is appropriate. There are some parts of the
abstract that need to be edited to improve clarity.

Reviewer 2: I recommend the title to include that ``longer vowel
duration'' (only) refers to the ``preceding'' vowel duration (not
necessarily the following). I also think the word `advancement' will be
more suitable than `displacement'. On the other hand, I wonder if the
title purposefully used the word `displacement' instead of
`advancement'.

In the abstract, it will be more informative if the author includes at
least a few references (e.g.~the Aerodynamic Voicing
Constraints--\textgreater{} Ohala, 2011).

The results in the abstract can be more clearly written, especially the
last three sentences.

Vowel duration and tongue root position at vowel offset are positively
correlated. Longer vowel durations correspond to greater tongue root
advancement: I wonder if this should be the main point of this study, as
I pointed out throughout my review `the later closure onset of voiced
stops within a temporally stable interval' needs more explanation
because `shorter closure duration' of voicing stop is not mentioned in
the previous sentences.


\end{document}
