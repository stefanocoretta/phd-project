%%%%%%%%%%%%%%%%%%%%%%%%%%%%%%%%%%%%%%%%%%%%%%%%%%
%	JASA LaTeX Template File
%  For use in making articles using JASAnew.cls
% July 26, 2017
%%%%%%%%%%%%%%%%%%%%%%%%%%%%%%%%%%%%%%%%%%%%%%%%%%

%% Step 1:
%% Uncomment the style that you want to use:

%%%%%%% For Preprint
%% For manuscript, 12pt, one column style

%% Comment this out if you'd rather use another style:
\documentclass[]{JASAnew}

%%%%% Preprint Options %%%%%
%% The track changes option allows you to mark changes
%% and will produce a list of changes, their line number
%% and page number at the end of the article.
%\documentclass[preprint,trackchanges]{JASAnew}

%% authaffil option will make affil immediately
% follow author, otherwise authors are grouped, and affiliations
% are stacked underneath all the authors.
%\documentclass[preprint,authaffil]{JASAnew}

%% NumberedRefs is used for numbered bibliography and citations.
%% Default is Author-Year style.
%% \documentclass[preprint,NumberedRefs]{JASAnew}

%%%%%%% For Reprint
%% For appearance of finished article; 2 columns, 10 pt fonts

% \documentclass[reprint]{JASAnew}

%%%%% Reprint Options %%%%%

%% For testing to see if author has exceeded page length request, use 12pt option
%\documentclass[reprint,12pt]{JASAnew}

% authaffil option will make affil immediately
% follow author, otherwise authors are grouped, and affiliations
% are stacked underneath all the authors.
%\documentclass[reprint,authaffil]{JASAnew}

%% NumberedRefs is used for numbered bibliography and citations.
%% Default is Author-Year style.
% \documentclass[reprint,NumberedRefs]{JASAnew}

%% TurnOnLineNumbers
%% Make lines be numbered in reprint style:
% \documentclass[reprint,TurnOnLineNumbers]{JASAnew}

\usepackage{natbib}


\usepackage{cleveref}
\usepackage{ctable}

\begin{document}
%% the square bracket argument will send term to running head in
%% preprint, or running foot in reprint style.

\title[A subtitle goes on another line]{This is a title and this is too}

% ie
%\title[JASA/Sample JASA Article]{Sample JASA Article}

%% repeat as needed

\author{Stefano Coretta}
% ie
%\affiliation{Department1,  University1, City, State ZipCode, Country}
\affiliation{The University of Manchester}
%% for corresponding author
\email{stefano.coretta@manchester.ac.uk}
%% for additional information
\thanks{other info}

% ie
% \author{Author Four}
% \email{author.four@university.edu}
% \thanks{Also at Another University, City, State ZipCode, Country.}

%% For preprint only,
%  optional, if you want want this message to appear in upper left corner of title page
% \preprint{}

%ie
%\preprint{Author, JASA}

% optional, if desired:
%\date{\today}

\begin{abstract}
% Put your abstract here. Abstracts are limited to 200 words for
% regular articles and 100 words for Letters to the Editor. Please no
% personal pronouns, also please do not use the words ``new'' and/or
% ``novel'' in the abstract. An article usually includes an abstract, a
% concise summary of the work covered at length in the main body of the
% article.
Put your abstract here.
\end{abstract}

%% pacs numbers not used

\maketitle

%  End of title page for Preprint option --------------------------------- %

%% See preprint.tex/.pdf or reprint.tex/.pdf for many examples


%  Body of the article
\hypertarget{introduction}{%
\section{Introduction}\label{introduction}}

Almost 100 years of research have repeatedly shown that consonantal
voicing has an effect on preceding vowel duration: vowels followed by
voiced obstruents are longer than when followed by voiceless ones
\citep{heffner1937, house1953, belasco1953, peterson1960, halle1967, chen1970, klatt1973, lisker1974, raphael1975, javkin1976, maddieson1976, farnetani1986, kluender1988, laeufer1992, fowler1992, hussein1994, esposito2002, lampp2004, warren2005, durvasula2012}.
Evidence for such so called `voicing effect' has been found in a variety
of languages, including (but not limited to) English, German, Hindi,
Russian, Italian, Arabic, and Korean \citep[see][ for a more
comprehensive, but still not exhaustive list]{maddieson1976}. Despite of
the plethora of evidence in support of the \emph{existence} of the
voicing effect, still after 100 years agreement hasn't been reached
regarding the source of this effect.

Several proposal have been put forward as to where to look for the
possible cause of the voicing effect \citep[see][ and
\citet{soskuthy2013} for an overview]{maddieson1976}. Most of the
proposed accounts place the source of the voicing effect in properties
of speech
production.\footnote{Two accounts that point to perceptual features are \citet{javkin1976} and \citet{kluender1988}. To the best of my knowledge, \citet{javkin1976}'s proposal remains empirically untested, while see \citet{fowler1992} for arguments against \citet{kluender1988}.}
One of these production accounts, which will be the focus of this study,
relates the voicing effect to some constant property of speech that is
held constant across contexts while the local property of voiceless
vs.~voiced obstruents varies, thus creating a trade-off solution within
the constant property. \citet{lindblom1967}, \citet{slis1969a},
\citet{slis1969}, and \citet{lehiste1970} (among others) argue that the
relevant invariant property of speech is a constant durational interval
within which segments of different duration results in different
duration of other segments. Both the syllable/VC sequence (Lindblom) and
the word (Lehiste, Slis) has been proposed as the fixed interval. The
closure of voiced stops is shorter than that of voiceless stops. It
follows that vowels followed by shorter closures (like in the case of
voiced stops) are longer than vowels followed by longer closures (like
in the case of voiceless stops).

However, \citet{chen1970} and \citet{maddieson1976} criticise the
compensatory temporal adjustment account on empirical grounds.
\citet{chen1970} shows that the duration of the syllable is affected by
consonant voicing \citep[compatible with findings in][]{jacewicz2009},
contrary to Lindblom's expectations. \citet{maddieson1976} reject any
compensatory account based on data from a phenomenon parallel to the
voicing effect, the aspiration effect, by which vowel tend to be longer
when followed by aspirated stops than when followed by non-aspirated
stops. They measured consonant duration and they found no compensatory
pattern in relation to vowel duration: the consonant /t/, which has the
shortest duration, is preceded by the shortest vowel, and the vowels
before /d/ and /tʰ/ have the same duration although the durations of the
two consonant are different.

They show that word duration is not affected by voicing of C2 but they
don't discuss the internal structure of the word. I will show that the
Release to Release is invariant and that this is compatible with a
gestural timing in which the C2 is right-edge aligned with C1/V. I will
also offer an interpretation of \citet{maddieson1976} that is compatible
with a compensatory temporal adjustment account.

\hypertarget{method}{%
\section{Method}\label{method}}

\hypertarget{participants}{%
\subsection{Participants}\label{participants}}

Seventeen subjects in total participated to this exploratory study.
Eleven participants were native speakers of Italian (5 female, 6 male),
while six were native speakers of Polish (3 female, 3 male). The Italian
speakers were from the North and Centre of Italy (8 speakers from
Northern Italy, 3 from Central Italy). The Polish group had 2 speakers
from Poznań and 4 speakers from Eastern Poland. For more information on
the speakers, see \Cref{a:socioling}. Ethical clearance was obtained for
this study from the University of Manchester (REF 2016-0099-76). The
participants signed a written consent and received a monetary
compensation.

\hypertarget{equipment}{%
\subsection{Equipment}\label{equipment}}

The acquisition of the audio signal was achieved with the software
Articulate Assistant Advanced™ (AAA, v2.17.2) running on a
Hawlett-Packard ProBook 6750b laptop with Microsoft Windows 7, with a
sample rate of 22050 MHz (16-bit) in a proprietary format. A FocusRight
Scarlett Solo pre-amplifier and a Movo LV4-O2 Lavalier microphone were
used for audio recording.

\hypertarget{materials}{%
\subsection{Materials}\label{materials}}

The target stimuli were disyllabic words with
C\textsubscript{1}V\textsubscript{1}C\textsubscript{2}V\textsubscript{2}
structure, where C\textsubscript{1} = /p/, V\textsubscript{1} = /a, o,
u/, C\textsubscript{2} = /t, d, k, g/, and V\textsubscript{2} =
V\textsubscript{1} (e.g. /pata/, /pada/, /poto/, etc.). The lexical
stress of the target words was placed by speakers of both Italian and
Polish on V\textsubscript{1}, as intended. The make-up of the target
words was constrained by the design of the experiment, which included
ultrasound tongue imaging (UTI). Front vowels are difficult to image
with UTI, since their articulation involves tongue positions which are
particularly far from the ultrasonic probe, hence reducing the
visibility of the tongue contour. For this reason, only central and back
vowels were included. Since one of the variables of interest in the
exploratory study was the closing gesture of C\textsubscript{2}, only
lingual consonants were used. A labial stop was chosen as the first
consonant to reduce possible coarticulation with the following vowel
(although see \citealt{vazquez-alvarez2007}). The target words were
embedded in a frame sentence, \emph{Dico X lentamente} `I say X slowly'
in Italian \citep[following][]{hajek2008}, and \emph{Mówię X teraz} `I
say X now' in Polish. These sentences were chosen in order to keep the
placement of stress and emphasis similar across languages, so to ensure
comparability of results.

\hypertarget{procedure}{%
\subsection{Procedure}\label{procedure}}

The participant was asked to read the sentences with the target words
which were sequentially presented on the computer screen. The order of
the sentence stimuli was randomised for each participant. Each
participant read the list of randomised sentence stimuli six times. Due
to software constraints, the order of the list was kept the same across
the six repetitions within each participant. Each speaker read a total
of 72 sentences, with a grand total of 576 tokens (288 per language).
The reading task lasted between 15 and 20 minutes, with optional short
breaks between one repetition and the other.

\hypertarget{data-processing-and-measurements}{%
\subsection{Data processing and
measurements}\label{data-processing-and-measurements}}

\ctable[caption = List of measurements as extracted from acoustics.,
label = t:dur-measures,
width=\textwidth,
star
]{ll>{\raggedright}p{9cm}}{}{
\FL
\textbf{landmark}               &                  & \textbf{criteria}                                                                                    \ML
vowel onset           & (V1 onset)         & appearance of higher formants in the spectrogram following the burst of /p/ (C1)            \NN
vowel offset          & (V1 offset)        & disappearance of the higher formants in the spectrogram preceding the target consonant (C2) \NN
consonant onset       & (C2 onset)         & corresponds to V1 offset                                                                    \NN
closure onset         & (C2 closure onset) & corresponds to V1 offset                                                                    \NN
consonant offset      & (C2 offset)        & appearance of higher formants of the vowel following C2 (V2); corresponds to V2 onset                                \NN
consonant release & (C1/C2 release)         & automatic detection + manual correction \citep{ananthapadmanabha2014}                                           \LL
}

The audio recordings were exported from AAA in \texttt{.wav} format for
further processing. A forced aligned transcription was accomplished
through the SPeech Phonetisation Alignment and Syllabification software
(SPPAS) \citep{bigi2015}. The outcome of the automatic annotation was
manually corrected when necessary, according to the criteria in
\Cref{t:dur-measures}. The releases of C1 and C2 were detected
automatically by means of a Praat scripting implementation of the
algorithm described in \citet{ananthapadmanabha2014}. The durations in
milliseconds of the following intervals were extracted from the
annotated acoustic landmarks with Praat scripting: sentence duration,
word duration, vowel duration (V1 onset to V1 offset), consonant closure
duration (V1 offset to C2 burst), and Release-to-Release duration (RR
duration, C1 release to C2 release). Syllable rate (syllables per
second) was used as a proxy to speech rate \citep{plug2018} for duration
normalisation, and was calculated as the number of syllables divided by
the duration of the sentence (8 syllables in Italian, 6 in Polish).

\hypertarget{statistical-analysis}{%
\subsection{Statistical analysis}\label{statistical-analysis}}

The durational measurements were analysed with linear mixed-effects
models using \texttt{lme4} v1.1-17 in R v3.5.0
\citep{r-core-team2018, bates2015}. \emph{P}-values for the individual
terms were obtained with \texttt{lmerTest} v3.0-1, which uses the
Satterthwaite's approximation to degrees of freedom
\citep{kuznetsova2017}. The estimates of the relevant effects are then
calculated by refitting the models including only the significant terms
\citep[step-down approach,][]{diggle2002, zuur2009}. Bayes factors were
used to specifically test the null hypotheses that word and RR duration
are not affected by C2 voicing (i.e., the effect of C2 voicing on
duration is \texttt{0}). The BIC approximation was used to calculate the
Bayes factors \citep{wagenmakers2007}. The approximation is performed
according to the equation in \ref{eq:bayes}
\citep[p.~796]{wagenmakers2007}.

\begin{equation}
\label{eq:bayes}
BF_{01} \approx exp(\Delta{}BIC_{10}/2)
\end{equation}

where \(\Delta{}BIC_{10} = BIC_1 - BIC_0\). Values of \(BF_{01} > 1\)
indicate a preference of H\textsubscript{0} over H\textsubscript{1}.

\hypertarget{results}{%
\section{Results}\label{results}}

Only the most relevant terms will be presented. For the others see
tables and appendixes.

\hypertarget{vowel-duration}{%
\subsection{Vowel duration}\label{vowel-duration}}

A linear mixed-effects model was fitted with the following terms: vowel
duration as the outcome variable; fixed effects for C2 voicing, C2 place
of articulation, vowel identity, language, and speech rate (as syllables
per second); by-speaker and by-word random intercept with by-speaker
random slopes for C2 voicing. All logical interactions were included.
According to t-tests with Satterthwaite's approximation to degrees of
freedom, the following terms and interactions were significant: C2
voicing, C2 place of articulation, vowel identity, language, speech
rate, the interaction between vowel and C2 voicing, and vowel and C2
place. Vowels are 14 ms longer when followed by a voiced stop, although
vowel identity enters in an interaction with C2 voicing. The effect of
voicing seems to be greater for /a/ and smaller for /u/, with /o/ having
an intermediate effect. Polish has on average shorter vowels than
Italian (-25.5 ms), although the effect of voicing is estimated to be
the same in both languages. /u/ is 13.5 ms shorter when followed by a
velar stop. The effect of C2 place on /a/ and /o/ is smaller. Speech
rate has a negative effect on vowel duration, such that faster rates
correlate with shorter vowel durations.

\hypertarget{consonant-closure-duration}{%
\subsection{Consonant closure
duration}\label{consonant-closure-duration}}

The same maximally specified model as with vowel duration has been
fitted to consonant closure durations as the outcome variable. The
following terms and interactions were significant: C2 voicing, C2 place
of articulation, vowel identity, language, and interactions between
language and C2 place, language and vowel identity, C2 voicing and
place, C2 voicing and vowel, and a three-way interaction between C2
voicing, place and vowel identity. Stop closure is 15 ms shorter (se = )
if the stop is voiced. Vowel identity has an effect on closure duration
in voiced stops, but not in voiceless stops, and more so in voiced velar
than in voiced coronal stops: closure after /a/ is the shortest, while
after /u/ is the longest, with closure after /o/ in the middle.

\hypertarget{release-to-release-interval-duration}{%
\subsection{Release to Release interval
duration}\label{release-to-release-interval-duration}}

The duration of the interval between the release of C1 and the release
of C2 is not affected by C2 voicing.

\hypertarget{discussion}{%
\section{Discussion}\label{discussion}}

A major drawback of the analysis in \citet{maddieson1976} is that the
consonant duration in fact was measured from the closure of the relevant
consonant to the release of the following consonant, due to difficulties
in detecting the release of the consonant of interest (e.g., in \emph{ab
sāth kaho}, the duration of /tʰ/ in \emph{sāth} was calculated as the
interval between the closure of /tʰ/ and the release of /k/). This
measure includes the burst and (eventual) aspiration of the consonant.
\citet{slis1969a}, however, states that the inverse correlation between
vowel duration and the following consonant raises when consonant
\emph{closure} duration is taken into account, and not entire
\emph{consonant} duration. If the correlation exists between vowel and
closure duration, the inclusion of burst/aspiration duration clearly
alters this relationship.

\begin{acknowledgments}
Thanks to...
\end{acknowledgments}

\appendix

\section{Socio-linguistic information of participants}
\label{a:socioling}

\begin{table}[]
\begin{tabular}{llllllll}
Participant ID & Age & Sex    & Native language & Other languages & City of birth & Spent most time & More than 6 mo abroad \\
it01 & 29 & Male   & Italian & English, Spanish                          & Verbania            & Verbania        & Yes \\
it02 & 26 & Male   & Italian & Friulian, English, Ladin-Venetan          & Udine               & Tricesimo (UD)  & Yes \\
it03 & 28 & Female & Italian & English, German                           & Verbania            & Verbania        & No  \\
it04 & 54 & Female & Italian & Calabrese                                 & Verbania            & Verbania        & No  \\
it05 & 28 & Female & Italian & Engligh                                   & Verbania            & Verbania        & No  \\
it07 & 29 & Male   & Italian & English                                   & Tradate             & Cairate         & No  \\
it09 & 35 & Female & Italian & English                                   & Vignola (MO), Italy & Vignola         & Yes \\
it11 & 24 & Male   & Italian & english                                   & Monza               & Monza           & Yes \\
it13 & 20 & Female & Italian & English, French, Arabic, Farsi            & Ancona              & Chiaravalle     & Yes \\
it14 & 32 & Male & Italian & English, Spanish & Frosinone & Frosinone & Yes \\
pl02 & 32 & Female & Polish  & English, Norwegian, French, German, Dutch & Koło                & Poznań          & Yes \\
pl03 & 26 & Male   & Polish  & Russian, English, French, German          & Nowa Sol            & Poznań          & Yes \\
pl04 & 34 & Female & Polish  & Spanish, English, French                  & Warsaw              & Warsaw          & No  \\
pl05 & 42 & Male   & Polish  & English, French                           & Przasnysz           & Warsaw        & No  \\
pl06 & 33 & Male   & Polish  & English                                   & Zgierz              & Zgierz          & Yes \\
pl07 & 32 & Female & Polish  & English, Russian                          & Bielsk Podlaski     & Bielsk Podlaski & Yes
\end{tabular}
\end{table}

%% before appendix (optional) and bibliography:
% \begin{acknowledgments}
%This research was supported by  ...
% \end{acknowledgments}

% -------------------------------------------------------------------------------------------------------------------
%   Appendix  (optional)

%\appendix
%\section{Appendix title}

%If only one appendix, please use
%\appendix*
%\section{Appendix title}


%=======================================================
%IMPORTANT

%Use \bibliography{<name of your .bib file>}+
%to make your bibliography with BibTeX.

%Once you have used BibTeX you
%should open the resulting .bbl file and cut and paste the entire contents
%into the end of your article.
%=======================================================

\bibliography{linguistics.bib}


\end{document}
