%%%%%%%%%%%%%%%%%%%%%%%%%%%%%%%%%%%%%%%%%%%%%%%%%%
%	JASA LaTeX Template File
%  For use in making articles using JASAnew.cls
% July 26, 2017
%%%%%%%%%%%%%%%%%%%%%%%%%%%%%%%%%%%%%%%%%%%%%%%%%%

%% Step 1:
%% Uncomment the style that you want to use:

%%%%%%% For Preprint
%% For manuscript, 12pt, one column style

%% Comment this out if you'd rather use another style:
\documentclass[preprint]{JASAnew}

%%%%% Preprint Options %%%%%
%% The track changes option allows you to mark changes
%% and will produce a list of changes, their line number
%% and page number at the end of the article.
%\documentclass[preprint,trackchanges]{JASAnew}

%% authaffil option will make affil immediately
% follow author, otherwise authors are grouped, and affiliations
% are stacked underneath all the authors.
%\documentclass[preprint,authaffil]{JASAnew}

%% NumberedRefs is used for numbered bibliography and citations.
%% Default is Author-Year style.
%% \documentclass[preprint,NumberedRefs]{JASAnew}

%%%%%%% For Reprint
%% For appearance of finished article; 2 columns, 10 pt fonts

% \documentclass[reprint]{JASAnew}

%%%%% Reprint Options %%%%%

%% For testing to see if author has exceeded page length request, use 12pt option
%\documentclass[reprint,12pt]{JASAnew}

% authaffil option will make affil immediately
% follow author, otherwise authors are grouped, and affiliations
% are stacked underneath all the authors.
%\documentclass[reprint,authaffil]{JASAnew}

%% NumberedRefs is used for numbered bibliography and citations.
%% Default is Author-Year style.
% \documentclass[reprint,NumberedRefs]{JASAnew}

%% TurnOnLineNumbers
%% Make lines be numbered in reprint style:
% \documentclass[reprint,TurnOnLineNumbers]{JASAnew}

\usepackage{fontspec}

    \setmainfont[]{FreeSerif}

\usepackage{natbib}


\usepackage{cleveref}
\usepackage{ctable}
\usepackage[caption=false]{subfig}

\begin{document}
%% the square bracket argument will send term to running head in
%% preprint, or running foot in reprint style.

\title[A subtitle goes on another line]{This is a title and this is too}

% ie
%\title[JASA/Sample JASA Article]{Sample JASA Article}

%% repeat as needed

\author{Stefano Coretta}
% ie
%\affiliation{Department1,  University1, City, State ZipCode, Country}
\affiliation{The University of Manchester}
%% for corresponding author
\email{stefano.coretta@manchester.ac.uk}
%% for additional information
\thanks{other info}

% ie
% \author{Author Four}
% \email{author.four@university.edu}
% \thanks{Also at Another University, City, State ZipCode, Country.}

%% For preprint only,
%  optional, if you want want this message to appear in upper left corner of title page
% \preprint{}

%ie
%\preprint{Author, JASA}

% optional, if desired:
%\date{\today}

\begin{abstract}
% Put your abstract here. Abstracts are limited to 200 words for
% regular articles and 100 words for Letters to the Editor. Please no
% personal pronouns, also please do not use the words ``new'' and/or
% ``novel'' in the abstract. An article usually includes an abstract, a
% concise summary of the work covered at length in the main body of the
% article.
Put your abstract here.
\end{abstract}

%% pacs numbers not used

\maketitle

%  End of title page for Preprint option --------------------------------- %

%% See preprint.tex/.pdf or reprint.tex/.pdf for many examples


%  Body of the article
\hypertarget{introduction}{%
\section{Introduction}\label{introduction}}

Almost 100 years of research have repeatedly shown that consonantal
voicing has an effect on preceding vowel duration: vowels followed by
voiced obstruents are longer than when followed by voiceless ones
\citep{heffner1937, house1953, belasco1953, peterson1960, halle1967, chen1970, klatt1973, lisker1974, raphael1975, javkin1976, maddieson1976, farnetani1986, kluender1988, laeufer1992, fowler1992, hussein1994, esposito2002, lampp2004, warren2005, durvasula2012}.
Evidence for such so called `voicing effect' has been found in a variety
of languages, including (but not limited to) English, German, Hindi,
Russian, Italian, Arabic, and Korean \citep[see][ for a more
comprehensive, but still not exhaustive list]{maddieson1976}. Despite of
the plethora of evidence in support of the \emph{existence} of the
voicing effect, still after 100 years agreement hasn't been reached
regarding the source of this effect.

Several proposal have been put forward as to where to look for the
possible cause of the voicing effect \citep[see][ and
\citet{soskuthy2013} for an overview]{maddieson1976}. Most of the
proposed accounts place the source of the voicing effect in properties
of speech
production.\footnote{Two accounts that point to perceptual features are \citet{javkin1976} and \citet{kluender1988}. To the best of my knowledge, \citet{javkin1976}'s proposal remains empirically untested, while see \citet{fowler1992} for arguments against \citet{kluender1988}.}
One of these production accounts, which will be the focus of this study,
relates the voicing effect to some constant property of speech that is
held constant across contexts while the local property of voiceless
vs.~voiced obstruents varies, thus creating a trade-off solution within
the constant property. \citet{lindblom1967}, \citet{slis1969}, and
\citet{lehiste1970} (among others) argue that the relevant invariant
property of speech is a constant durational interval within which
segments of different duration results in different duration of other
segments. Both the syllable/VC sequence \citep{lindblom1967} and the
word \citep{slis1969a, slis1969, lehiste1970, lehiste1970a} has been
proposed as the fixed interval. The closure of voiced stops is shorter
than that of voiceless stops. It follows that vowels followed by shorter
closures (like in the case of voiced stops) are longer than vowels
followed by longer closures (like in the case of voiceless stops).
However, the compensatory temporal adjustment account has been
criticised in several occasions.

The proposal of the syllable or the word as the targets for compensation
and encounter difficulties when confronted with empirical evidence and
when scrutinised by logic. First, Lindblom's \citeyearpar{lindblom1967}
argument that the syllable is the interval within which compensation
happens is not supported by the findings in \citet{chen1970} and
\citet{jacewicz2009}. \citet{chen1970} rejects a syllable-based
compensatory account on the light of the fact that the duration of the
syllable is affected by consonant voicing. More recently,
\citet{jacewicz2009} further shows that the duration of monosyllabic
words in American English does not change dependent on the voicing of
the coda consonant. Second, although \citet{slis1969} results confirm
that the word does not change in duration whether the stop following the
stressed vowel is voiceless or voiced, it does not follow from this that
compensation should necessarily happen on the stressed vowel. Indeed, it
could also happen in the following unstressed vowel.

\citet{maddieson1976} reject any compensatory account for the voicing
effect based on data from Hindi on the so called `aspiration effect', by
which vowels tend to be longer when followed by aspirated stops than
when followed by non-aspirated stops. Vowels before voiceless
unaspirated stops are the shortest, followed by vowels before voiced
unaspirated and voiceless aspirated stops, which have similar duration
between each other, followed by vowels before voiced aspirated stops,
which are the longest. \citet{maddieson1976} find no compensatory
pattern between vowel and consonant duration: the consonant /t/, which
has the shortest duration, is preceded by the shortest vowel, and vowels
before /d/ and /tʰ/ have the same duration although the durations of the
two consonant are different.

However, an reevaluation of the way consonant duration is measured in
\citet{maddieson1976} might actually turn the situation in favour of a
compensatory account. Consonant duration is in fact measured from the
closure of the relevant consonant to the release of the following
consonant, due to difficulties in detecting the release of the consonant
of interest (e.g., in \emph{ab sāth kaho}, the duration of /tʰ/ in
\emph{sāth} was calculated as the interval between the closure of /tʰ/
and the release of /k/). This measure includes the burst and (eventual)
aspiration of the consonant. \citet{slis1969a}, however, states that the
inverse correlation between vowel duration and the following consonant
raises when consonant \emph{closure} duration is taken into account, and
not the entire \emph{consonant} duration. If the correlation exists
between vowel and closure duration, the inclusion of burst/aspiration
duration clearly alters this relationship. Indeed, the data in
\citet{durvasula2012} show that closure duration, appropriately
measured, decreases from voiceless unaspirated \textgreater{} voiced
\textgreater{} voiceless aspirated \textgreater{} voiced aspirated,
which closely resembles the order of increasing vowel duration in
\citet{maddieson1976}.\footnote{\citet{durvasula2012} does not find a negative correlation between vowel duration and consonant closure duration, but rather a (small) positive effect: vowel duration increases with closure duration when including an voicing and aspiration as fixed effects. However, it is likely that this result is a consequence of not controlling for speech rate, so it will not be discussed here.}

To summarise, a compensatory temporal adjustment account of the voicing
effect is possible after a careful review of the critiques advanced by
\citet{chen1970} and \citet{maddieson1976}, although issues about the
actual implementation of the compensation still remain. In conclusion,
for the compensatory account to gain plausibility, an invariant interval
within which compensation is implemented needs to be better defined, on
the light of empirical data.

\hypertarget{the-present-study}{%
\subsection{The present study}\label{the-present-study}}

This paper reports on results from a broader exploratory study that
investigates the relationship between vowel duration and consonant
voicing from an articulatory perspective. Synchronised recordings of
audio, ultrasound tongue imaging, and electroglottography were carried
out to enable a data-driven approach to the analysis of features related
to the voicing effect in the context of disyllabic (CVCV) words in
Italian and Polish. The design of the study has been constrained by the
use of these articulatory techniques (see \Cref{s:method}). Moreover,
given the exploratory nature of the study, the experimental design was
not implemented to directly test the compensatory account. Here, only
the results from acoustic will be discussed.

Italian and Polish reportedly differ in the magnitude of the voicing
effect. Italian has been unanimously reported as a voicing effect
language \citep{caldognetto1979, farnetani1986, esposito2002}. The mean
difference in vowel duration when followed by voiceless vs.~voiced
consonants ranges between 22 and 24 ms \citep[with longer vowels
followed by voiced
consonants,][]{farnetani1986, esposito2002}.\footnote{These estimates should be taken as a gross approximation.
There are several issues: number of speakers, different contexts, statistical modelling.}
On the other hand, the results regarding the presence and magnitude of
the effect in Polished are mixed. While \citet{keating1984} reports no
effect of voicing on vowel duration in data from 24 speakers,
\citet{nowak2006} finds that vowels followed by voiced stops are 4.5 ms
longer in the 4 speakers recorded. Moreover, \citet{malisz2008} argue
based on data from 40 speakers that the magnitude of the voicing effect
in Polish is highly idiosyncratic, and claim their results to be
inconclusive on this matter. The difference in presence or magnitude of
the voicing effect in Italian vs.~Polish should enable us to find an
underlying property that differs in the two languages and that might
indicate a possible source for the voicing effect.

The acoustic data from the exploratory study reported here reveal that
the duration of the interval between the releases of the two consonants
in CVCV words (the Release to Release interval) is not affected by the
voicing of the second consonant. This finding is compatible with a
compensatory temporal adjustment account by which the timing of the stop
closure onset within said interval determines the respective durations
of the vowel and the stop closure. I further propose that the invariant
duration of the Release to Release interval is congruent with current
views on gestural timing \citep{goldstein2014} and I discuss the
insights it provides in relation to our understanding of gestural
organisation in speech.

\hypertarget{method}{%
\section{Method}\label{method}}

\label{s:method}

\hypertarget{participants}{%
\subsection{Participants}\label{participants}}

Seventeen subjects in total participated to this exploratory study.
Eleven participants were native speakers of Italian (5 female, 6 male),
while six were native speakers of Polish (3 female, 3 male). The Italian
speakers were from the North and Centre of Italy (8 speakers from
Northern Italy, 3 from Central Italy). The Polish group had 2 speakers
from Poznań and 4 speakers from Eastern Poland. For more information on
the speakers, see \Cref{a:socioling}. Ethical clearance was obtained for
this study from the University of Manchester (REF 2016-0099-76). The
participants signed a written consent and received a monetary
compensation.

\hypertarget{equipment}{%
\subsection{Equipment}\label{equipment}}

The acquisition of the audio signal was achieved with the software
Articulate Assistant Advanced™ (AAA, v2.17.2) running on a
Hawlett-Packard ProBook 6750b laptop with Microsoft Windows 7, with a
sample rate of 22050 MHz (16-bit) in a proprietary format. A FocusRight
Scarlett Solo pre-amplifier and a Movo LV4-O2 Lavalier microphone were
used for audio recording.

\hypertarget{materials}{%
\subsection{Materials}\label{materials}}

\label{s:materials}

The target stimuli were disyllabic words with
C\textsubscript{1}V\textsubscript{1}C\textsubscript{2}V\textsubscript{2}
structure, where C\textsubscript{1} = /p/, V\textsubscript{1} = /a, o,
u/, C\textsubscript{2} = /t, d, k, g/, and V\textsubscript{2} =
V\textsubscript{1} (e.g. /pata/, /pada/, /poto/, etc.). The lexical
stress of the target words was placed by speakers of both Italian and
Polish on V\textsubscript{1}, as intended. The make-up of the target
words was constrained by the design of the experiment, which included
ultrasound tongue imaging (UTI). Front vowels are difficult to image
with UTI, since their articulation involves tongue positions which are
particularly far from the ultrasonic probe, hence reducing the
visibility of the tongue contour. For this reason, only central and back
vowels were included. Since one of the variables of interest in the
exploratory study was the closing gesture of C\textsubscript{2}, only
lingual consonants were used. A labial stop was chosen as the first
consonant to reduce possible coarticulation with the following vowel
(although see \citealt{vazquez-alvarez2007}). The target words were
embedded in a frame sentence, \emph{Dico X lentamente} `I say X slowly'
in Italian \citep[following][]{hajek2008}, and \emph{Mówię X teraz} `I
say X now' in Polish. These sentences were chosen in order to keep the
placement of stress and emphasis similar across languages, so to ensure
comparability of results.

\hypertarget{procedure}{%
\subsection{Procedure}\label{procedure}}

The participant was asked to read the sentences with the target words
which were sequentially presented on the computer screen. The order of
the sentence stimuli was randomised for each participant. Each
participant read the list of randomised sentence stimuli 6 times. Due to
software constraints, the order of the list was kept the same across the
six repetitions within each participant. Each speaker read a total of 12
sentences for 6 times (with the exceptions of IT02, who repeated the 12
sentences 5 times, and IT07, with whom words containing /u/ were not
recorded due to technical difficulties relating to the ultrasound data
collection).\footnote{IT01 and IT02 (the first two participants of this study) read also sentences with words starting with /b/, which were later excluded from the experimental design. The data from /b/-initial words are not included in the analysis reported in this paper.}
with a grand total of 1224 tokens (792 from Italian, 432 from Polish).
The reading task lasted between 15 and 20 minutes, with optional short
breaks between one repetition and the other.

\hypertarget{data-processing-and-measurements}{%
\subsection{Data processing and
measurements}\label{data-processing-and-measurements}}

\ctable[caption = List of measurements as extracted from acoustics.,
label = t:dur-measures,
width=\textwidth,
star
]{ll>{\raggedright}p{9cm}}{}{
\FL
\textbf{landmark}               &                  & \textbf{criteria}                                                                                    \ML
vowel onset           & (V1 onset)         & appearance of higher formants in the spectrogram following the burst of /p/ (C1)            \NN
vowel offset          & (V1 offset)        & disappearance of the higher formants in the spectrogram preceding the target consonant (C2) \NN
consonant onset       & (C2 onset)         & corresponds to V1 offset                                                                    \NN
closure onset         & (C2 closure onset) & corresponds to V1 offset                                                                    \NN
consonant offset      & (C2 offset)        & appearance of higher formants of the vowel following C2 (V2); corresponds to V2 onset                                \NN
consonant release & (C1/C2 release)         & automatic detection + manual correction \citep{ananthapadmanabha2014}                                           \LL
}

The audio recordings were exported from AAA in \texttt{.wav} format for
further processing. A forced aligned transcription was accomplished
through the SPeech Phonetisation Alignment and Syllabification software
(SPPAS) \citep{bigi2015}. The outcome of the automatic annotation was
manually corrected when necessary, according to the criteria in
\Cref{t:dur-measures}. The releases of C1 and C2 were detected
automatically by means of a Praat scripting implementation of the
algorithm described in \citet{ananthapadmanabha2014}. The durations in
milliseconds of the following intervals were extracted from the
annotated acoustic landmarks with Praat scripting: sentence duration,
word duration, vowel duration (V1 onset to V1 offset), consonant closure
duration (V1 offset to C2 burst), and Release-to-Release duration (RR
duration, C1 release to C2 release). \Cref{f:segmentation} shows an
example of the segmentation of /pata/ (a) and /pada/ (b) from an Italian
speaker. Syllable rate (syllables per second) was used as a proxy to
speech rate \citep{plug2018} for duration normalisation, and was
calculated as the number of syllables divided by the duration of the
sentence (8 syllables in Italian, 6 in Polish).

\begin{figure*}
  \figline{
    \fig{img/annotation-1.pdf}{0.5\textwidth}{(a)}
    \fig{img/annotation-2.pdf}{0.5\textwidth}{(b)}
  }
  \caption{Segmentation example.}
  \label{f:segmentation}
\end{figure*}

\hypertarget{statistical-analysis}{%
\subsection{Statistical analysis}\label{statistical-analysis}}

Given the exploratory nature of the study, all statistical analyses
reported here are to be considered data-driven or hypothesis-generating
rather than hypothesis-driven \citep{kerr1998, gelman2013}. The
durational measurements were analysed with linear mixed-effects models
using \texttt{lme4} v1.1-17 in R v3.5.0
\citep{r-core-team2018, bates2015}. All factors were coded with
treatment contrasts. \emph{P}-values for the individual terms were
obtained with \texttt{lmerTest} v3.0-1, which uses the Satterthwaite's
approximation to degrees of freedom \citep{kuznetsova2017, luke2017}.
\emph{P}-values below the alpha level 0.05 were considered significant.

Bayes factors were used to specifically test the null hypotheses that
word and RR duration are not affected by C2 voicing (i.e., the effect of
C2 voicing on duration is \texttt{0}). For each set of null/alternative
hypotheses, a full model (with the predictor of interest) and a null
model (excluding it) were fitted separately using Maximum Likelihood
estimation \citep[p.~34]{bates2015}. The BIC approximation was then used
to obtain Bayes factors
\citep{raftery1995, raftery1999, wagenmakers2007, jarosz2014}. The
approximation is calculated according to the equation in \ref{eq:bayes}
\citep[p.~796]{wagenmakers2007}.

\begin{equation}
\label{eq:bayes}
BF_{01} \approx exp(\Delta{}BIC_{10}/2)
\end{equation}

where \(\Delta{}BIC_{10} = BIC_1 - BIC_0\), \(BIC_1\) is the BIC of the
full model, and \(BIC_0\) is the BIC of the null model. Values of
\(BF_{01} > 1\) indicate a preference of H\textsubscript{0} over
H\textsubscript{1}. The interpretation of the Bayes factors follows the
recommendations in \citet[p.~139]{raftery1995}.

The extracted measurements were filtered before statistical analysis.
Measures of vowel duration, closure duration, word duration, and RR
duration that are 3 standard deviations lower or higher than the
respective means were excluded from the final dataset. This operation
yields a total of 920 tokens of vowel and closure durations, 1176 tokens
of word duration, and 848 tokens of RR duration.

\hypertarget{results}{%
\section{Results}\label{results}}

The following sections report the results of the study in relation to
the durations of vowels, consonant closure, word, and the Release to
Release interval. When discussing the output of statistical modelling,
only the relevant predictors and interactions will be presented. For the
full output of the models and \emph{p}-values, see \Cref{a:stats}.

\hypertarget{vowel-duration}{%
\subsection{Vowel duration}\label{vowel-duration}}

\begin{figure}
\includegraphics{2018-jasa_files/figure-latex/vowels-plot-1} \caption{Vowel duration in Italian and Polish.}\label{f:vowels-plot}
\end{figure}

\Cref{f:vowels-plot} shows boxplots and the raw data of vowel duration
in Italian (on the left) and Polish (on the right) for the three vowels
/a, o, u/. Vowel tend to be longer when followed by a voiced stop both
in Italian and Polish. The effect appears to be greater in Italian than
in Polish, especially for the vowels /a/ and /o/ There is no clear
effect of C2 voicing in /u/ in Italian, but the effect is discernible in
Polish /u/.

A linear mixed-effects model with vowel duration as the outcome variable
was fitted with the following predictors: fixed effects for C2 voicing
(voiceless, voiced), C2 place of articulation (coronal, velar), vowel
(a, o, u), language (Italian, Polish), and speech rate (as syllables per
second); by-speaker and by-word random intercept with by-speaker random
slopes for C2 voicing. All possible interactions between C2 voicing,
vowel, and language were included. The following terms are significant
according to \emph{t}-tests with Satterthwaite's approximation to
degrees of freedom: C2 voicing, vowel, language, and speech rate. Only
the interaction between C2 voicing and vowel is significant. Vowels are
19 ms longer (se = 4.4) when followed by a voiced stop (C2 voicing). The
effect of C2 voicing is smaller with /u/ (around 5 ms, \(\hat{\beta}\) =
-14.4 ms, se = 6). Polish has on average shorter vowels than Italian
(\(\hat{\beta}\) = -28 ms, se = 8), and the effect of voicing is
estimated to be about 11 ms (although recall that the interaction
between language and C2 voicing is deemed not significant). Speech rate
has unsurprisingly a negative effect on vowel duration, such that faster
rates correlate with shorter vowel durations (\(\hat{\beta}\) = -15 ms,
se = 1).

\hypertarget{consonant-closure-duration}{%
\subsection{Consonant closure
duration}\label{consonant-closure-duration}}

\begin{figure}
\includegraphics{2018-jasa_files/figure-latex/closure-plot-1} \caption{Stop closure duration in Italian and Polish.}\label{f:closure-plot}
\end{figure}

\Cref{f:closure-plot} illustrates stop closure durations with boxplots
and individual raw data points. A pattern opposite to that with vowel
duration can be noticed: Closure duration is shorter for voiced than for
voiceless stops. The same model specification as with vowel duration has
been fitted with consonant closure durations as the outcome variable. C2
voicing, C2 place, and speech rate are significant. Stop closure is 16.5
ms shorter (se = 3) if the stop is voiced and 3.5 ms longer (se = 1.5)
if velar. Finally, faster speech rates correlate with shorter closure
durations (\(\hat{\beta}\) = -8.5 ms, se = 1 ms).

\hypertarget{vowel-and-closure-duration}{%
\subsection{Vowel and closure
duration}\label{vowel-and-closure-duration}}

\begin{figure}
\includegraphics{2018-jasa_files/figure-latex/vow-clo-plot-1} \caption{Linear regression of closure and vowel duration per vowel.}\label{f:vow-clo-plot}
\end{figure}

A model addressing the relationship between vowel and stop closure
duration was fitted with the following terms and interactions: vowel
duration as the outcome variable; as fixed effects, closure duration,
vowel, speech rate; an interaction between closure duration and vowel;
by-speaker and by-word random intercepts, and by-speaker random slopes
for C2 voicing. Closure duration has a significant effect on vowel
duration (\(\hat{\beta}\) = -0.15 ms, se = 0.06 ms). The effect with /u/
is greater than with /a/ and /o/ (\(\hat{\beta}\) = -0.35 ms, se = 0.06
ms). In general, closure duration is inversely correlated with vowel
duration. However such correlation is quite weak. A 1 ms increase in
closure duration corresponds to a 0.2--0.5 ms decrease in vowel
duration. \Cref{f:vow-clo-plot} shows for each of /a, o, u/ the
individual data points and the regression lines with confidence
intervals extracted from the linear model.

\hypertarget{word-duration}{%
\subsection{Word duration}\label{word-duration}}

The following full and null models were fitted to test for the effect of
C2 voicing on word duration. The full model has the following fixed
effects: C2 voicing, C2 place, vowel, speech rate, and language. The
model also includes by-speaker and by-word random intercepts, and a
by-speaker random slope for C2 voicing. The null model excludes the
fixed effect of C2 voicing. The Bayes factor of the null model against
the full model is 24. Thus, the null model (in which the effect of C2
voicing is 0) is 24 times more likely under the observed data than the
full model. This indicates that there is strong evidence for word
duration not being affected by C2 voicing.

\hypertarget{release-to-release-interval-rr-duration}{%
\subsection{Release to Release interval (RR)
duration}\label{release-to-release-interval-rr-duration}}

\begin{figure}
\centering
\includegraphics{2018-jasa_files/figure-latex/rr-plot-1.pdf}
\caption{Release to Release interval duration.}
\end{figure}

The models specifications for the Release to Release duration are the
same as for word duration. The Bayes factor of the null model against
the full model is 23, which means that the null model (without C2
voicing) is 23 times more likely than the full model. The data suggests
there is positive evidence that duration of the RR interval is not
affected by C2 voicing.

\hypertarget{discussion}{%
\section{Discussion}\label{discussion}}

In this exploratory study of Italian and Polish acoustic data, I found
that the duration of interval between the releases of two consecutive
consonants is insensitive to the phonological voicing of the second
consonant. The difference in vowel duration before voiceless vs.~voiced
stops derives from differences in placement of the closure onset within
the fixed interval between the releases of C1 and C2.

\hypertarget{gestural-alignment}{%
\subsection{Gestural alignment}\label{gestural-alignment}}

\label{s:gestural}

\begin{figure}
  \centering
  \includegraphics{img/gorganisation.pdf}
  \caption{Gestural organisation patterns for onsets (a), codas (b), heterosyllabic onsets (c). See \Cref{s:gestural} for details. Based on \citet{marin2010}.}
  \label{f:gorganisation}
\end{figure}

According to the coupled oscillator model of syllabic structure
\citep{browman1988, browman2000, goldstein2006, goldstein2014},
articulatory gestures can be timed according to two coupling modes:
in-phase (synchronous) mode, by which two gestures start in synchrony,
or anti-phase (sequential) mode, in which one gesture starts when the
preceding one has reached its target. \citet{marin2010} showed that
onset consonants in American English are in-phase with the vowel nucleus
and anti-phase with each other. Such phasing pattern establishes a
stable relationship between the centre of the consonant or consonant
cluster and the following vowel. Independent of the number of onset
consonants, the midpoint of the onset, the so-called `C-centre', is
maintained at a fixed distance from the vowel, such that increasing
number of consonants in the onset does not change the C-centre/vowel
distance (\Cref{f:gorganisation}(a)). On the other hand, coda consonants
are timed anti-phase with the preceding vowel and between themselves.
Stability in codas is seen in the lag between the vowel and the
left-most edge of the coda, which is not affected by the number of coda
consonants (\Cref{f:gorganisation}(b)). Other studies found further
evidence for the synchronous and sequential coupling modes (see
extensive review in \citet{marin2010} and \citet{marin2014}), although
the use of one mode over the other depends on the language and the
consonants under study.

Consonants can thus be said to follow either a C-centre organisation
pattern or a left-edge organisation pattern. In both cases, of course,
the pattern is relative to the tautosyllabic vowel (the following vowel
for onsets, the preceding vowel for codas). To the best of my knowledge,
no study has reported the timing of onset consonants relative to the
\emph{preceding} (heterosyllabic) vowel. The results from this acoustic
study on Italian and Polish are compatible with a right-edge
organisation pattern for onset consonants and preceding stressed vowels
\Cref{f:gorganisation}(c). The release of C2 (which is the onset of the
second syllable in CV́CV words)---which can be thought as the acoustic
parallel of the articulatory right edge of C2---is invariantly timed
relative to V1 (which is the nucleus of the first syllable).

A consequence of a right-edge organisation pattern of C2 relative to V1
in CV́CV words is that differences in C2 closure duration do not affect
the lag between V1 and the release of C2, as shown by the results of
this study. The invariance of the lag between the release of C1 and that
of C2 then can be seen to follow from the invariance in timing between,
on the one hand, C1 (which is always /p/ in this study) and V1, and, on
the other, between V1 and the right edge of C2.

A right-edge organisation account is compatible with findings from
electromyographic, x-ray microbeam, and ultrasonic data by,
respectively, \citet{raphael1975}, \citet{de-jong1991}, and
\citet{celata2018}. \citet{celata2018} show that vowels before
tautosyllabic clusters have the same duration as before heterosyllabic
clusters. However, vowels followed by geminates are shorter than when
followed by singletons, although from a syllabic structure point of view
geminates correspond to heterosyllabic clusters and singletons to
tautosyllabic clusters (i.e., V-final syllables followed by singletons
and tautosyllabic clusters are open, while those followed by geminates
and heterosyllabic clusters are closed). \citet{celata2018} argue that
these results corroborate a rhythmic account in which the relevant unit
is the rhythmic syllable, i.e.~the VC(C) sequence (independent of the
traditional syllabic structure), which is kept constant. Such view
reflects a gestural timing view in which the timing of the right edge of
the consonant is held constant relative to the vowel.

\citet{de-jong1991} reports that the closing gesture of voiceless stops
(following stressed vowels) is faster than that of voiced stops, and
that also it is timed earlier with respect to the opening gesture of the
stressed vowel. According to \citet{de-jong1991}, the differences in
vowel duration are driven by the timing of the consonantal closing
gesture relative to the vocalic opening gesture (also see
\citealt{hertrich1997}). Moreover, the data in \citet{de-jong1991} show
that the final portion of the vocalic opening gesture is prolonged
before voiced stops. This finding corresponds to what
\citet{raphael1975} reported based on electromyographic data. The
electromyographic signal corresponding to the vocalic gesture reaches
its plateaux at the same time in the voiceless and voiced context, but
the plateaux is held for longer in the case of vowels followed by voiced
stops, indicating that muscular activation is kept for longer.

These studies taken together, plus the results from this study, bring
evidence to the view that two factors contribute to the difference in
vowel duration observed before consonants varying in their voicing
specification. These two factors are: (1) the right-edge alignment of
coda consonants following stressed vowels relative to the latter, and
(2) the differential timing of the closing gesture onset for voiceless
vs.~voiced stops. These two factors together can be synthesised into a
compensatory temporal adjustment account, in which the fixed interval is
generated by factor (1) and the temporal adjustment is brought about by
factor (2).

\hypertarget{limitations-and-future-work}{%
\subsection{Limitations and future
work}\label{limitations-and-future-work}}

The generalisations reported in this paper strictly apply to disyllabic
words with stress on the first syllable. It is possible that the
organisation pattern found in this context does not occur in sequences
including an unstressed vowel. For example, it is known that the
difference in closure duration between voiceless and voiced stops is not
stable when the stops precede a stressed vowel, although the vowels
preceding the pre-stress stops have different durations
\citep{davis1989}. According to the gestural interpretation given here,
no differences in closure durations should correspond to no difference
in vowel durations. The constraints on experimental material brought
about by the use of ultrasound tongue imaging have been already
mentioned in \Cref{s:materials}. In the previous section I mention that
the invariance of the RR duration could be a consequence of the timing
of C2 rather than of a holistic CVC motor plan in which the RR interval
is held constant. Disambiguating between these two interpretations is
not possible based on the data from this study.

\hypertarget{conclusion}{%
\section{Conclusion}\label{conclusion}}

\begin{acknowledgments}
Thanks to...
\end{acknowledgments}

\appendix

\hypertarget{output-of-statistical-models}{%
\section{Output of statistical
models}\label{output-of-statistical-models}}

\label{a:stats}

\hypertarget{socio-linguistic-information-of-participants}{%
\section{Socio-linguistic information of
participants}\label{socio-linguistic-information-of-participants}}

\label{a:socioling}

\ctable[caption = Participants' sociolinguistic information.,
label = t:socio,
width=\textwidth,
star,
doinside = \footnotesize
]{llll>{\raggedright}p{5cm}lll}{}{
\FL
ID & Age & Sex    & Native L & Other Ls& City of birth & Spent most time in & > 6 mo \ML
it01 & 29 & Male   & Italian & English, Spanish                          & Verbania            & Verbania        & Yes \NN
it02 & 26 & Male   & Italian & Friulian, English, Ladin-Venetan          & Udine               & Tricesimo & Yes \NN
it03 & 28 & Female & Italian & English, German                           & Verbania            & Verbania        & No  \NN
it04 & 54 & Female & Italian & Calabrese                                 & Verbania            & Verbania        & No  \NN
it05 & 28 & Female & Italian & English                                   & Verbania            & Verbania        & No  \NN
it09 & 35 & Female & Italian & English                                   & Vignola & Vignola         & Yes \NN
it11 & 24 & Male   & Italian & English                                   & Monza               & Monza           & Yes \NN
it13 & 20 & Female & Italian & English, French, Arabic, Farsi            & Ancona              & Chiaravalle     & Yes \NN
it14 & 32 & Male & Italian & English, Spanish & Frosinone & Frosinone & Yes \NN
pl02 & 32 & Female & Polish  & English, Norwegian, French, German, Dutch & Koło                & Poznań          & Yes \NN
pl03 & 26 & Male   & Polish  & Russian, English, French, German          & Nowa Sol            & Poznań          & Yes \NN
pl04 & 34 & Female & Polish  & Spanish, English, French                  & Warsaw              & Warsaw          & No  \NN
pl05 & 42 & Male   & Polish  & English, French                           & Przasnysz           & Warsaw        & No  \NN
pl06 & 33 & Male   & Polish  & English                                   & Zgierz              & Zgierz          & Yes \NN
pl07 & 32 & Female & Polish  & English, Russian                          & Bielsk Podlaski     & Bielsk Podlaski & Yes \LL
}

%% before appendix (optional) and bibliography:
% \begin{acknowledgments}
%This research was supported by  ...
% \end{acknowledgments}

% -------------------------------------------------------------------------------------------------------------------
%   Appendix  (optional)

%\appendix
%\section{Appendix title}

%If only one appendix, please use
%\appendix*
%\section{Appendix title}


%=======================================================
%IMPORTANT

%Use \bibliography{<name of your .bib file>}+
%to make your bibliography with BibTeX.

%Once you have used BibTeX you
%should open the resulting .bbl file and cut and paste the entire contents
%into the end of your article.
%=======================================================

\bibliography{linguistics.bib}


\end{document}
