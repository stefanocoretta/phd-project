\documentclass[12pt,]{article}
\usepackage{lmodern}
\usepackage{amssymb,amsmath}
\usepackage{ifxetex,ifluatex}
\usepackage{fixltx2e} % provides \textsubscript
\ifnum 0\ifxetex 1\fi\ifluatex 1\fi=0 % if pdftex
  \usepackage[T1]{fontenc}
  \usepackage[utf8]{inputenc}
\else % if luatex or xelatex
  \ifxetex
    \usepackage{mathspec}
  \else
    \usepackage{fontspec}
  \fi
  \defaultfontfeatures{Ligatures=TeX,Scale=MatchLowercase}
    \setmainfont[]{Charis SIL}
\fi
% use upquote if available, for straight quotes in verbatim environments
\IfFileExists{upquote.sty}{\usepackage{upquote}}{}
% use microtype if available
\IfFileExists{microtype.sty}{%
\usepackage{microtype}
\UseMicrotypeSet[protrusion]{basicmath} % disable protrusion for tt fonts
}{}
\usepackage[margin=1in]{geometry}
\usepackage{hyperref}
\hypersetup{unicode=true,
            pdftitle={Vowel duration, voicing duration, and vowel height: Acoustic and articulatory data from Italian},
            pdfauthor={Stefano Coretta},
            pdfborder={0 0 0},
            breaklinks=true}
\urlstyle{same}  % don't use monospace font for urls
\usepackage{natbib}
\bibliographystyle{unified}
\usepackage{graphicx,grffile}
\makeatletter
\def\maxwidth{\ifdim\Gin@nat@width>\linewidth\linewidth\else\Gin@nat@width\fi}
\def\maxheight{\ifdim\Gin@nat@height>\textheight\textheight\else\Gin@nat@height\fi}
\makeatother
% Scale images if necessary, so that they will not overflow the page
% margins by default, and it is still possible to overwrite the defaults
% using explicit options in \includegraphics[width, height, ...]{}
\setkeys{Gin}{width=\maxwidth,height=\maxheight,keepaspectratio}
\IfFileExists{parskip.sty}{%
\usepackage{parskip}
}{% else
\setlength{\parindent}{0pt}
\setlength{\parskip}{6pt plus 2pt minus 1pt}
}
\setlength{\emergencystretch}{3em}  % prevent overfull lines
\providecommand{\tightlist}{%
  \setlength{\itemsep}{0pt}\setlength{\parskip}{0pt}}
\setcounter{secnumdepth}{5}
% Redefines (sub)paragraphs to behave more like sections
\ifx\paragraph\undefined\else
\let\oldparagraph\paragraph
\renewcommand{\paragraph}[1]{\oldparagraph{#1}\mbox{}}
\fi
\ifx\subparagraph\undefined\else
\let\oldsubparagraph\subparagraph
\renewcommand{\subparagraph}[1]{\oldsubparagraph{#1}\mbox{}}
\fi

%%% Use protect on footnotes to avoid problems with footnotes in titles
\let\rmarkdownfootnote\footnote%
\def\footnote{\protect\rmarkdownfootnote}

%%% Change title format to be more compact
\usepackage{titling}

% Create subtitle command for use in maketitle
\newcommand{\subtitle}[1]{
  \posttitle{
    \begin{center}\large#1\end{center}
    }
}

\setlength{\droptitle}{-2em}

  \title{Vowel duration, voicing duration, and vowel height: Acoustic and
articulatory data from Italian}
    \pretitle{\vspace{\droptitle}\centering\huge}
  \posttitle{\par}
    \author{Stefano Coretta}
    \preauthor{\centering\large\emph}
  \postauthor{\par}
      \predate{\centering\large\emph}
  \postdate{\par}
    \date{29/11/2018}

\usepackage{cleveref}
\usepackage{ctable}
\usepackage{enumerate}
\usepackage{lineno}
\linenumbers

\begin{document}
\maketitle

\section{Methods}\label{methods}

\subsection{Participants}\label{participants}

Participants for this study were recruited in the province of
Verbano-Cusio-Ossola (VCO), Piedmont, Northern Italy. Inclusion in the
study was dependent on the participant being a native speaker of
Italian, of adult age (18 yo or older), with no reported hearing or
speaking disorders, and normal or corrected to normal vision. The target
sample size was set up to be 30, to be reached within a maximum of three
months from the start of data collection. A total of 19 participants
were recorded within the established timeframe. Fifteen were female and
the mean age was 40 (range 18--59). Seventeen participants lived all or
most of their lives in the VCO province. IT03 is from a neighbouring
province (Varese), and the variety of Italian he speaks is
indistinguishable from that of the VCO speakers. IT07 is from Sardinia,
but she spent the 10 years ahead of recording in the VCO province. Her
variety of Italian is different from the one of the other speakers, but
excluding her data from the analyses produced negligible or no
differences in the results, so they have been kept. Ethical clearance
for this study was obtained from the University of Manchester (REF
2016-0099-76). The speakers were not paid for their participation.

\subsection{Equipment}\label{equipment}

Synchronised audio and electroglottographic signals were obtained with a
Glottal Enterprises EG2-PCX2 electroglottograph and a RØDE Lavalier
microphone, at a sample rate of 44100 Hz (16-bit; downsampled to 22050
Hz for analysis). The acquisition of the signals was controlled with
Audacity (2.2.2) running on a MacBook Pro (13-inch, Mid 2014, \ldots{})
with macOS \ldots{} . The EGG electrodes were placed around the
participant's neck with a velcro strap, at the height of the glottis,
one on each side of the thyroid cartilage. Accuracy of the electrode
vertical placement was assessed by checking the EGG unit placement
monitor light (green indicates correct placement). The low frequency
(LF) limit selector was set on 20 Hz. The microphone was clipped on the
participants cloths, about 20 cm from their mouth.

\subsection{Materials}\label{materials}

To test the influence of vowel height on vowel and voicing duration, a
set of target words with a stressed vowel flanked by two voiceless tops.
The words have a CVCV structure, with lexical stress on the first
syllable, and where C\textsubscript{1} = /p, t, k/, V\textsubscript{1} =
/a, e, ɔ, i, u/, C\textsubscript{2} = /p, t, k/, and V\textsubscript{2}
= /o/. All possible combinations were used, except for \emph{peto} and
\emph{caco} (which are bad words). The full list of target words is
given in \ldots{} The traditional description of the Italian vocalic
system includes both mid-low /ɛ, ɔ/ and a mid-high vowels /o, ɔ/. While
these vowels are usually ascribed to different phonemes
\citep{kramer2009}, their functional load and actual phonetic
realisation are less straightforward \citep{renwick2016}. Although some
minimal pairs exists both for the front and the back contrast, stressed
open syllables in Italian tend to have the mid-high front vowel /e/ and
the mid-low back vowel /ɔ/. Note also that penultimate stressed open
syllables in Italian have a phonetically long vowel, for example /pɔko/
= {[}pɔːko{]} \citep{renwick2016}.

These words were embedded in 4 frame sentences, adapted from
\citet{renwick2016}: \emph{Scrivete X sul foglio} `Write (you pl.) X on
the sheet', \emph{Ha detto X sei volte} `S/he said X six times',
\emph{Ripete X da sempre} `S/he's being repeating X for ever',
\emph{Sentivo X di nuovo} `I heard X again'. In all the sentences, the
target word is preceded and followed by three syllables, where the
second syllable is stressed. Using different frames reduces possible
repetition effects, while allowing the acquisition of 4 word tokens per
speaker. All word/frame combinations were used, generating a total of 43
words × 4 frames = 172 observations per speaker.

\subsection{Procedure}\label{procedure}

Participants familiarised themselves with the materials prior to
recording and practiced reading them aloud. The stimuli sentences were
presented in a randomised order on the laptop screen used for recording,
through the software PsychoPy v1.90.3 \citet{peirce2009}. The
participants were given a change to take a break half the way through
the task, which lasted 8--10 minutes. At the end of the speaking task,
the participants filled a short sociolinguistic questionnaire. The whole
experiment session lasted 30 minutes.

\subsection{Data processing and
measurements}\label{data-processing-and-measurements}

The audio recordings were downsampled to 22050 Hz (16-bit) for
acoustical analysis. Noise removal with the Praat built-in function
\texttt{Remove\ noise} \citep{boersma2018} was applied to the audio
recordings of IT01, IT02, IT03, IT04, IT07, IT10, and IT12. An automatic
time-aligned transcription was performed with the SPeech Phonetisation
Alignment and Syllabification software (SPPAS, \citealt{bigi2015}), and
subsequently manually fixed according to the criteria in
\citet{machac2009}. In particular, vowel onset and offset were placed in
correspondence of the appearance and disappearance of higher formant
structure in the spectrogram. Only the following boundaries were
corrected to shorten processing time: sentence onset and offset, target
word onset and offset, target word segment boundaries.

\subsection{Statistical analysis}\label{statistical-analysis}

\subsection{Open Science statement}\label{open-science-statement}

\bibliography{linguistics}


\end{document}
