\documentclass[11pt,]{article}
\usepackage{lmodern}
\usepackage{amssymb,amsmath}
\usepackage{ifxetex,ifluatex}
\usepackage{fixltx2e} % provides \textsubscript
\ifnum 0\ifxetex 1\fi\ifluatex 1\fi=0 % if pdftex
  \usepackage[T1]{fontenc}
  \usepackage[utf8]{inputenc}
\else % if luatex or xelatex
  \ifxetex
    \usepackage{mathspec}
  \else
    \usepackage{fontspec}
  \fi
  \defaultfontfeatures{Ligatures=TeX,Scale=MatchLowercase}
    \setmainfont[]{DejaVu Sans}
\fi
% use upquote if available, for straight quotes in verbatim environments
\IfFileExists{upquote.sty}{\usepackage{upquote}}{}
% use microtype if available
\IfFileExists{microtype.sty}{%
\usepackage{microtype}
\UseMicrotypeSet[protrusion]{basicmath} % disable protrusion for tt fonts
}{}
\usepackage[margin=1in]{geometry}
\usepackage{hyperref}
\hypersetup{unicode=true,
            pdftitle={Pre-registration of a study on vowel duration, voicing duration, and vowel height in Italian},
            pdfauthor={Stefano Coretta},
            pdfborder={0 0 0},
            breaklinks=true}
\urlstyle{same}  % don't use monospace font for urls
\usepackage{natbib}
\bibliographystyle{unified.bst}
\usepackage{color}
\usepackage{fancyvrb}
\newcommand{\VerbBar}{|}
\newcommand{\VERB}{\Verb[commandchars=\\\{\}]}
\DefineVerbatimEnvironment{Highlighting}{Verbatim}{commandchars=\\\{\}}
% Add ',fontsize=\small' for more characters per line
\usepackage{framed}
\definecolor{shadecolor}{RGB}{248,248,248}
\newenvironment{Shaded}{\begin{snugshade}}{\end{snugshade}}
\newcommand{\AlertTok}[1]{\textcolor[rgb]{0.94,0.16,0.16}{#1}}
\newcommand{\AnnotationTok}[1]{\textcolor[rgb]{0.56,0.35,0.01}{\textbf{\textit{#1}}}}
\newcommand{\AttributeTok}[1]{\textcolor[rgb]{0.77,0.63,0.00}{#1}}
\newcommand{\BaseNTok}[1]{\textcolor[rgb]{0.00,0.00,0.81}{#1}}
\newcommand{\BuiltInTok}[1]{#1}
\newcommand{\CharTok}[1]{\textcolor[rgb]{0.31,0.60,0.02}{#1}}
\newcommand{\CommentTok}[1]{\textcolor[rgb]{0.56,0.35,0.01}{\textit{#1}}}
\newcommand{\CommentVarTok}[1]{\textcolor[rgb]{0.56,0.35,0.01}{\textbf{\textit{#1}}}}
\newcommand{\ConstantTok}[1]{\textcolor[rgb]{0.00,0.00,0.00}{#1}}
\newcommand{\ControlFlowTok}[1]{\textcolor[rgb]{0.13,0.29,0.53}{\textbf{#1}}}
\newcommand{\DataTypeTok}[1]{\textcolor[rgb]{0.13,0.29,0.53}{#1}}
\newcommand{\DecValTok}[1]{\textcolor[rgb]{0.00,0.00,0.81}{#1}}
\newcommand{\DocumentationTok}[1]{\textcolor[rgb]{0.56,0.35,0.01}{\textbf{\textit{#1}}}}
\newcommand{\ErrorTok}[1]{\textcolor[rgb]{0.64,0.00,0.00}{\textbf{#1}}}
\newcommand{\ExtensionTok}[1]{#1}
\newcommand{\FloatTok}[1]{\textcolor[rgb]{0.00,0.00,0.81}{#1}}
\newcommand{\FunctionTok}[1]{\textcolor[rgb]{0.00,0.00,0.00}{#1}}
\newcommand{\ImportTok}[1]{#1}
\newcommand{\InformationTok}[1]{\textcolor[rgb]{0.56,0.35,0.01}{\textbf{\textit{#1}}}}
\newcommand{\KeywordTok}[1]{\textcolor[rgb]{0.13,0.29,0.53}{\textbf{#1}}}
\newcommand{\NormalTok}[1]{#1}
\newcommand{\OperatorTok}[1]{\textcolor[rgb]{0.81,0.36,0.00}{\textbf{#1}}}
\newcommand{\OtherTok}[1]{\textcolor[rgb]{0.56,0.35,0.01}{#1}}
\newcommand{\PreprocessorTok}[1]{\textcolor[rgb]{0.56,0.35,0.01}{\textit{#1}}}
\newcommand{\RegionMarkerTok}[1]{#1}
\newcommand{\SpecialCharTok}[1]{\textcolor[rgb]{0.00,0.00,0.00}{#1}}
\newcommand{\SpecialStringTok}[1]{\textcolor[rgb]{0.31,0.60,0.02}{#1}}
\newcommand{\StringTok}[1]{\textcolor[rgb]{0.31,0.60,0.02}{#1}}
\newcommand{\VariableTok}[1]{\textcolor[rgb]{0.00,0.00,0.00}{#1}}
\newcommand{\VerbatimStringTok}[1]{\textcolor[rgb]{0.31,0.60,0.02}{#1}}
\newcommand{\WarningTok}[1]{\textcolor[rgb]{0.56,0.35,0.01}{\textbf{\textit{#1}}}}
\usepackage{graphicx,grffile}
\makeatletter
\def\maxwidth{\ifdim\Gin@nat@width>\linewidth\linewidth\else\Gin@nat@width\fi}
\def\maxheight{\ifdim\Gin@nat@height>\textheight\textheight\else\Gin@nat@height\fi}
\makeatother
% Scale images if necessary, so that they will not overflow the page
% margins by default, and it is still possible to overwrite the defaults
% using explicit options in \includegraphics[width, height, ...]{}
\setkeys{Gin}{width=\maxwidth,height=\maxheight,keepaspectratio}
\IfFileExists{parskip.sty}{%
\usepackage{parskip}
}{% else
\setlength{\parindent}{0pt}
\setlength{\parskip}{6pt plus 2pt minus 1pt}
}
\setlength{\emergencystretch}{3em}  % prevent overfull lines
\providecommand{\tightlist}{%
  \setlength{\itemsep}{0pt}\setlength{\parskip}{0pt}}
\setcounter{secnumdepth}{5}
% Redefines (sub)paragraphs to behave more like sections
\ifx\paragraph\undefined\else
\let\oldparagraph\paragraph
\renewcommand{\paragraph}[1]{\oldparagraph{#1}\mbox{}}
\fi
\ifx\subparagraph\undefined\else
\let\oldsubparagraph\subparagraph
\renewcommand{\subparagraph}[1]{\oldsubparagraph{#1}\mbox{}}
\fi

%%% Use protect on footnotes to avoid problems with footnotes in titles
\let\rmarkdownfootnote\footnote%
\def\footnote{\protect\rmarkdownfootnote}

%%% Change title format to be more compact
\usepackage{titling}

% Create subtitle command for use in maketitle
\newcommand{\subtitle}[1]{
  \posttitle{
    \begin{center}\large#1\end{center}
    }
}

\setlength{\droptitle}{-2em}

  \title{Pre-registration of a study on vowel duration, voicing duration, and
vowel height in Italian}
    \pretitle{\vspace{\droptitle}\centering\huge}
  \posttitle{\par}
    \author{Stefano Coretta}
    \preauthor{\centering\large\emph}
  \postauthor{\par}
      \predate{\centering\large\emph}
  \postdate{\par}
    \date{03/07/2018}

\usepackage{cleveref}

\begin{document}
\maketitle

\hypertarget{study-information}{%
\section{Study information}\label{study-information}}

\hypertarget{title}{%
\subsection{Title}\label{title}}

Vowel duration, voicing duration, and vowel height: Acoustic and
articulatory data from Italian.

\hypertarget{authorship}{%
\subsection{Authorship}\label{authorship}}

Stefano Coretta (The University of Manchester).

\hypertarget{research-questions}{%
\subsection{Research questions}\label{research-questions}}

Vowel height has been shown to correlate with vowel duration, such that
high vowels are shorter than low vowels \citep[see][ and
\citet{esposito2002} and references therein]{toivonen2015}. Moreover,
data from Italian shows that the duration of the C1 Release to Vowel
Offset interval (RVoffT) in VCV́C words is affected by vowel height in
the same direction: higher vowels have a shorter RVoffT
\citep{esposito2002}. Less attention has been given to the duration of
the voiced interval which includes the vowel. In the context of two
flanking voiceless stops, like in \emph{pata}, it is possible to measure
the duration of the portion with voicing (vocal fold vibration) flanked
by the voiceless stops. This allows us to answer the following question:

\begin{itemize}
\tightlist
\item
  Q1: Is the duration of the voicing interval between two voiceless
  stops in CV́CV words affected by vowel height?
\end{itemize}

\hypertarget{hypotheses}{%
\subsection{Hypotheses}\label{hypotheses}}

In the following hypotheses and subsequent section, `vowel' refers to
the stressed vowel in CV́CV words. In relation to Q1, the following null
and alternative hypotheses will be tested:

\begin{itemize}
\tightlist
\item
  H1\textsubscript{0}: The duration of the voiced interval is not
  affected by vowel height.
\item
  H1\textsubscript{1}: The duration of the voiced interval is shorter
  for higher than for lower vowels.
\end{itemize}

As an extension of \citet{esposito2002} (which is based on VCV́C words),
the following hypotheses will also be tested:

\begin{itemize}
\tightlist
\item
  H2: Higher vowels are shorter than lower vowels.
\item
  H3: VOT is shorter in higher than in lower vowels.
\item
  H4: RVoffT is shorter for higher than for lower vowels.
\end{itemize}

\hypertarget{sampling-plan}{%
\section{Sampling plan}\label{sampling-plan}}

\hypertarget{existing-data}{%
\subsection{Existing data}\label{existing-data}}

\textbf{Registration prior to creation of data}: As of the date of
submission of this research plan for preregistration, the data have not
yet been collected, created, or realised.

\hypertarget{explanation-of-existing-data}{%
\subsection{Explanation of existing
data}\label{explanation-of-existing-data}}

NA.

\hypertarget{data-collection-procedures}{%
\subsection{Data collection
procedures}\label{data-collection-procedures}}

\hypertarget{participants}{%
\subsubsection{Participants}\label{participants}}

Inclusion rule: Native speakers of Italian, from the VCO province
(Italy), 18+ yo, with no reported hearing or speaking disorders, with
normal or corrected to normal vision.

\hypertarget{procedure}{%
\subsubsection{Procedure}\label{procedure}}

The participants will be recorded while reading sentences with CV́CV
target words (see \Cref{s:other} for the list of target words and
\Cref{s:manipulated} for the list of frame sentences) presented on a
computer screen with PsychoPy \citep{peirce2009}, in a quiet room in
Verbania (Italy). Time-synchronised audio and electroglottographic data
will be collected using a Glottal Enterprises EG2-PCX2
electroglottograph and a RØDE Lavalier microphone, at a sample rate of
44100 Hz (16-bit; downsampled to 22050 Hz for analysis). The acquisition
of the signals will be controlled with Audacity running on a MacBook Pro
(Retina, 13-inch, Mid 2014). The recordings will be subject to force
alignment with SPPAS \citep{bigi2015} for analysis in Praat
\citep{boersma2018}.

\hypertarget{sample-size}{%
\subsection{Sample size}\label{sample-size}}

30 participants, 43 words, 4 repetitions. Grand total: 5,160
observations (= 172 tokens per participant * 30 participants).

\hypertarget{sample-size-rationale}{%
\subsection{Sample size rationale}\label{sample-size-rationale}}

\citet{brysbaert2018} suggest to have around 1600 observations per
condition (e.g., 40 participants, 40 stimuli). Given the limited time
available for this study, 30 participants seems a more reasonable
target. According to a power analysis with simulated data (mean and
standard deviations based on a pilot study) using \texttt{simr}
\citep{green2016}, 12 subjects * 100 tokens per subject are sufficient
to detect a difference of 5 ms at 80\% of power. See \texttt{power.Rmd}
for the code.

\hypertarget{stopping-rule}{%
\subsection{Stopping rule}\label{stopping-rule}}

Data collection will be terminated earlier if the 30 participants target
hasn't been reached by the end of September 2018.

\hypertarget{variables}{%
\section{Variables}\label{variables}}

\hypertarget{manipulated-variables}{%
\subsection{Manipulated variables}\label{manipulated-variables}}

\label{s:manipulated}

\begin{itemize}
\tightlist
\item
  \textbf{Vowel height}: high (/i, u/), mid-high (/e/), mid-low (/ɔ/),
  low (/a/).
\item
  \textbf{Place of articulation of C1}: labial, coronal, velar.
\item
  \textbf{Place of articulation of C2}: labial, coronal, velar.
\item
  \textbf{Frame sentence}: \emph{Scrivete X sul foglio}, \emph{Ha detto
  X sei volte}, \emph{Sentivo X di nuovo}, \emph{Ripete X da sempre}.
\end{itemize}

See \Cref{s:other} for a list of word stimuli.

\hypertarget{measured-variables}{%
\subsection{Measured variables}\label{measured-variables}}

\begin{itemize}
\tightlist
\item
  From the acoustic signal:

  \begin{itemize}
  \tightlist
  \item
    \textbf{Vowel duration}: from onset to offset of higher formant
    structure.
  \item
    \textbf{Release to Vowel Offset}: from release of C1 to V1 offset.
  \item
    \textbf{Speech rate}: calculated as syllables per second
    \citep[\texttt{n\ of\ syllables\ /\ sentence\ duration},][]{plug2018}.
  \end{itemize}
\item
  From the EGG signal:

  \begin{itemize}
  \tightlist
  \item
    \textbf{Duration of voiced interval}: from voice onset to voice
    offset.
  \end{itemize}
\item
  From both the acoustic and EGG signal

  \begin{itemize}
  \tightlist
  \item
    \textbf{Voice Onset Time}: from the release of C1 to voice onset.
  \end{itemize}
\end{itemize}

\hypertarget{indices}{%
\subsection{Indices}\label{indices}}

NA.

\hypertarget{design-plan}{%
\section{Design plan}\label{design-plan}}

\hypertarget{study-type}{%
\subsection{Study type}\label{study-type}}

Experiment---A researcher randomly assigns treatments to study subjects,
this includes field or lab experiments. This is also known as an
intervention experiment and includes randomized controlled trials.

\hypertarget{blinding}{%
\subsection{Blinding}\label{blinding}}

No blinding is involved in this study.

\hypertarget{study-design}{%
\subsection{Study design}\label{study-design}}

Repeated measures, mixed design.

\hypertarget{randomisation}{%
\subsection{Randomisation}\label{randomisation}}

The sentence stimuli will be randomised within participant by means of
the built-in randomisation procedure in PsychoPy \citep{peirce2009}.

\hypertarget{analysis-plan}{%
\section{Analysis plan}\label{analysis-plan}}

\hypertarget{statistical-models}{%
\subsection{Statistical models}\label{statistical-models}}

\label{s:stats}

Separate models will be fitted in \texttt{R} \citep{r-core-team2018} to
independently assess effects on vowel duration, voicing interval
duration, VOT, and RVoffT. \emph{P}-values (\(\alpha\) = 0.05) will be
obtained with the \texttt{lmerTest} package \citep{kuznetsova2017},
which uses the Satterthwaite's method of approximation to degrees of
freedom.

\begin{Shaded}
\begin{Highlighting}[]
\KeywordTok{library}\NormalTok{(lme4)}
\KeywordTok{library}\NormalTok{(lmerTest)}

\NormalTok{vowel_lm <-}\StringTok{ }\KeywordTok{lmer}\NormalTok{(}
\NormalTok{  vowel_duration }\OperatorTok{~}
\StringTok{    }\NormalTok{height }\OperatorTok{+}
\StringTok{    }\NormalTok{c2_place }\OperatorTok{+}
\StringTok{    }\NormalTok{height}\OperatorTok{:}\NormalTok{c2_place }\OperatorTok{+}
\StringTok{    }\NormalTok{speech_rate }\OperatorTok{+}
\StringTok{    }\NormalTok{(}\DecValTok{1}\OperatorTok{+}\NormalTok{height}\OperatorTok{|}\NormalTok{speaker) }\OperatorTok{+}
\StringTok{    }\NormalTok{(}\DecValTok{1}\OperatorTok{|}\NormalTok{item)}
\NormalTok{)}

\NormalTok{vot_lm <-}\StringTok{ }\KeywordTok{lmer}\NormalTok{(}
\NormalTok{  vot }\OperatorTok{~}
\StringTok{    }\NormalTok{height }\OperatorTok{+}
\StringTok{    }\NormalTok{c2_place }\OperatorTok{+}
\StringTok{    }\NormalTok{height}\OperatorTok{:}\NormalTok{c2_place }\OperatorTok{+}
\StringTok{    }\NormalTok{speech_rate }\OperatorTok{+}
\StringTok{    }\NormalTok{(}\DecValTok{1}\OperatorTok{+}\NormalTok{height}\OperatorTok{|}\NormalTok{speaker) }\OperatorTok{+}
\StringTok{    }\NormalTok{(}\DecValTok{1}\OperatorTok{|}\NormalTok{item)}
\NormalTok{)}

\NormalTok{rvofft_lm <-}\StringTok{ }\KeywordTok{lmer}\NormalTok{(}
\NormalTok{  rvofft }\OperatorTok{~}
\StringTok{    }\NormalTok{height }\OperatorTok{+}
\StringTok{    }\NormalTok{c2_place }\OperatorTok{+}
\StringTok{    }\NormalTok{height}\OperatorTok{:}\NormalTok{c2_place }\OperatorTok{+}
\StringTok{    }\NormalTok{speech_rate }\OperatorTok{+}
\StringTok{    }\NormalTok{(}\DecValTok{1}\OperatorTok{+}\NormalTok{height}\OperatorTok{|}\NormalTok{speaker) }\OperatorTok{+}
\StringTok{    }\NormalTok{(}\DecValTok{1}\OperatorTok{|}\NormalTok{item)}
\NormalTok{)}
\end{Highlighting}
\end{Shaded}

Since I wish to directly test both the null and the alternative
hypotheses concerning the duration of the voiced interval (H1), the BIC
approximation of Bayes factors will be used instead of \emph{p}-values
\citep{raftery1995, raftery1999, wagenmakers2007, jarosz2014}.

\begin{Shaded}
\begin{Highlighting}[]
\NormalTok{voicing_lm_}\DecValTok{1}\NormalTok{ <-}\StringTok{ }\NormalTok{lme4}\OperatorTok{::}\KeywordTok{lmer}\NormalTok{(}
\NormalTok{  voicing_duration }\OperatorTok{~}
\StringTok{    }\NormalTok{height }\OperatorTok{+}
\StringTok{    }\NormalTok{c2_place }\OperatorTok{+}
\StringTok{    }\NormalTok{speech_rate }\OperatorTok{+}
\StringTok{    }\NormalTok{(}\DecValTok{1}\OperatorTok{+}\NormalTok{height}\OperatorTok{|}\NormalTok{speaker) }\OperatorTok{+}
\StringTok{    }\NormalTok{(}\DecValTok{1}\OperatorTok{|}\NormalTok{item),}
  \DataTypeTok{REML =} \OtherTok{FALSE}
\NormalTok{)}

\NormalTok{voicing_lm_}\DecValTok{0}\NormalTok{ <-}\StringTok{ }\NormalTok{lme4}\OperatorTok{::}\KeywordTok{lmer}\NormalTok{(}
\NormalTok{  voicing_duration }\OperatorTok{~}
\StringTok{    }\CommentTok{# height +}
\StringTok{    }\NormalTok{c2_place }\OperatorTok{+}
\StringTok{    }\NormalTok{speech_rate }\OperatorTok{+}
\StringTok{    }\NormalTok{(}\DecValTok{1}\OperatorTok{+}\NormalTok{height}\OperatorTok{|}\NormalTok{speaker) }\OperatorTok{+}
\StringTok{    }\NormalTok{(}\DecValTok{1}\OperatorTok{|}\NormalTok{item),}
  \DataTypeTok{REML =} \OtherTok{FALSE}
\NormalTok{)}

\NormalTok{bf_}\DecValTok{01}\NormalTok{ <-}\StringTok{ }\KeywordTok{exp}\NormalTok{((}\KeywordTok{BIC}\NormalTok{(voicing_lm_}\DecValTok{1}\NormalTok{) }\OperatorTok{-}\StringTok{ }\KeywordTok{BIC}\NormalTok{(voicing_lm_}\DecValTok{0}\NormalTok{)) }\OperatorTok{/}\StringTok{ }\DecValTok{2}\NormalTok{)}
\end{Highlighting}
\end{Shaded}

\hypertarget{transformations}{%
\subsection{Transformations}\label{transformations}}

NA.

\hypertarget{follow-up-analyses}{%
\subsection{Follow-up analyses}\label{follow-up-analyses}}

NA.

\hypertarget{inference-criteria}{%
\subsection{Inference criteria}\label{inference-criteria}}

\emph{P}-values will be used for the effect of vowel height on vowel
duration, VOT, and RVoffT, with \(\alpha\) = 0.05. \emph{P}-values below
0.05 will be deemed significant. Bayes factors will be employed for the
effect of vowel height on voicing interval duration, since I am
interested in testing both the null and the alternative hypotheses. The
recommendations in \citet[139]{raftery1995} for the interpretation of
the Bayes factors will be followed.

\hypertarget{data-exclusion}{%
\subsection{Data exclusion}\label{data-exclusion}}

Tokens in which the consonant closure is entirely voiced will be
excluded.

\hypertarget{missing-data}{%
\subsection{Missing data}\label{missing-data}}

NA.

\hypertarget{exploratory-analysis}{%
\subsection{Exploratory analysis}\label{exploratory-analysis}}

An exploratory analysis will assess the effect of vowel identity and the
interaction between vowel height and vowel backness on vowel duration,
voicing duration, RVoffT. Also, the effects of vowel and place of
articulation on voicing during closure will be examined.

\hypertarget{script-optional}{%
\section{Script (Optional)}\label{script-optional}}

\hypertarget{analysis-scripts-optional}{%
\subsection{Analysis scripts
(Optional)}\label{analysis-scripts-optional}}

See \Cref{s:stats}.

\hypertarget{other}{%
\section{Other}\label{other}}

\label{s:other}

List of target words: * \emph{peto} `fart' and \emph{caco} `I shit' are
not included in the list of target words due to their meaning.

\begin{tabular}{lll}
papo & tapo & capo  \\
pepo & tepo & chepo \\
pipo & tipo & chipo \\
popo & topo & copo  \\
pupo & tupo & cupo  \\
pato & tato & cato  \\
*    & teto & cheto \\
pito & tito & chito \\
poto & toto & coto  \\
puto & tuto & cuto  \\
paco & taco & *     \\
peco & teco & checo \\
pico & tico & chico \\
poco & toco & coco  \\
puco & tuco & cuco
\end{tabular}

\bibliography{linguistics.bib}


\end{document}
